\chapter{Теория групп}

\section{Теорема Кэли}

\begin{theorem}[Кэли]
	$ G $ -- конечная группа, $ |G| = n $. Тогда $ \exist H < S_n : G \simeq H $
\end{theorem}

\begin{eg}
	$ G = Z_3 $ \\
    $ G = \set{g_1, g_2, g_3}, \qquad g_1 = \overline{0}, \quad g_2 = \overline{1}, \quad g_3 = \overline{2} $
    $$
    \begin{cases}
        g_1g_1 = \overline{0} + \overline{0} = \overline{0} = g_1 \\
        g_1g_2 = \overline{0} + \overline{1} = g_2 \\
        g_1g_3 = \overline{0} + \overline{2} = g_3
    \end{cases} $$
    $ g_1 $ будет соотетствовать тождественной перестановке
    $$
    \begin{cases}
    	g_2g_1 = g_2 \\
        g_2g_2 = g_3 \\
        g_2g_3 = g_1
    \end{cases} $$
    Получили, что $ g_2 $ соответствует перестановка $
    \begin{cases}
    	1 \to 2 \\
        2 \to 3 \\
        3 \to 1
    \end{cases} $, то есть $ g_2 \to
    \begin{pmatrix}
    	1 & 2 & 3 \\
        2 & 3 & 1
    \end{pmatrix} $ \\
    Аналогично, $ g_3 \to
    \begin{pmatrix}
    	1 & 2 & 3 \\
        3 & 1 & 2
    \end{pmatrix} $
    $$ H = \left\{
        \begin{pmatrix}
        	1 & 2 & 3 \\
            1 & 2 & 3
        \end{pmatrix},
        \begin{pmatrix}
        	1 & 2 & 3 \\
            2 & 3 & 1
        \end{pmatrix},
        \begin{pmatrix}
        	1 & 2 & 3 \\
            3 & 1 & 2
        \end{pmatrix}
        \right\} $$
\end{eg}

\begin{proof}
	Пронумеруем элементы группы $ g_1, ..., g_n $ \\
    Докажем, что $ \forall a \in G $ элементы $ ag_1, ..., ag_n $ -- это элементы $ g_1, ..., g_n $ в некотором порядке \\
    Достаточно проверить, что $ ag_1, ..., ag_n $ различны:
    $$ ag_i = ag_j \underimp{\text{свойство сокращения}} g_i = g_j \quad \contra $$
    Пусть $ \sigma \in S_n $:
    $$ ag_1 = g_{\sigma(1)}, \widedots[3em], ag_n = g_{\sigma(n)} $$
    Определим $ \vphi : G \to S_n : \vphi(a) = \sigma : \forall i \quad ag_i = g_{\sigma(i)} $ \\
    Проверим, что $ \vphi $ -- гомоморфизм: \\
    Пусть $ \vphi(a) = \sigma, \quad \vphi(b) = \tau $
    $$ (ab)g_i = a(bg_i) \underset{(\vphi(b) = \tau)}= ag_{\tau(i)} \underset{(\vphi(a) = \sigma)}= g_{\sigma(\tau(i))} = g_{(\sigma \circ \tau)(i)} \quad \forall i \implies \vphi(ab) = \sigma \circ \tau $$
    Положим $ H = \Img \vphi \implies H < S_n $ \\
    Проверим, что $ \vphi : G \to H $ -- изоморфизм:
    \begin{itemize}
    	\item $ \vphi $ сюръективна по определению $ \Img $
        \item Проверим инъективность: \\
        Пусть $ \vphi(a) = e_{S_n} =
        \begin{pmatrix}
        	1 & 2 & ... & n \\
            1 & 2 & ... & n
        \end{pmatrix} $
        $$ ag_1 = g_1, \widedots[3em], ag_n = g_n $$
        $$ ag_1 = eg_1 \implies a = e \implies \ker \vphi = \set{e} $$
    \end{itemize}
\end{proof}

\section{Действие группы на множество}

\begin{definition}
	Пусть $ G $ -- группа, $ M $ -- множество \\
    Говорят, что группа $ G $ действует (слева) на множество $ M $, если каждой паре $ g \in G, m \in M $ сопоставлен элемент $ g(m) \in M $, и при этом выполнены свойства:
    \begin{enumerate}
    	\item $ (gh)(m) = g \big( h(m) \big) \quad \forall g, h \in G, m \in M $
        \item $ e(m) = m \quad \forall m \in M $
    \end{enumerate}
\end{definition}

\begin{remark}
	Для $ g, h \in G, m \in M $ запись $ ghm $ означает $ (gh)(m) = g \big( h(m) \big) $
\end{remark}

\begin{exmpls}
	\item $ M $ -- множество, $ G $ -- множество биекций в себя
    \item $ D_n $ -- группа самосовмещений правильного $ n $-угольника
    \item $ M $ -- множество вершин $ n $-угольника, раскрашенных в несколько цветов, $ G $ -- группа самосовмещений
    \item $ M = G $, $ (m, g) \to gm $
    \item \label{eg:1} $ M = G $, действие -- сопряжение: $ (m, g) \to g^{-1}mg $
\end{exmpls}

\begin{definition}
	Введём на множестве $ M $ отношение эквивалентности $ \sim $: \\
    $ m \sim l $, если $ \exist g \in G : gm = l $
\end{definition}

\begin{definition}
	Классы эквивалентности по отношению $ \sim $ называются орбитами
\end{definition}

\begin{notation}
	$ \Orb(m), \quad Gm $
\end{notation}

\begin{undefthm}{Корректность}
	Проверим, что $ \sim $ является отношением эквивалентности:
    \begin{itemize}
    	\item $ em = m \implies m \sim m $
        \item $ m \sim e \implies l = gm \implies g^{-1}l = g^{-1}gm = m \implies l \sim m $
        \item $
        \begin{rcases}
        	m \sim l \implies l = gm \\
            l \sim k \implies k = hl
        \end{rcases} \implies k = h(gm) = (hg)(m) \implies m \sim k $
    \end{itemize}
\end{undefthm}

\begin{remark}
    Если говорят про орбиты в группе без множества, имеются в виду орбиты по сопряжению (см. пример \ref{eg:1})
\end{remark}

\begin{definition}
	Действие группы называется транзитивным, если есть ровно одна орбита
\end{definition}

\begin{definition}
    Стабилизатором элемента $ m \in M $ называется множество $ \set{g \in G | gm = m} $
\end{definition}

\begin{notation}
    $ \Fix(m), \quad G_m $
\end{notation}

\begin{definition}
    Фиксатором элемента $ g \in G $ называется множество $ \set{m \in M | gm = m} $
\end{definition}

\begin{notation}
    $ \Fix(g) $
\end{notation}

\begin{props}
    \item $ \St(m) < G \quad \forall m \in M $
    \begin{proof}
    	$$
        \begin{rcases}
        	gm = m \\
            hm = m
        \end{rcases} \implies (gh)m = g(hm) = gm = m $$
        $$ gm = m \implies g^{-1}m = g^{-1}gm = em = e $$
    \end{proof}
    \item Пусть $ G $ -- конечная группа. Тогда
    $$ |G| = |\Orb(m)| \cdot |\St(m)| \quad \forall m \in M $$
    \begin{proof}
    	Пусть $ k \define |\Orb(m)| $
        $$ \Orb(m) = \set{m_1, ..., m_k} $$
        $ g_1, ..., g_k \in G $ -- такие, что $ g_im = m $ \\
        Докажем, что $ g_1, ..., g_k $ принадлежат разным смежым классам по подгруппе $ \St(m) $ и являются представителями всех классов: \\
        \begin{itemize}
        	\item Различность: \\
            Пусть $ g_i $ и $ g_j $ в одном классе
            $$ g_i^{-1}g_j \in \St(m) \implies g_i^{-1}g_jm = m \implies g_ig_i^{-1}g_jm = g_im $$
        \end{itemize}

    \end{proof}
\end{props}
