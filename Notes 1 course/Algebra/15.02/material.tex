\chapter{Векторные пространства}

\section{Определение и примеры}

\begin{definition}
	$K$ -- поле, $V$ -- множество. Заданы опреации сложения на $V$ ($V \times V \to V$) и умножения на скаляр ($V \times K \to V$) \\
    Множество $V$ называется веторным пространством над $K$, если выполнены следующие свойства:
    \begin{enumerate}
    	\item $V$ -- абелева группа по сложению
        \item Дистрибутивность: $ a(u + v) = au + av, \qquad \forall a \in K, \quad u, v \in V $
        \item Дистрибутивность: $ (a + b)u = au + bu, \qquad \forall a, b \in K, \quad u \in V $
        \item Ассоциативность: $ a(bu) = (ab)u, \qquad \forall a, b \in K, \quad u \in U $
        \item $ 1 \cdot u = u, \qquad 1 \in K, \quad \forall u \in U $
    \end{enumerate}
    Элементы $V$ называют векторами, элементы $K$ -- скалярами
\end{definition}

\begin{exmpls}
	\item Геометрические векторы на плоскости -- векторное пространство над $\R$
    \item $\R^n$ -- векторное пространство над $\R$
    \item $K^n$, где $K$ -- поле -- векторное пространство над $K$
    \item $M_{m \times n}$ -- векторное пространство над $\R$
    \item $\Co$ -- векторное пространство над $\R$
    \item $K[x]$ -- векторное пространство над $K$
    \item Множество многочленов степени $\bm\le n$ -- векторное пространство над $K$
\end{exmpls}

\begin{props}
    \item $ 0 \cdot u = \vect{0}, \qquad \forall u \in V $
    \begin{proof}
    	$ 0 \cdot u = (0 + 0)u = 0 \cdot u + 0 \cdot u $ \\
        $ 0 = 0 \cdot u = 0 \cdot u + 0 \cdot u $ \\
        $ 0 = 0 \cdot u $
    \end{proof}
    \item $ a \cdot \vect{0} = \vect{0}, \qquad \forall a \in K $
    \item $a \cdot u = 0 \implies a = 0 $ или $ u = \vect{0} $
\end{props}

\section{Линейные комбинации и линейная зависимость}

\begin{definition}
	Линейной комбинацией векторов $u_1, ..., u_k \in V$ называется вектор
    $$ a_1u_1 + ... + a_ku_k, \quad a_i \in K $$
    $a_i$ -- коэффициенты
\end{definition}

\begin{definition}
	Линейная комбинация называется тривиальной, если все коэффициенты равны нулю
\end{definition}

\begin{definition}
	Векторы $u_i$ называются линейно зависимыми, если существует их нетривиальная линейная комбинация, равная нулю \\
    Иначе -- линейно независимые
\end{definition}

\begin{props}
	\item
    \begin{enumerate}
    	\item Векторы линейно зависимы $\iff$ один из векторов является ЛК остальных
        \begin{proof}
        	\hfill
            \begin{itemize}
            	\item $\impliedby$ \\
                Пусть $u_1$ -- ЛК, то есть $u_1 = a_2u_2 + ... + a_nu_n $ \\
                $ (-1)u_1 + a_2u_2 + ... + a_nu_n = 0 $ -- нетривиальная ЛК
                \item $\implies$ \\
                Пусть $a_1u_1 + ... + a_nu_n = 0 $ -- нетривиальная ЛК \\
                Пусть $a_1 \ne 0$
                $$ a_1 = -\frac{a_2}{u_1}u_2 - ... - \frac{a_n}{u_1}u_n $$
            \end{itemize}
        \end{proof}
        \item Если $u_1, ..., u_n$ ЛНЗ, а $u_1, ..., u_n, v$ ЛЗ, то $v$ является ЛК остальных
        \begin{proof}
            $u_1, ..., u_n, v$ ЛЗ $ \iff \exist a_1, ..., a_n $ (не все нули) $ : a_1u_1 + ... + a_nu_n + a_{n + 1}v = 0 $
            \begin{itemize}
            	\item Если $a_{n + 1} \ne 0 $, то можно выразить $v$
                \item Если $a_{n + 1} = 0 $, то: \\
                Не все $a_i$ равны 0, $a_1u_1 + ... + a_nu_n = 0 $ -- нетривиальная. Противоречие
            \end{itemize}
        \end{proof}
    \end{enumerate}
    \item
    \begin{enumerate}
    	\item Если к ЛЗ добавить несколько векторов, то она останется ЛЗ
        \item Если из ЛНЗ убрать несколько векторов, то она останется ЛНЗ
    \end{enumerate}
    \item
    \begin{enumerate}
        \item \label{en:1} $ c \ne 0 \in K$. \\
        $u_1, ..., u_n$ ЛЗ $\iff cu_1, ..., cu_n$ ЛЗ
        \item \label{en:2} $c \in K$ \\
        $u_1, ..., u_n$ ЛЗ $\iff u_1 + cu_2, u_2, ..., u_n $ ЛЗ
    \end{enumerate}
    \begin{proof}
    	$$ u_1' \define
        \begin{cases}
            cu_1 \quad \eqref{en:1} \\
            u_1 + cu_2 \quad \eqref{en:1}
        \end{cases} $$
        $$ u_1 =
        \begin{cases}
            \frac1cu_1' \quad \eqref{en:1} \\
            u_1' + (-c)u_2 \quad \eqref{en:1}
        \end{cases} $$
        Набор $u_1, ..., u_n$ получается из $u_1', u_2, ..., u_n$ преобразованием того же типа \\
        Достаточно доказать $\implies$
        \begin{enumerate}
        	\item Пусть $a_1u_1 + ... + a_nu_n = 0$, не все $a_i$ равны 0
            $$ \frac{a}c u_1' + a_2u_2 + ... + a_nu_n = 0, \quad \text{не все коэфф. равны 0} $$
            \item $a_1u_1 + a_2u_2 + ... + a_nu_n = 0$, не все $a_i$ равны 0
            $$ a_1u_1' + (a_2 - ca_1)u_2 + ... + a_nu_n = 0 $$
            $$ a_1(u_1 + cu_2) + ... $$
            Пусть $ a_1 = a_2 - ca_1 = a_3 = ... = a_n = 0 $
        \end{enumerate}
    \end{proof}
\end{props}

\begin{theorem}[линейная зависимость линейных комбинаций]
	Пусть $k > m$ и векторы $v_1, v_2, ..., v_k$ являются ЛК векторов $u_1, ..., u_m$ \\
    Тогда $v_1, ..., v_k$ ЛЗ
\end{theorem}

\begin{proof}
    \textbf{Индукция} по $m$
    \begin{itemize}
        \item \textbf{База.} $m = 1$ \\
        Есть вектор $u_1$. Все остальные -- его ЛК:
        $$ v_1 = a_1u_1, \qquad v_2 = a_2u_2, ... $$
        \begin{itemize}
        	\item $a_1 = 0 \implies v_1 = 0, \qquad 1 \cdot v_1 + 0 \cdot v_2 + 0 \cdot v_3 + ... = 0 $
            \item $a_1 \ne 0 $
            $$ v_2 = a_2u_1 = a_2 \cdot \frac{v_1}{a_1} $$
            $$ \frac{a_2}{a_1} v_1 + (-1) \cdot v_2 + 0 \cdot v_3 + ... = 0 $$
        \end{itemize}
        \item \textbf{Переход.} $ m - 1 \to m $
        $$ v_1 = a_{11}u_1 + a_{12}u_2 + ... + a_{1m}u_m $$
        $$ \widedots $$
        $$ v_k = a_{k1}u_! + a_{k2}u_2 + ... + a_{km}u_m $$
        Исключим $u_1$ из всех векторов, кроме первого:
        \begin{itemize}
            \item $a_{11} = a_{21} = ... = a_{k1} = 0 $ \\
            Применяем индукционное предположение к $v_1, ..., v_k$ и $u_2, ..., u_m$
            \item Пусть не все $a_{i1}$ равны нулю. НУО считаем, что $a_{i1} \ne 0$ \\
            При $i > 1$ положим $v_i' = v_i - \dfrac{a_{i1}}{a_{11}}v_1 $ \\
            Векторы $v_2', v_3', ..., v_k'$ являются ЛК $u_2, u_3, ..., u_m$ \\
            $ k - 1 > m - 1 $ \\
            По индукционному предположению, $v_2', ..., v_k'$ ЛЗ \\
            Добавим к этому набору $v_1$ (пользуемся свойством 2a) \\
            Воспользуемся свойством 3b: \\
            $ v_1, v_2, ..., v_k $ ЛЗ
        \end{itemize}
    \end{itemize}
\end{proof}

\section{Порождающие системы}

\begin{definition}
    Пусть $V$ -- векторное пространство \\
    Множество векторов $ \set{v_i}$ называется порождающим для $V$, если любой вектор $v \in V$ является ЛК некоторого конечного подмножества $\set{v_i}$
\end{definition}

\begin{definition}
	Если у $V$ есть конечная порождаяющая система, то $V$ называется конечномерным \\
    Иначе -- бесконечномерным
\end{definition}

\begin{property}
	Пусть $V$ -- конечномерное \\
    Тогда в $V$ \textbf{не} существует сколь угодно больших ЛНЗ систем
\end{property}

\begin{undefthm}{Другая формулировка}
	$ \exist N : \forall k > N \quad \forall v_1, ..., v_k \in V \quad v_1, ..., v_k$ ЛЗ
\end{undefthm}

\begin{proof}
	Пусть $u_1, ..., u_N$ -- конечная порождающая система. По теореме о линейной зависимости линейных комбинаций $v_1, ..., v_k$ ЛЗ
\end{proof}

\begin{theorem}[порождающие и ЛНЗ системы]
	Пусть $V$ -- конечномерное пространство
    \begin{enumerate}
        \item Пусть $u_1, ..., u_n$ -- минимальная по включению\footnote{Если из неё убрать вектор, она перестанет быть порождающей. Не обязательно минимальная по количеству векторов} порождающая система. Тогда она ЛНЗ
        \begin{proof}
        	Пусть $u_1, ..., u_n$ -- ЛЗ \\
            Тогда некоторый вектор -- ЛК остальных. Пусть это $u_n$
            $$ u_n = c_1u_1 + c_2u_2 + ... + c_{n - 1}u_{n - 1} $$
            Докажем, что $u_1, ..., u_{n - 1}, u_n$ -- не минимальная, то есть, что $u_1, ..., u_{n - 1}$ -- тоже порождающая \\
            Пусть $v \in V, \qquad v = a_1u_1 + ... + a_{n - 1}u_{n - 1} + a_nu_n $
            $$ v = a_1u_1 + ... + a_n \bigg( c_1u_1 + ... \bigg) = (a_1 - a_nc_1)u_1 + ... + (a_{n - 1} + a_nc_{n - 1})u_{n - 1} $$
        \end{proof}
        \item Пусть $u_1, ..., u_n$ -- максимальная по включению ЛНЗ. Тогда она порождающая
        \begin{proof}
        	Пусть $v \in V$ \\
            $u_1, ..., u_n$ -- ЛНЗ, $u_1, ..., u_n, v$ -- ЛЗ (т. к. $u_i$ -- минимальная) \\
            Применяем свойство 1b
        \end{proof}
    \end{enumerate}
\end{theorem}

\section{Базис}

\begin{definition}
	Пусть $V$ -- конечномерное векторное пространство \\
    Система векторов называется базисом $V$, если она ЛНЗ и порождающая
\end{definition}

\begin{theorem}[равносильные определения базиса]
	Следующие утверждения равносильны:
    \begin{enumerate}
    	\item $u_1, ..., u_n$ -- базис $V$
        \item $u_1, ..., u_n$ -- максимальная по включению ЛНЗ
        \item $u_1, ..., u_n$ -- минимальная по включению порождающая система
        \item Любой вектор можно единственным образом представить в виде ЛК $u_i$
    \end{enumerate}
\end{theorem}
