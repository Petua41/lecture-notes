\chapter{Графы}

\section{Сетевой график}

\subsection{Нахождение резервов работ}

Задан сетевой график $ G = \braket{M, N} $, работы -- дуги, $ u \in N, t(u) \ge 0 $ \\
Знаем $ t_{\text{кр}} $ -- критическое время \\
$ v[i] $ -- времена наступления событий (закончены все работы, которые туда входят, и можно приступать к любому из тех, которое выходит)

\begin{remark}
	В форме работы -- вершины, $ v[i] $ -- самое раннее возможное время входа в эту вершину (начала работы)
\end{remark}

$ w[i] $ -- самое позднее время наступления события $ i $

\begin{algo}[определения {$ w[i] $}]
	\item Топологическая сортировка, $ w[i] \define t_{\text{кр}} \quad \forall i $
    \item
    \begin{algorithm2e}[H]
        \For{$ i \define m$; $ i $-{}- to 1}
        {
            \For{$ u \in N_i^+ $}
            {
                \uIf{$ w[\operatorname{beg}(u)] > w[i] - t[u] $}
                {
                    $ w[\operatorname{beg}(u)] \define w[i] - t[u] $
                }
            }
        }
    \end{algorithm2e}
\end{algo}

\begin{problem}
	Есть задачи, которые выполняются на $ m $ параллельных процессорах \\
    Задан сетевой график в форме работы -- вершины \\
    Для каждой вершины задано время работы $ t[i] > 0 $ \\
    Для каждой дуги задано $ c(i, j) $ -- время передачи данных от $ i $ к $ j $, если $ i $ и $ j $ выполняются на разных процессорах
\end{problem}

\section{Паросочетания}

\begin{definition}
	Паросочетание в графе -- набор дуг, не имеющих общих начал и концов
\end{definition}

\begin{problem}
	Найти паросочетание наибольшего размера
\end{problem}

\begin{definition}
    Вершинная база графа -- подмножество вершин, которым инцидентны все другие
\end{definition}

\begin{statement}
    Размер максимального паросочетания равен размеру минимальной вершинной базы
\end{statement}

Наша задача эквивалентна этой:

\begin{problem}
	Найти минимальную вершинную базу
\end{problem}

Мы будем решать задачу о паросочетании, тем самым решим и задачу о вершинной базе

Задачу о паросочетании будем рассматривать на двудольном графе

\begin{definition}
    Двудольный граф $ G = \braket{M_1 \cup M_2, N} : u = (i, j) \in N \quad i \in M_1, ~ j \in M_2 $, т. е. вершины рабиты на два множества, и все рёбра идут из одного множества в другое
\end{definition}

\begin{algorithm}[построения максимального паросочетания]
    Дан двудольный граф $ G = \braket{M_1 \cup M_2, N} $ \\
    $ \overline{N} $ -- паросочетание, $ \overline{N} \ne \O $ \\
    $ X(\overline{N}) \sub M_1, Y(\overline{N}) \sub M_2 $ -- вершины из $ M_1 $ и $ M_2 $ соответственно, которые \textbf{не} покрываются паросочетанием \\
    \begin{undefthm}{Начальное состояние}
    	$ X = M_1, \quad Y = M_2 $
    \end{undefthm}
    Считаем, что все дуги идут из $ M_1 $ в $ M_2 $
    \begin{enumerate}
        \item Дуги из $ \overline{N} $ перенаправляем, чтобы они шли из $ M_2 $ в $ M_1 $
        \item Ищем путь из $ X(\overline{N}) $ в $ Y(\overline{N}) $
        \item Если пути нет, алгоритм заканчивается
        \item $ P = u_1v_1u_2v_2...u_k $, где $
        \begin{cases}
            u_i \notin \overline{N} \\
            v_j \in \overline{N}
        \end{cases} $ (последняя обязательно $ u $)
        \item $ \overline{N} \define \bigg( \overline{N} \setminus \set{v_j | v_j \in P} \bigg) \cup \set{u_i | u_i \in P} $
        \item Все дуги идут из $ M_1 $ в $ M_2 $
    \end{enumerate}
\end{algorithm}

\begin{algo}[нахождения минимального вершинного покрытия]
    \item
\end{algo}
