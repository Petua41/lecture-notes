\section{Конечные автоматы}

Мы рассматриваем конечный автомат как некий процесс с конечным числом состояний $ M $ \\
За один такт работы он переходит от одного состояния к другому \\
Есть $ A $ -- входящий алфавит -- некое внешнее воздействие \\
Следующее состояние выбирается в зависимости от текущего состояния и $ A $ \\
При этом строится $ B $ -- выходящий алфавит
$$ T : M \times A \to M, \qquad D : M \times A \to B $$

\begin{eg}[сканирующий автомат]
    На вход поступает последовательность символов, из которой надо выделить целое число и его напечатать \\
    Перед целым числом может стоять знак $ + $ или $ - $ \\
    Запрещены несколько подряд знаков ($ ++-+- $) и . \\
    Запрещён знак после цифры (число должно быть окаймлено прочими символами)
    $$ A: \set{+, -, [0-9], \text{ прочие символы}} $$
    $ M $:
    \begin{enumerate}
        \item[\textbf{S0}. ] готовность
        \item[\textbf{S1}. ] прочтён знак
        \item[\textbf{S2}. ] прочтена цифра
    \end{enumerate}
    Таблица переходов (отображение $ T $): \\
    \begin{tabular}{c | c c c}
    	& Цифра & Знак & Прочие \\
        \hline \\
        \textbf{S0} & \textbf{S2} & \textbf{S1} & \textbf{S0} \\
        \textbf{S1} & \textbf{S2} & \textbf{S0} & \textbf{S0} \\
        \textbf{S2} & \textbf{S2} & \textbf{S0} & \textbf{S0}
    \end{tabular} \\
    Действия:
    \begin{itemize}
        \item[\textbf{D0}. ] начальное состояние -- ничего не делаем
        \item[\textbf{D1}. ] прочтён знак -- устанавливаем $ \text{number} \define 0 $ и $ \text{sign} \define
        \begin{cases}
        	1 \\
            -1
        \end{cases} $
        \item[\textbf{D2}. ] прочтена первая цифра $ c $ -- устанавливаем $ \text{number} \define c $ и $ \text{sign} \define 1 $
        \item[\textbf{D3}. ] прочтена очередная цифра $ c $ -- устанавливаем $ \text{number} \define 10 \cdot \text{number} + c $
        \item[\textbf{D4}. ] прочтён знак (не первым) -- \textit{ОШИБКА}
        \item[\textbf{D5}. ] прочтён посторонний символ до начала чтения числа -- ничего не делаем
        \item[\textbf{D6}. ] печать -- устанавливаем $ \text{number} \define \text{number} \cdot \text{sign} $ и печатаем
    \end{itemize}

    Таблица действий (отображение $ D $): \\
    \begin{tabular}{c | c c c}
        & Цифра & Знак & Прочие \\
        \hline \\
        \textbf{S0} & \textbf{D2} & \textbf{D1} & \textbf{D0} \\
        \textbf{S1} & \textbf{D3} & \textbf{D4} & \textbf{D5} \\
        \textbf{S2} & \textbf{D3} & \textbf{D4} & \textbf{D6}
    \end{tabular}
\end{eg}


\begin{eg}[алгоритм Кнута-Морриса-Пратта]
    Автомат для поиска подстроки в строке \\
    На вход подаётся строка символов, нужно найти образец в ней \\
    Состояния -- наш образец \\
    Для образца ``\textit{hello}'':
    $$ \implies \boxed{0} \to \boxed{h} \to \boxed{e} \to \boxed{l} \to \boxed{l} \to \boxed{o} $$
    Переходы:
    \begin{itemize}
    	\item Из $ 0 $:
        \begin{itemize}
        	\item Если пришёл первый символ -- переходим в первое состояние
            \item Иначе -- остаёмся в $ 0 $
        \end{itemize}
        \item Из остальных состояний:
        \begin{itemize}
        	\item Если пришёл символ следующего состояния -- переходим в него
            \item Иначе -- возвращаемся в $ 0 $
        \end{itemize}
    \end{itemize}
    Если мы в последнем состоянии -- образец найден
\end{eg}

\section{Производящие функции}

