\chapter{Графы}

\section{Продолжение про деревья}

\begin{algorithm}[Прима]
	\hfill
    \begin{enumerate}
    	\item Выбираем $ i \in M $ \\
        $ \operatorname{num}(i) \define 0 $
        \item Рассмотрим рёбра $ (u, v) : $ помечена ровно одна из вершин \\
        $ \overline{N} $ -- кайма
        \item $ l(\overline{e}) = \min\limits_{l \in \overline{N}} l(e) $
        \item Добавляем $ \overline{e} $ к решению \\
        $ N' = N' \cup \set{\overline{e}} $
    \end{enumerate}
\end{algorithm}

\begin{theorem}
	Алгоритм Прима строит оптимальное остовное дерево
\end{theorem}

\begin{theorem}
	Следующие определения дерева эквивалентны:
    \begin{enumerate}
        \item \label{it:1} Связный граф без циклов
        \item \label{it:2} Максимальный граф без циклов\footnote{Если добавить ребро между любыми двумя вершинами, то появится цикл}
        \item \label{it:3} Любые две вершины соединены единственной цепью
        \item \label{it:4} Минимальный связный граф\footnote{Если убрать любое ребро, то связность пропадёт}
        \item \label{it:5} Связный граф: $ |N| = |M| - 1 $
        \item \label{it:6} Граф без циклов: $ |N| = |M| - 1 $
    \end{enumerate}
\end{theorem}

\begin{proof}
	\hfill
    \begin{itemize}
        \item $ \ref{it:1} \implies \ref{it:2} $ \\
        Если в связный граф добавить ребро, то он замкнётся
        \item $ \ref{it:2} \implies \ref{it:3} $ \\
        Граф \textbf{максимальный} без циклов, значит, любые две вершины соединены (если добавить ребро, они будут соединены двумя цепями) \\
        Отсюда же следует единственность
        \item $ \ref{it:3} \implies \ref{it:4} $ \\
        Любые две вершины соединены цепью $ \implies $ граф связный \\
        Минимальность следует из единственности
        \item $ \ref{it:4} \implies \ref{it:5} $ \\
        \begin{itemize}
        	\item $ |N| > |M| - 1 \implies $ есть цикл
            \item $ |N| < |M| - 1 \implies $ граф не связный
        \end{itemize}
        \item $ \ref{it:5} \implies \ref{it:6} $
        \item $ \ref{it:6} \implies \ref{it:1} $
    \end{itemize}
\end{proof}

\section{Реализация алгоритмов Прима и \textit{Крастера}}

\begin{algorithm}[\textit{Крастера}]
    Будем хранить промежуточный результат в виде компонент связности
    \hfill
    \begin{itemize}
    	\item Сортировка: $ l(e_1) \le l(e_2) \le ... \le l(e_n) $
        \item
        \begin{verbatim}
        	for e = (u, v) \in E do
                if FindSet(u) != FindSet(v)
                    then {A := A \cup {e}, Union(u, v)}
        \end{verbatim}
        \item Процедура Union:
        \begin{verbatim}
        	p(x) := x       # родитель вершины x
            define Link(x, y) {
                if rank(x) > rank(y)
                    then p(y) := x
                    else p(x) := y
                if rank(x) = rank(y)
                    then rank(y) := rank(x) + 1
            define Union(u, v) = Link(FindSet(u), FindSet(u))
        \end{verbatim}
    \end{itemize}

\end{algorithm}
