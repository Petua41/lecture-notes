\chapter{Базы данных}

\begin{definition}
	$U \sub A \times B$ -- отношение
\end{definition}

Схема БД: $R(A_1, A_2, ..., A_k)$ \\
Для каждого атрибута определяется домен \\
Арность -- количество атрибутов (столбцов)

\vspace{1em}

Читатель:
\begin{center}
    \begin{tabular}{c | c | c | c}
        \textbf{\textnumero~билета} & ФИО & Адрес & Телефон\\
        \hline \hline
        × & × & × & ×\\
        \hline
        × & × & × & ×
    \end{tabular}
\end{center}

\vspace{1em}

Книга:
\begin{center}
    \begin{tabular}{c | c | c | c}
        \textbf{Шифр} & Автор & Название & Количество \\
        \hline \hline
        × & × & × & × \\
        \hline
        × & × & × & ×
    \end{tabular}
\end{center}

\vspace{1em}

Связь:
\begin{center}
    \begin{tabular}{c | c | c}
        \textnumero~билета & Шифр & Дата \\
        \hline \hline
        × & × & × \\
        \hline
        × & × & ×
    \end{tabular}
\end{center}

Не должно быть избыточности

\vspace{1em}

Операции: \\
$R, S$ -- две таблицы
\begin{enumerate}
	\item $ R \cup S$
    \item $ R \cap S$
    \item $ R \setminus S$
    \item $ R \times S$
    \item $ \delta (R) $
    \item что-то $(R)$
    \item $R \bowtie S$ -- естественное соединение
    \item $R \underset\Theta\bowtie S$ -- $\Theta$-соединение
\end{enumerate}

