\chapter{Числовые ряды}

\section{Продолжение доказательства какого-то признака сходимости}

Рассмотрим ряд $\sum_{n = 1}^\infty a_n, \qquad a_n \ge 0 $
Положим $ q \define \ulim_{n \to \infty} \sqrt[n]{a_n} $

\begin{itemize}
	\item $ q > 1 $ \\
	Возьмём $ \veps \define q - 1 > 0, \quad q - \veps = 1 $ \\
	Вспомним, что $ \exist \seq{n_k}k $
	$$ \sqrt[n_k]{a_{n_k}} > q - \veps = 1 \iff a_{n_k} > q^{n_k} \implies a \not\to 0 $$
	Ряд расходится
\end{itemize}

\section{Признак Даламбера}

\begin{theorem}[признак Даламбера]
	\begin{equ}1
		\sum_{n = 1}^\infty a_n, \qquad a_n > 0
	\end{equ}
	$$ \exist \limi{n} \frac{a_{n + 1}}{a_n} = q $$
	Тогда:
	\begin{itemize}
		\item $ q < 1 \implies \eref1 $ сходится
		\item $ q > 1 \implies \eref1 $ расходится
		\item $ q = 1 $
	\end{itemize}
\end{theorem}

\begin{proof}
	\hfill
	\begin{itemize}
		\item $ q < 1 $ \\
		Возьмём $ \veps > 0 : r \define q + \veps < 1 $
		\begin{equ}2
			\exist N : \forall n > N \quad \frac{a_{n + 1}}{a_n} < q + \veps = r
		\end{equ}
		Будем считать, что $ n \ge N + 1 $
		\begin{equ}3
			\eref2 \implies
			\begin{cases}
				\dfrac{a_{N + 2}}{a_{N + 1}} < r \\
				\dfrac{a_{N + 3}}{a_{N + 2}} < r \\
				\widedots[3em] \\
				\dfrac{a_{n + 1}}{a_n} < r
			\end{cases}
		\end{equ}
		Здесь $ n - N $ неравенств. Перемножим их:
		\begin{equ}4
			\eref3 \implies \frac{\cancel{a_{N + 2}}}{a_N + 1} \cdot \frac{\cancel{a_{N + 3}}}{\cancel{a_{N + 2}}} \cdot ... \cdot \frac{\cancel{a_n}}{\cancel{a_{n + 1}}} \cdot \frac{a_{n + 1}}{\cancel{a_n}} < r^{n - N}
		\end{equ}
		\begin{equ}{41}
			\eref4 \iff \frac{a_{n + 1}}{a_{N + 1}} < \frac{r^n}{r^N}
		\end{equ}
		\begin{equ}{411}
			\eref{41} \iff a_{n + 1} < \frac{a_{N + 1}}{r^N} \cdot r^n
		\end{equ}
		$$ \sum_{n = N + 1}^\infty \frac{a_{N + 1}}{r^N} \cdot r^n \text{ сходится } \implies \sum_{n = N + 1}^\infty a_n \text{ сходится} $$
		Последний ряд -- это остаток ряда \eref1. Значит, ряд \eref1 сходится
		\item $ q > 1 $ \\
		Пусть $ \veps \define q - 1 > 0 $ \\
		По свойствам предела,
		\begin{equ}5
			\exist N : \forall n > N \quad \frac{a_{n + 1}}{a_n} > q - \veps = 1
		\end{equ}
		Будем считать, что $ n > N + 1 $
		$$ \eref5 \implies
		\begin{cases}
			\dfrac{a_{N + 2}}{a_{N + 1}} > 1 \\
			\dfrac{a_{N + 3}}{a_{N + 2}} > 1 \\
			\widedots[3em] \\
			\dfrac{a_{n + 1}}{a_n} > 1
		\end{cases} $$
		Здесь $ n - N $ неравенств. Перемножим их:
		$$ \frac{\cancel{a_{N + 2}}}{a_N + 1} \cdot \frac{\cancel{a_{N + 3}}}{\cancel{a_{N + 2}}} \cdot ... \cdot \frac{\cancel{a_n}}{\cancel{a_{n + 1}}} \cdot \frac{a_{n + 1}}{\cancel{a_n}} > 1 \iff \frac{a_{n + 1}}{a_{N + 1}} > 1 \iff a_{n + 1} > a_{N + 1} > 0 \implies \limi{n} a_n \ne 0 $$
		Значит, ряд \eref1 расходится
	\end{itemize}
\end{proof}

\section{Интегральный признак сходимости рядов}

\begin{theorem}
	$ f : [1, \infty], \qquad f(x) \ge 0, \qquad f(x) $ убывает
	\begin{equ}7
		\sum_{n = 1}^\infty f(n)
	\end{equ}
	\begin{equ}8
		\dint1\infty{f(x)}
	\end{equ}
	Тогда \eref7 и \eref8 сходятся или расходятся одновременно
\end{theorem}

\begin{proof}
	\hfill
	\begin{itemize}
		\item Пусть \eref7 сходится \\
		Тогда, если $ x \in [n, n + 1] $, имеем неравенство
		\begin{equ}9
			f(n) \ge f(x)
		\end{equ}
		Применим свойство определённых интегралов:
		\begin{equ}{10}
			\eref9 \implies \underbrace{\dint{n}{n + 1}{f(n)}}_{= f(n)(n + 1 - n) = f(n)} \ge \dint{n}{n + 1}{f(x)}
		\end{equ}
		\begin{equ}{11}
			\eref{10} \iff f(n) \ge \dint{n}{n + 1}{f(x)}
		\end{equ}
		Возьмём произволное $ N $ и $ n = 1, 2, ..., N $
		\begin{equ}{12}
			\eref{11} \implies
			\begin{cases}
				f(1) \ge \dint12{f(x)} \\
				f(2) \ge \dint23{f(x)} \\
				\widedots[8em] \\
				f(N) \ge \dint{N}{N + 1}{f(x)}
			\end{cases}
		\end{equ}
		Сложим все неравенства:
		\begin{equ}{13}
			\eref{12} \implies f(1) + ... + f(N) \ge \dint12{f(x)} + \dint23{f(x)} + ...  + \dint{N}{N + 1}{f(x)} = \dint1{N + 1}{f(x)}
		\end{equ}
		Поскольку ряд \eref7 сходится,
		$$ f(1) + ...  f(N) \le \sum_{n = 1}^\infty f(n) = M \in \R $$
		Таким образом,
		\begin{equ}{14}
			\eref{13} \implies \dint1{N + 1}{f(x)} \le M \quad \forall N
		\end{equ}
	\end{itemize}
	Возьмём $ \forall 1 < b < \infty $ и рассомтрим интеграл $ \dint1b{f(x)} $ \\
	Возьмём $ N : N + 1 > b $. Тогда
	\begin{equ}{15}
		\dint1b{f(x)} = \dint1{N + 1}{f(x)} - \underbrace{\dint{b}{N + 1}{f(x)}}_{\ge 0} \le \dint1{N + 1}{f(x)} \underset{\eref{14}}\le M
	\end{equ}
	\eref{15} $ \implies $ \eref8 сходится
	\item Пусть сходится интеграл \eref8 \\
	Для $ x \in [n, n + 1] $, в силу монотонности $ f $, справедливо
	\begin{equ}{16}
		f(x) \ge f(n + 1) \implies \dint{n}{n + 1}{f(x)} \ge \dint{n}{n + 1} = f(n + 1)
	\end{equ}
	Распишем \eref{16} для $ n = 1, 2, ..., N $
	\begin{multline}\lbl{17}
		\eref{16} \implies
		\begin{cases}
			\dint12{f(x)} \ge f(2) \\
			\dint23{f(x)} \ge f(3) \\
			\widedots[8em] \\
			\dint{N}{N + 1}{f(x)} \ge f(N + 1)
		\end{cases} \underimp{\sum} \\ \implies \dint12{f(x)} + \dint23{f(x)} + ... + \dint{N}{N + 1}f(x) \ge f(2) + f(3) + ... + f(N + 1)
	\end{multline}
	$$ \eref{17} \iff \dint1{N + 1}{f(x)} \ge f(2) + f(3) + ... + f(N + 1) $$
	В наших обозначениях, это означает, что
	\begin{equ}{18}
		\dint1{N + 1}{f(x)} \le \dint1\infty{f(x)} \define L
	\end{equ}
	\begin{equ}{19}
		\eref{17}, \eref{18} \implies f(2) + f(3) + ... + f(N + 1) \le L
	\end{equ}
	Получили, что последовательность ограничена сверху числом $ L $, которое не зависит от $ N $, а значит, ряд $ \sum_{n = 2}^\infty f(n) $ сходится. Это -- остаток ряда \eref7, значит, \eref7 сходится
\end{proof}

\begin{eg}
	Рассмотрим ряд $ \sum_{n = 1}^\infty \dfrac1{n^p}, \quad p > 0 $ и $ f(x) \define \dfrac1{x^p} $ \\
	Этот ряд сходится одновременно с интегралом $ \dfint1\infty{x^p} $, который сходится при $ p > 1 $ \\
	Получили, что наш ряд:
	\begin{itemize}
		\item сходится при $ p > 1 $
		\item расходится при $ 0 < p < 1 $
	\end{itemize}
	В частности, ряд $ \sum_{n = 1}^\infty \dfrac1n $ расходится. Этот ряд называется \textbf{гармоническим}
\end{eg}

\begin{eg}
	Рассотрим ряд
	$$ \sum_{n = 1}^\infty \frac1{(n + 1)\ln^p(n + 1)}, \qquad p > 0 $$
	И функцию
	$$ f(x) = \frac1{(x + 1)\ln^p(x + 1)} $$
	Ряд сходится одновременно с интегралом
	$$ \dfint1\infty{(x + 1)\ln^p(x + 1)} \underset{x + 1 \define y}= \dfint[y]2\infty{y\ln^py} $$
	Этот интеграл сходится при $ p > 1 $
\end{eg}

\section{Абсолютно сходящиеся ряды}

Пусть имеется некий ряд
\begin{equ}{21}
	\sum_{n = 1}^\infty a_n
\end{equ}
Рассмотрим также ряд
\begin{equ}{22}
	\sum_{n = 1}^\infty |a_n|
\end{equ}

\begin{definition}
	Говорят, что ряд \eref{21} абсолютно сходится, если сходится ряд \eref{22}
\end{definition}

\begin{theorem}
	Если ряд абсолютно сходится, то он сходится
\end{theorem}

\begin{proof}
	Выпишем критерий Коши для ряда \eref{22}:
	\begin{equ}{23}
		\forall \veps > 0 \quad \exist N : \forall m > n > N \quad \bigg| \sum_{k = n + 1}^m |a_k| \bigg| < \veps
	\end{equ}
	$$ \eref{23} \iff \sum_{k = n + 1}^m |a_k| < \veps $$
	$$ \bigg| \sum_{k = n + 1}^m a_k \bigg| \le \sum_{k = n + 1}^m |a_k| \underset{\eref{23}}< \veps $$
	Значит, ряд \eref{21} сходится
\end{proof}

\begin{definition}
	Если ряд \eref{21} сходится, а ряд \eref{22} расходится, то говорят, что ряд \eref{21} сходится неабсолютно или условно
\end{definition}

\section{Преобразование Абеля}

Пусть имеется сумма $ \sum_{n = 1}^N a_nb_n $ \\
Определим числа следующим образом:
$$
\begin{cases}
	A_0 = 0 \\
	A_1 = a_1 \\
	A_2 = a_1 + a_2 \\
	\widedots[2em] \\
	A_k = a_1 + ... + a_k
\end{cases} $$
Тогда
$$
\begin{cases}
	a_1 = A_1 - A_0 \\
	a_2 = A_2 - A_1 \\
	\widedots[2em] \\
	a_k = A_k - A_{k - 1}
\end{cases} $$
Тогда наша сумма равна
\begin{multline*}
	\sum_{n = 1}^N a_nb_n = \sum_{n = 1}^N(A_n - A_{n - 1})b_n = \sum_{n = 1}^NA_nb_n - \sum_{n = 1}^NA_{n - 1}b_n \underset{\left(
	\begin{subarray}{c}
		n - 1 \define k \\
		n \define k + 1
	\end{subarray}\right)}= \sum_{n = 1}^NA_nb_n - \sum_{k = 0}^NA_kb_{k + 1} \underset{\eref{txt}}= \\ = \sum_{n = 1}^NA_nb_n - \sum_{n = 0}^{N - 1}A_nb_{n + 1} \underset{(A_0 = 0)}= \sum_{n = 1}^NA_nb_n - \sum_{n = 1}^{N - 1}A_nb_{n + 1} = \bigg( A_Nb_N + \sum_{n = 1}^{N - 1}A_nb_n \bigg) - \sum_{n = 1}^{N - 1}A_nb_{n + 1} = \\ = A_Nb_N + \sum_{n = 1}^{N - 1}(A_nb_n - A_nb_{n + 1}) = A_Nb_N + \sum_{n = 1}^{N - 1}A_n(b_n - b_{n + 1})
\end{multline*}
\begin{equ}{txt}
	\text{В качестве счётчика суммы можно записать любую букву}
\end{equ}

\begin{theorem}[признак Абеля сходимости рядов]
	\begin{equ}{31}
		\sum_{n = 1}^\infty a_nb_n
	\end{equ}
	Пусть ряд
	\begin{equ}{32}
		\sum_{n = 1}^\infty a_n \text{ сходится}
	\end{equ}
	Последовательность
	\begin{equ}{33}
		\seq{b_n}n \text{ монотонна}
	\end{equ}
	и ограничена:
	\begin{equ}{34}
		\exist M : \forall n \quad |b_n| \le M
	\end{equ}
	Тогда ряд \eref{31} сходится
\end{theorem}

\begin{proof}
	Так как ряд \eref{32} сходится, то, по критерию Коши:
	\begin{equ}{35}
		\forall \veps > 0 \quad \exist N : \forall m > n > N \quad \bigg| \sum_{k = n + 1}^m a_k \bigg| < \veps
	\end{equ}
	Положим
	$$
	\begin{cases}
		A_1 \define a_{n + 1} \\
		A_2 \define a_{n + 1} + a_{n + 2} \\
		\widedots[9em] \\
		A_k \define a_{n + 1} + ... + a_{n + k}
	\end{cases} $$
	Применим преобразование Абеля в наших обозначениях:
	\begin{multline*}
		\sum_{k = n + 1}^ma_kb_k = A_{m - n}b_m + \sum_{k = 1}^{m - n - 1}A_k(b_{n + k} - b_{n + k + 1}) \implies \bigg| \sum_{k = n + 1}^m a_kb_k \bigg| \le \\ \le |A_{m - n}| \cdot |b_m| + \sum_{k = 1}^{m - n - 1} |A_k| \cdot |b_{n + k} - b_{n + k + 1}| \underset{\eref{34}}\le \\ \le M \big| a_{n + 1} + ... + a_m \big| + \sum_{k = 1}^{m - n - 1} \big| a_{n + 1} + ... + a_{n + k} \big| \cdot \big| b_{n + k} - b_{n + k + 1} \big| \underset{\eref{35}}\le M \cdot \veps + \sum_{k = 1}^{m - n - 1}\veps |b_{n + k} - b_{n + k + 1}| = \\ = M \veps + \veps \bigg| \sum_{k = 1}^{m + n - 1}(b_{n + k} - b_{n + k + 1}) \bigg| = M \veps + \veps |b_{n + 1} - b_m| \le M \veps + \veps \big( |b_{n + 1}| + |b_m| \big) \le 3 \veps
	\end{multline*}
	Значит, при $ m > n > N $
	$$ \bigg| \sum_{k = n + 1}^m a_kb_k \bigg| < 3 M \veps \implies \eref{31} \text{ сходится} $$
\end{proof}

\begin{theorem}[признак Дирихле сходимости рядов]
	\begin{equ}{36}
		\sum_{n = 1}^\infty a_nb_n
	\end{equ}
	\begin{equ}{37}
		\exist L : \forall n \quad \bigg| \sum_{n = 1}^N a_n \le L
	\end{equ}
	\begin{equ}{38}
		\seq{b_n}n \text{ монотонна}
	\end{equ}
	\begin{equ}{39}
		b_n \underarr{n \to \infty} 0
	\end{equ}
	Тогда ряд \eref{36} сходится
\end{theorem}

\begin{proof}
	Будем пользоваться критерием Коши \\
	Возьмём $ \forall \veps > 0 $
	\begin{equ}{310}
		\eref{39} \implies \exist N : \forall n > N \quad |b_n| < \veps
	\end{equ}
	Возьмём $ \forall m > n > N $ \\
	Выберем числа:
	$$
	\begin{cases}
		A_1 \define a_{n + 1} \\
		A_2 = a_{n + 1} + a_{n + 2} \\
		\widedots[9em] \\
		A_k = a_{n + 1} + ... + a_{n + k}
	\end{cases} $$
	Воспользуемся преобразованием Абеля для такой суммы:
	\begin{equ}{311}
		\sum_{k = n + 1}^m a_kb_k = A_{m - n}b_m + \sum_{k = 1}^{m - n - 1}A_k(b_{n + k} - b_{n + k + 1})
	\end{equ}
	$$ A_k = (a_1 + ... + a_{n + k}) - (a_1 + ... + a_n) $$
	\begin{equ}{312}
		\eref{37} \implies |A_k| \le |a_1 + ... + a_{n + k}| + |a_1 + ... + a_n| \le L + L = 2L
	\end{equ}
	\begin{multline*}
		\eref{310}, \eref{311}, \eref{312} \implies \bigg| \sum_{k = n + 1}^m a_kb_k \bigg| \le 2L \cdot |b_m| + \sum_{k = 1}^{m - n - 1}2L|b_{k + n} - b_{k + n + 1}| = \\ = 2L \bigg( |b_m| + \bigg| \sum_{k = 1}^{m - n - 1}(b_{n + k} - b_{n + k + 1}) \bigg| \bigg) = 2L \big( |b_m| + |b_{n + 1} - b_m| \big) \le 2L \big( |b_m| + |b_{n + 1}| + |b_m| \big) \underset{\eref{310}}< 6L\veps \implies \\ \implies \eref{36} \text{ сходится}
	\end{multline*}
\end{proof}

\begin{definition}
	Знакопеременным рядом называется ряд вида
	\begin{equ}{313}
		\sum_{n = 1}^\infty (-1)^{n - 1}b_n, \qquad b_n > 0
	\end{equ}
\end{definition}

\begin{theorem}[признак сходимости знакопеременного ряда]
	\begin{equ}{314}
		\begin{rcases}
			b_n \text{ монотонна} \\
			b_n \underarr{n \to \infty} 0
		\end{rcases}
	\end{equ}
	Тогда ряд \eref{313} сходится
\end{theorem}

\begin{proof}
	Положим $ a_k \define (-1)^{k - 1} $
	$$ 1 - 1 + 1 - 1 + ... + (-1)^{N - 1} = \left[
	\begin{aligned}
		0 \\
		1
	\end{aligned} \right. $$
\end{proof}

\section{Перестановка слагаемых в рядах}

\begin{undefthm}{Приготовления к теореме}
	Пусть имеется некое взаимно однозначное отображение (биекция) $ \sigma : \N \to \N $ \\
	Положим $ \tau \define \sigma^{-1} $ \\
	Пусть имеется некая последовательность $ \seq{a_n}n $ \\
	Определим
	\begin{equ}{51}
		b_n \define a_{\sigma(n)} \iff a_n = b_{\tau(n)}
	\end{equ}
	Будем считать, что $ \sigma(n) \not\equiv n $
\end{undefthm}

\begin{theorem}
	$ \forall n \quad a_n \ge 0, \qquad \sum_{n = 1}^\infty a_n $ сходится \\
	Тогда
	\begin{equ}{52}
		\sum_{n = 1}^\infty b_n \text{ сходится}
	\end{equ}
	\begin{equ}{53}
		\sum_{n = 1}^\infty a_n = \sum_{n = 1}^\infty b_n = \sum_{n = 1}^\infty a_{\sigma(n)}
	\end{equ}
	(Строго говоря, третья сумма не определена, и в формулировку входит только равенство первых двух)
\end{theorem}

\begin{proof}
	\begin{itemize}
		\item \eref{52} \\
		Обозначим $ \sum_{n = 1}^\infty a_n \define A \in \R $, т. е.
		\begin{equ}{54}
			\forall N \quad \sum_{n = 1}^N a_n \le A
		\end{equ}
		Возьмём $ \forall K $ и рассмотрим сумму
		$$ \sum_{\nu = 1}^K b_\nu = a_{\sigma(1)} + ... + a_{\sigma(K)} $$
		Пусть $ N \define \max\set{\sigma(1), ..., \sigma(K)} $. Тогда
		\begin{equ}{55}
			\sum_{\nu = 1}^K b_\nu = a_{\sigma(1)} + ... + a_{\sigma(K)} \le a_1 + a_2 + ... + a_N \underset{\eref{54}}\le A
		\end{equ}
		\begin{equ}{56}
			\eref{55} \implies \sum_{\nu = 1}^\infty b_\nu \le A
		\end{equ}
		\item \eref{53}
		Обозначим
		$$ \sum_{\nu = 1}^\infty b_\nu \define B $$
		Вспомним, что $ a_n = b_{\tau(n)} $ \\
		Заменяя буквы, аналогичными соображениями получаем $ B \le A $ \\
		При этом, $ A \le B $. Значит, суммы этих рядов совпадают $ \implies \eref{53} $
	\end{itemize}
\end{proof}

Возьмём $ a \in \R $ и определим
$$ a_+ \define
\begin{cases}
	a, \quad a \ge 0 \\
	0, \quad a < 0
\end{cases}, \qquad a_- \define
\begin{cases}
	|a|, \quad a < 0 \\
	0, \quad a \ge 0
\end{cases} $$
\begin{intuition}
	$ a = a_+ - a_-, \qquad |a| = a_+ + a_- $
\end{intuition}


\begin{theorem}
	Пусть есть ряд
	$$ \sum_{n = 1}^\infty a_n \text{ \textbf{-- абсолютно сходящийся}} $$
	Опять есть биекция $ \sigma $ и $ b_n = a_{\sigma(n)} $
	Тогда
	\begin{equ}{56}
		\sum_{n = 1}^\infty b_n = \sum_{n = 1}^\infty
	\end{equ}
\end{theorem}

\begin{proof}
	$$ \sum_{n = 1}^\infty a_n = \sum_{n = 1}^\infty a_{\sigma(n)} $$
	\begin{equ}{57}
		\sum_{n = 1}^\infty |a_n| = \sum_{n = 1}^\infty (a_{n+} + a_{n-}) \text{ сходится}
	\end{equ}
	\begin{equ}{58}
		\eref{57} \implies
		\begin{cases}
			\sum_{n = 1}^\infty a_{n+} \text{ сходится} \\
			\sum_{n = 1}^\infty a_{n-} \text{ сходится}
		\end{cases}
	\end{equ}
	$$ b_{n+} = a_{\sigma(n)+}, \qquad b_{n-} = a_{\sigma(n)-} $$
	\begin{equ}{59}
		\begin{rcases}
			\sum_{n = 1}^\infty b_{n+} = \sum_{n = 1}^\infty a_{n+} \\
			\sum_{n = 1}^\infty b_{n-} = \sum_{n = 1}^\infty a_{n-}
		\end{rcases}
	\end{equ}
	\begin{equ}{510}
		\eref{59} \implies \sum_{n = 1}^\infty (b_{n+} - b_{n-}) = \sum_{n = 1}^\infty(a_{n+}-a_{n-})
	\end{equ}
\end{proof}

\begin{theorem}[Римана]
	Пусть \textbf{не}абсолютно сходящийся ряд $ \sum_{n = 1}^\infty a_n $ \\
	Пусть $ c \in \R $
	Тогда $ \exist \sigma : \N \to \N $ такая, что при $ b_n = a_{\sigma(n)} $, $ \sum_{n = 1}^\infty b_n = c $
\end{theorem}
