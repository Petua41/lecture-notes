\chapter(R\textasciicircum{}n){$ \R^n $}

\section{Непрерывность суперпозиции отображений}

\begin{theorem}
	$ E \sub \R^n, \qquad G \sub \R^k, \qquad A \in E $ -- т. сг. $ E, \qquad \alpha \in G $ -- т. сг. $ G, \qquad n, k, l \ge 1 $ \\
	$ F : E \to \R^k, \qquad \forall x \in E \quad F(X) \in G, \qquad F(A) = \alpha, \qquad F $ непр. в $ A $ \\
	$ \Phi : G \to \R^l, \qquad \Phi $ непр. в $ \alpha, \qquad K : E \to \R^l, \qquad K(X) = \Phi(F(X)) $
	\begin{equ}1
		\text{Тогда } K \text{ непр. в } A
	\end{equ}
\end{theorem}

\begin{proof}
	Т. к. $ \Phi $ непр. в $ \alpha $,
	\begin{equ}2
		\forall \veps > 0 \quad \exist \eta > 0 : \forall y \in B_\eta(\alpha) \cap G \quad \norm{\Phi(y) - \Phi(\alpha)}_l < \veps
	\end{equ}
	Т. к. $ F $ непр. в $ A $,
	\begin{equ}3
		\exist \delta > 0 : \forall X \in B_\delta(A) \cap E \quad \norm{F(X) - F(A)}_k < \eta
	\end{equ}
	\begin{equ}4
		\eref3 \implies F(X) \in B_\eta(F(A)) \cap G \implies F(X) \in B_\eta(\alpha) \cap G
	\end{equ}
	\begin{equ}5
		\eref4, \eref2 \implies \forall X \in B_\delta(A) \cap E \quad \norm{\Phi(F(X)) - \Phi(\alpha)}_l < \veps
	\end{equ}
	$$ \eref5 \underset{(F(A) = \alpha)}\iff \norm{\Phi(F(X)) - \Phi(F(A))}_l < \veps \iff \norm{K(X) - K(A)}_l < \veps $$
\end{proof}

\begin{definition}
	$ K \sub \R^n, \qquad n \ge 1, \qquad K \ne \O $ \\
	Множество $ K $ называется компактом, если оно ограничено и замкнуто
\end{definition}

\begin{theorem}[первая Вейерштрасса]\label{th:1}
	$ K $ -- компакт в $ \R^n, \quad n \ge 1, \qquad f : K \to \R^n, \qquad f $ непр. во всех т. сг. \\
	Тогда $ f $ ограничена, т. е.
	\begin{equ}{61}
		\exist M : \forall X \in K \quad |f(X)| \le M
	\end{equ}
\end{theorem}

\begin{proof}
	Пусть это не так, т. е.
	\begin{equ}6
		\forall N \ge 2 \quad \exist X_N \in K : |f(X_N)| > |f(X_1)| + ... + |f(X_{N - 1})| + N, \qquad |f(X_1)| > 1
	\end{equ}
	$$ \eref6 \implies \forall p \ne q \quad X_p \ne X_q $$
	Поскольку все они принадлежат $ K $, а $ K $ -- это ограниченное множество, то
	\begin{equ}7
		\exist R > 0 : \forall N \quad \norm{X_N} \le R
	\end{equ}
	Можно применить принцип выбора Больцано-Вейерштрасса:
	\begin{equ}8
		\exist \seq{X_{N_m}}m, \quad \exist X_* : X_{N_m} \underarr{m \to \infty} X_*
	\end{equ}
	$$ X_{N_m} \in K $$
	Поскольку они все различны, $ X_* $ -- т. сг. $ K \underimp{K \text{ (замкн.)}} X_* \in K $ \\
	Возьмём $ \veps = 1 $. Тогда, в силу непрерывности $ f $ в $ X_* $,
	\begin{equ}9
		\exist \delta_0 : \forall X \in K \cap B_{\delta_0}(X_*) \quad |f(X) - f(X_*) < 1
	\end{equ}
	\begin{equ}{10}
		\eref9 \implies |f(X)| \trile |f(X) - f(X_*) + |f(X_*)| < 1 + |f(x_*)
	\end{equ}
	\begin{equ}{101}
		\text{Возьмём } N_0 > 1 + |f(X_*)|
	\end{equ}
	\begin{equ}{11}
		\eref8 \implies \exist N_1 : \forall m > N_1 \quad \norm{X_{N_m} - X_*} < \delta_0
	\end{equ}
	\begin{equ}{12}
		\text{Возьмём } m_0 \define \max\set{N_0, N_1 + 1}
	\end{equ}
	\begin{equ}{13}
		\eref6, \eref{101}, \eref{12} \implies |f(X_{N_{m_0}})| > N_{m_0} \ge N_0 > |f(X_*)| + 1
	\end{equ}
	С другой стороны,
	\begin{equ}{14}
		\eref{10}, \eref{11}, \eref{12} \implies |f(X_{N_{m_0}}) < |f(X_*)| + 1
	\end{equ}
	$$ \eref{13} \contra \eref{14} $$
\end{proof}

\begin{theorem}[вторая Вейерштрасса]
	$ K \sub \R^n $ -- компакт, $ \qquad f : K \to \R^n $ непрерывна во всех т. сг. $ K $ \\
	Тогда
	\begin{equ}{15}
		\exist X_-, X_+ \in K : \forall X \in K \quad f(X_-) \le f(X) \le f(X_+)
	\end{equ}
\end{theorem}

\begin{proof}
	\hfill
	\begin{itemize}
		\item $ X_+ $ \\
		Пусть такого $ X_+ $ не существует \\
		По теореме \ref{th:1},
		\begin{equ}{16}
			\exist M : \forall X \in K \quad f(X) \le M
		\end{equ}
		\begin{equ}{17}
			\eref{16} \implies \sup_{x \in K}f(X) \define M_0 \le M
		\end{equ}
		\begin{equ}{18}
			\forall X \in K \quad f(X) < M_0
		\end{equ}
		\begin{equ}{19}
			\vphi(X) \define \frac1{M_0 - f(X)}
		\end{equ}
		\begin{equ}{20}
			\eref{18} \implies \forall X \in K \quad \vphi(X) > 0
		\end{equ}
		$ \eref{18} \implies \vphi $ непрерывна во всех т. сг. $ K $ \\
		По теореме \ref{th:1}, $ \vphi $ ограничена, т. е
		\begin{equ}{21}
			\exist L : \forall X \in K \quad \vphi(X) \le L
		\end{equ}
		\begin{equ}{22}
			\eref{19}, \eref{21} \implies \frac1{M_0 - f(X)} \le L
		\end{equ}
		\begin{equ}{23}
			\eref{22} \iff M_0 - f(X) \ge \frac1L \iff f(X) \le M_0 - \frac1L \quad \forall X \in K
		\end{equ}
		\begin{equ}{24}
			\eref{23} \implies \sup_{X \in K} f(X) \le M_0 - \frac1L
		\end{equ}
		Вспомним, что через $ M_0 $ мы обозначили $ \sup_{X \in K} f(X) $ -- \contra
		\item $ X_- $ \\
		Рассмотрим $ g(X) = -f(X) $ \\
		По только что доказанному,
		$$ \exist X_- : g(X) \le g(X_-) \iff -f(X) \le -f(X_-) \iff f(X) \ge f(X_-) $$
	\end{itemize}
\end{proof}

\begin{definition}
	$ E \sub \R^n, \quad n \ge 1, \qquad X_0, X_1, X_2 \in E $ \\
	$ X_0 $ будем называть внутренней точкой $ E $, если $ \exist B_r(X_0) \supset E $ \\
	$ X_1 $ будем называть внешней точкой $ E $, если $ \exist \delta : B_\delta(X_1) \cap E \ne \O $ \\
	$ X_2 $ будем называть граничной точкой $ E $, если она не внутренняя, и не внешняя
\end{definition}

\begin{statement}
	$ E \sub \R^n, \quad n \ge 1, \qquad E \ne \O, \qquad E \ne \R^n $ \\
	Тогда множество его граничных точек не пусто
\end{statement}

\begin{definition}
	Множество граничных точек называется границей $ E $
\end{definition}

\begin{notation}
	$ \partial E $
\end{notation}

\begin{definition}
	$ E \sub \R^n, \quad n \ge 2, \qquad X = (x_1, ..., x_n) \in E $ -- внутр. т. $ E $ \\
	Обозначим $ e_k \define (0, ..., \underset{k \text{ место}}1, ..., 0) $
	\begin{equ}{31}
		\exist \delta : \forall \underset{h \ne 0}{|h| < \delta} \quad \forall k = 1, ..., n \quad X + he_k = E
	\end{equ}
	$ f : E \to \R $
	Чатсной производной по переменной $ x_k $ называется
	\begin{equ}{32}
		f_{x_k}'(X) \define \limz{h} \frac{f(X + he_k) - f(X)}h
	\end{equ}
	Рассмотрим $ g(y) = f(x_1, ..., \underset{k \text{ место}}y, ..., x_n), \qquad y \in (x_k - \delta, x_k + \delta) $
	$$ \eref{32} \implies g'(x_k) = f'_{x_k}(X) $$
\end{definition}

\begin{eg}
	$$ f(x_1, x_2) =
	\begin{cases}
		\dfrac{x_1x_2}{x_1^2 + x_2^2}, \qquad (x_1, x_2) \ne \On[2] \\
		f(\On[2]) = 0
	\end{cases} $$
	$$
	\begin{cases}
		(0, 0) + he_1 = (h, 0) \\
		(0, 0) + he_2 = (0, h)
	\end{cases} $$
	$$ \frac{f(\On[2] + he_1) - f(\On[2])}h = \frac{0 - 0}h = 0 \underarr{h \to 0} 0 $$
	$$ \frac{f(\On[2] + he_2) - f(\On[2])}h = \frac{0 - 0}h = 0 \underarr{h \to 0} 0 $$
	$$ \implies \exist f_{x_1}'(\On[2]), f_{x_2}'(\On[2]) $$
	$$ X_n \define \bigg( \frac1n, \frac1n \bigg), \qquad X_n \underarr{n \to \infty} \On[2] $$
	$$ f(X_n) = \frac{\dfrac1{n^2}}{\dfrac2{n^2}} = \half \underarr{n \to \infty} \half $$
\end{eg}

\begin{definition}
	$ E \sub \R^n, \quad n \ge 1, \qquad X \in E $ -- внутр. т., $ \qquad f : E \to \R $ \\
	Бдуем говорить, что $ f $ дифференцируема в $ X $, если
	\begin{equ}{41}
		\exist a_1, ..., a_n \in \R : \forall
		\begin{cases}
			H \in \R^n \\
			X + H \in E \\
			H = (h_1, ..., h_n)
		\end{cases} \quad f(X + H) - f(X) = a_1h_1 + ... + a_nh_n + r(H)
	\end{equ}
	\begin{equ}{42}
		\frac{r(H)}{\norm{H}} \underarr{H \to \On} 0
	\end{equ}
\end{definition}

\begin{theorem}
	$ f $ дифференц. в $ X $
	\begin{equ}{43}
		\implies \forall k = 1, ..., n \quad \exist f_{x_k}'(X) = a_k
	\end{equ}
\end{theorem}

\begin{proof}
	Возьмём $ \veps = 1 $
	\begin{equ}{44}
		\exist \delta > 0 : \forall 0 < \norm{H} < \delta \quad \bigg| \frac{r(H)}{\norm{H}} \bigg| < 1
	\end{equ}
	$$ \eref{44} \implies |r(H)| < \norm{H} $$
	Возьмём $ A \define \sqrt{a_1^2 + ... + a_n^2} $
	\begin{equ}{45}
		|a_1h_1 + ... + a_nh_n| \le A \norm{H}
	\end{equ}
	$$ \eref{41}, \eref{44}, \eref{45} \implies |f(X + H) - f(X)| \le |a_1h_1 + ... + a_nh_n| + |r(H)| \le A \norm{H} + \norm{H} \underarr{H \to \On} 0 $$
	Возьмём $ \forall 1 \le k \le n $
	\begin{undefthm}{Напоминание}
		Мы обозначаем $ e_k \define (0, ..., \underset{k}1, ..., 0) $
	\end{undefthm}
	Определим $ H_k \define he_k $ \\
	Тогда $ \norm{H_k} = |h| $
	\begin{equ}{46}
		f(X + H_k) - f(X) = a_kh + r(H_k)
	\end{equ}
	Поделим на $ h $:
	$$ \underset{\define F}{\frac{f(X + H_k) - f(X)}h} = a_k + \frac{r(H_k)}h $$
	$$ \eref{42} \implies \bigg| \frac{r(H_k)}h \bigg| = \bigg| \frac{r(H_k)}{\norm{H_k}} \bigg| \underarr{h \to 0} 0 \implies F \underarr{h \to 0} a_k $$
\end{proof}

\begin{implication}
	$ f $ дифференцируема в $ X $
	\begin{equ}{47}
		\implies f(X + H) - f(X) = f_{x_1}'(X)h_1 + ... + f_{x_n}'(X)h_n + r(H)
	\end{equ}
\end{implication}

\begin{definition}
	Эта функция называется дифференциалом функции $ f $ в точке $ X $ при значении $ H $
\end{definition}

\begin{notation}
	$ \di f(X, H) \define f_{x_1}'(X)h_1 + ... + f_{x_n}'(X)h_n + r(H) $
\end{notation}

\section{Производная по направлению}

\begin{definition}
	$ \nu \in \R^n, \qquad \norm\nu = 1, \qquad \nu = \alpha_1e_1 + ... + \alpha_ne_n, \qquad n \ge 2 $ \\
	$ \limz{h} \dfrac{f(X + h\nu) - f(X)}h \define f_\nu'(X) $ называется частной производной по направлению $ \nu $
\end{definition}

\begin{theorem}
	$ f $ дифференцируема в $ X $. Тогда для любого $ \nu $
	\begin{equ}{49}
		\exist f_\nu'(X) = \alpha_1f_{x_1}'(X) + ... + \alpha_nf_{x_n}'(X)
	\end{equ}
\end{theorem}

\begin{proof}
	$ H = h\nu, \qquad \norm{H} = |h| \cdot \norm\nu = |h| $
	\begin{equ}{410}
		f(X + H) - f(X) = f_{x_1}'(X)h_1 + ... + f_{x_n}'(X)h_n + r(H)
	\end{equ}
	$$ H \underset{(h_k = h\alpha_k)}= (h\alpha_1, ..., h\alpha_n) $$
	$$ \eref{410} \iff f(X + H) - f(X) = hf_{x_1}'(X)\alpha_1 + ... + hf_{x_n}'(X)\alpha_n + r(H) $$
	\begin{equ}{411}
		\eref{410} \implies \frac{f(X + h\nu) - f(X)}h = f_{x_1}'(X)\alpha_1 + ... + f_{x_n}'(X)\alpha_n + \frac{r(h\nu)}h
	\end{equ}
	\begin{equ}{412}
		\bigg| \frac{r(h\nu)}h \bigg| = \bigg| \frac{r(H)}{\norm{H}} \bigg| \underarr{h \to 0} 0
	\end{equ}
	$$ \eref{411}, \eref{412} \implies \frac{f(X + h\nu) - f(X)}h \underarr{h \to 0} f_{x_1}'(X)\alpha_1 + ... + f_{x_n}'(X)\alpha_n $$
\end{proof}

\begin{definition}
	$ f $ дифференцируема в $ X $ \\
	Градиентом $ f $ в точке $ X $ называется \textbf{вектор-строка} $ \big( f_{x_1}'(X), ..., f_{x_n}'(X) \big) $
\end{definition}

\begin{notation}
	$ \grad f $
\end{notation}

\begin{statement}
	$ \nu = (\alpha_1, ..., \alpha_n) $ \\
	Тогда $ f $ дифференцируема в $ X $ \\
	Воспользуемся нераенством Коши-Буняковского:
	\begin{equ}{413}
		|f_\nu'(X)| = \sqrt{f_{x_1}'^2(X) + ... + f_{x_n}'^2(X)} \cdot \sqrt{\alpha_1^2 + ... + \alpha_n^2} = \sqrt{f_{x_1}'^2(X) + ... + f_{x_n}'^2(X)}
	\end{equ}
	$$ \eref{413} \iff |f_\nu'(X)| \le \norm{\grad f(X)} $$
	Пусть $ \grad f(X) \ne \On $ \\
	Положим $ t \define \norm{\grad f(X)} > 0 $ и
	\begin{equ}{414}
		\nu_0 \define \dfrac1t \grad f(X), \qquad \norm{\nu_0} = 1
	\end{equ}
	\begin{equ}{415}
		\eref{414} \implies \nu_0 = \bigg( \frac1tf_{x_1}'(X), ..., \frac1tf_{x_n}'(X))
	\end{equ}
	\begin{equ}{416}
		\eref{49}, \eref{415} \implies f_{\nu_0}'(X) = \frac1tf_{x_1}'^2(X) + ... + \frac1tf_{x_n}'^2(X) = \frac1t \norm{\grad f(X)}^2 \underset{t \bydef \norm{\grad f(X)}}= \norm{\grad f(X)}
	\end{equ}
	\begin{statement}
		Если $ \nu \ne \nu_0 $, то $ f_\nu'(X) < f_{\nu_0}'(X) $
	\end{statement}
	\begin{proof}
		Без доказательства
	\end{proof}
\end{statement}

\section{Необходимое условие локального экстремума функции}

\begin{theorem}
	$ X \in E \sub \R^n, \quad n \ge 2, \qquad X $ -- внутр. т., $ \qquad f : E \to \R, \qquad f $ дифференц. в $ X $ \\
	$ X $ -- точка локального экстремума $ f $
	$$ \implies \grad f(X) = \On $$
\end{theorem}

\begin{proof}
	Возьмём $ \forall 1 \le k \le n $
	$$ \exist \delta > 0 : \forall |h| < \delta \quad X + he_k \in E $$
	Зафиксируем $ k $ и рассомтрим функцию $ g(y) \define f(x_1, ..., \underset{k}y, ..., x_n), \qquad y \in (x_k - \delta, x_k + \delta) $ \\
	$ x_k $ -- точка локального экстремума
	$$ \exist g'(x_k) = f_{x_k}'(X) $$
	По теореме Ферма,
	$$ g'(X_k) = 0 \implies f_{x_k}'(X) = 0 $$
\end{proof}

Начиная с этого момента, будем трактовать $ \R^n $ как пространство вектор-столбцов, т. е.
$$ X =
\begin{bmatrix}
	x_1 \\
	. \\
	. \\
	. \\
	x_n
\end{bmatrix} \in \R^n, \qquad X^T = (x_1, ..., x_n) $$
Дифференциал можно записать как $ \di f(X, H) = \grad f(X) H $

\begin{definition}
	$ n \ge 2, \quad k \ge 1, \qquad L : \R^n \to \R^k $ \\
	$ L $ называется линейным, если
	\begin{enumerate}
		\item $ L(cX) = cL(X), \qquad c \in \R $
		\item $ L(X_1 + X_2) = L(X_1) + L(X_2) $
	\end{enumerate}
\end{definition}

\begin{theorem}
	$ L $ -- линейное
	$$ \implies \exist! A_{k \times n} : L(X) = A_{k \times n}X $$
\end{theorem}
