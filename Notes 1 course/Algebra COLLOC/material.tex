\section{Матричная запись квадратичной формы. Изменение матрицы при линейном преобразовании}

\begin{definition}
	Квадратичной формой называется однородный многочлен $ f = f(x_1, x_2, ..., x_n) $ второй степени от $ n $ переменных
\end{definition}

\begin{notation}
	Считая, что в форме $ f = f(x_1, x_2, ..., x_n) $ уже выполнено приведение подобных членов, обозначим коэффициент при $ x_i^2 $ через $ a_{ii} $, а коэффициент при $ \underset{(i \ne j)}{x_ix_j = x_jx_i} $ через $ 2a_{ij} = a_{ij} + a_{ji} = 2a_{ji} $, так что
	\begin{equ}5
		a_{ij} = a_{ji}
	\end{equ}
\end{notation}

\begin{notation}
	Для члена, содержащего $ x_ix_j$, мы получим следующую симметричную форму записи:
	$$ 2a_{ij}x_ix_j = a_{ij}x_ix_j + a_{ji}x_jx_i $$
\end{notation}

\begin{notation}
	Вся квадратичная форма $ f $ может быть записана теперь в виде
	\begin{equ}6
		\begin{aligned}
			f(x_1, x_2, ..., x_n) = a_{11}x_1^2 + a_{12}x_1x_2 + ... + a_{1n}x_1x_n + \\ + a_{21}x_2x_1 + a_{22}x_2^2 + ... + a_{2n}x_2x_n + \\ + \widedots + \\ + a_{n1}x_nx_1 + a_{n2}x_nx_2 + ... + a_{nn}x_n^2
		\end{aligned}
	\end{equ}
\end{notation}

\begin{definition}
	Составленная из коэффициентов матрица
	$$ A =
	\begin{Vmatrix}
		a_{11} & a_{12} & ... & a_{1n} \\
		a_{21} & a_{22} & ... & a_{2n} \\
		. & . & . & . \\
		a_{n1} & a_{n2} & ... & a_{nn}
	\end{Vmatrix} $$
	называется матрицей квадратичной формы $ f(x_1, x_2, ..., x_n) $
\end{definition}

Ввиду условия \eref5 элементы матрицы $ A $, расположенные симметрично относительно главной диагонали, равны между собой. Следовательно, $ A' = A $

Очевидно, что для любой симметричной матрицы $ A $ всегда можно указать такую квадратичную форму, что её матрица совпадает с $ A $. Если две квадратичные формы имеют одну и ту же матрицу, то они могут отличаться друг от друга только обозначениями переменных, и мы можем считать их одинаковыми. Таким образом, квадратичные формы определятся своими матрицами

\begin{remark}
	Для квадратичной формы может быть дана компактная матричная запись, а именно, мы можем выражение \eref6 записать в виде:
	\begin{multline*}
		f(x_1, x_2, ..., x_n) =
		\begin{Vmatrix}
			x_1 & x_2 & ... & x_n
		\end{Vmatrix} \cdot
		\begin{Vmatrix}
			a_{11}x_1 + a_{12}x_2 + ... + a_{1n}x_n \\
			a_{21}x_1 + a_{22}x_2 + ... + a_{2n}x_n \\
			\widedots \\
			a_{n1}x_1 + a_{n2}x_2 + ... + a_{nn}x_n
		\end{Vmatrix} = \\ =
		\begin{Vmatrix}
			x_1 & x_2 & ... & x_n
		\end{Vmatrix} \cdot
		\begin{Vmatrix}
			a_{11} & a_{12} & ... & a_{1n} \\
			a_{21} & a_{22} & ... & a_{2n} \\
			. & . & . & . \\
			a_{n1} & a_{n2} & ... & a_{nn}
		\end{Vmatrix} \cdot
		\begin{Vmatrix}
			x_1 \\
			x_2 \\
			\vdots \\
			x_n
		\end{Vmatrix}
	\end{multline*}
	Таким образом,
	\begin{equ}7
		f(x_1, x_2, ..., x_n) = X'AX
	\end{equ}
\end{remark}

Возьмём квадратичную форму $ f(x_1, x_2, ..., x_n) $ и преобразование:
\begin{equ}8
	\begin{aligned}[c]
		x_1 = c_{11}y_1 + c_{12}y_2 + ... + c_{1n}y_n \\
		x_2 = c_{21}y_1 + c_{22}y_2 + ... + c_{2n}y_n \\
		\widedots \\
		x_n = c_{n1}y_1 + c_{n2}y_2 + ... + c_{nn}y_n
	\end{aligned}
\end{equ}
Обозначим через $ C $ матрицу преобразования \eref8, а через $ X $ и $ Y $ -- столбцы из старых и новых переменных соответственно. Преобразование \eref8 принимает следующую матричную форму:
\begin{equ}9
	X = CY
\end{equ}
Будем рассматривать только неособенные линейные преобразования (т. е. будем предполагать, что $ |C| \ne 0 $). В этом случае для преобразования \eref9 существует обратное преоразование
\begin{equ}{10}
	Y = C^{-1}X
\end{equ}
Подставив выражения для $ x_1, x_2, ..., x_n $ из преобразования \eref8 в квадратичную форму $ f(x_1, x_2, ..., x_n) $, получаем
$$ f(x_1, x_2, ..., x_n) = f \bigg( \sum_{j = 1}^n c_{1j}u_j, \sum_{j = 1}^n c_{2j}y_j, ..., \sum_{j = 1}^n c_{nj}y_j \bigg) $$
После выполнения всех необходимых действий правая часть превратится, очевидно, в квадратичную форму $ g(y_1, y_2, ..., y_n) $ от новых переменных $ y_1, y_2, ..., y_n $
\begin{definition}
	Будем говорить в этом случае, что форма $ g(y_1, y_2, ..., y_n) $ получена из $ f(x_1, x_2, ..., x_n) $ в результате линейного преобразования переменных \eref8
\end{definition}

Очевидно, если полученную форму $ g(y_1, y_2, ..., y_n) $ подвергнуть обратному линейному преобразованию переменных \eref{10}, то мы вернёмся к исходной форме $ f(x_1, x_2, ..., x_n) $

\begin{theorem}
	Если в квадратичной форме с матрицей $ A $ сделано линейное преобразовавние переменных с матрицей $ C $, то полученная квадратичная форма будет иметь матрицу $ C'AC $
\end{theorem}

\begin{proof}
	Подвергнем форму \eref7 преобразованию \eref9. Так как $ X' = (CY)' = Y'C' $, то мы получим
	$$ f = X'AX = Y'C'ACY = g(y_1, y_2, ..., y_n) $$
	Рассмотрим квадратную матрицу $ B \define C'AC $. Так как $ B' = (C'AC)' = C'A'C'' = C'AC = B $, то матрица $ B $ симметрична, а значит, она и является матрицей квадратичной формы $ g = Y'BY $
\end{proof}

\section{Теорема Лагранжа}

\begin{lemma}\label{le:1}
	Произведение двух (или нескольких) последовательно выполненных неособенных линейных преобразований переменных является также неособенным преобразованием
\end{lemma}

\begin{proof}
	Неособенность преобразований $ X = CY $ и $ Y = DZ $ означает, что матрицы $ C $ и $ D $ неособенные. Но тогда матрица $ CD $, т. е. матрица линейного преобразования $ X = C(DZ) = (CD)Z $ также неособенная
\end{proof}

\begin{lemma}\label{le:2}
	Если у квадратичной формы $ f(y_1, ..., y_n) $ имеется хотя бы один ненулевой коэффициент, то надлежацим неособенным линейным преобразованием переменных она может быть преобразована в форму, у которой коэффициент при $ y_1^2 $ отличен от нуля
\end{lemma}

\begin{proof}
	\hfill
	\begin{itemize}
		\item Пусть $ a_{11} \ne 0 $ \\
		В этом случае сама форма $ f $ обладает требуемым свойством
		\item Пусть $ a_{11} = 0 $, но при некотором $ i \ge 2 $ отличен от нуля коэффициент при $ x_i^2 $, т. е. $ a_{ii} \ne 0 $ \\
		В этом случае достаточно изменить нумерацию переменных, т. е. сделать преобразование вида
		$$ x_1 \define y_i, \qquad x_i \define y_1, \qquad x_k \define y_k \quad \text{при }
		\begin{cases}
			k \ne 1 \\
			k \ne i
		\end{cases} $$
		Это преобразвание, очевидно, неособенное, и после его выполнения получим $ f = ... + a_{ii}x_i^2 + ... $ \\
		Выписанный член $ a_{ii}y_1^2 $ не имеет себе подобных, а потому он сократиться не может.
		\item Пусть $ a_{11} = a_{22} = ... = a_{nn} = 0 $, то есть равны нулю все диагональные коэффициенты \\
		По условию, форма имеет хоть один ненулевой коэффициент. Пусть $ a_{ij} \ne 0 ~ (i \ne j) $ \\
		Чтобы свести рассматривемый случай к предыдущему, нам достаточно сделать какое-нибудь неособенное преобразование, только бы появился квадрат одной из переменных с ненулевым коэффициентом. \\
		Сделаем, например, преобразование
		$$ x_j \define y_j + y_i, \qquad x_k \define y_k \quad \text{при } k \ne j \qquad (\text{в частности, } x_i = y_i) $$
		Преобразование, как легко видеть, неособенное. После выполнения преобразования получим
		$$ f = ... + 2a_{ij}x_ix_j + ... = ... + 2a_{ij}y_i(y_j + y_i) + ... = ... + 2a_{ij}y_iy_j + 2a_{ij}y_i^2 + ... $$
		Член $ 2a_{ij}y_i^2 $ является здесь единственным членом с $ y_i^2 $, поэтому после приведения подобных членов он не сократится \\
		Полученная нами форма содержит, таким образом, квадрат переменной с ненулевым коэффициентом ($ 2a_{ij} \ne 0 $), и, значит, к ней применимо рассуждение предыдущего пункта \\
		Для завершения доказательства остаётся сослаться на лемму \ref{le:1}
	\end{itemize}
\end{proof}

\begin{theorem}[Лагранжа]
	Всякая квадратичная форма при помощи неособенного линейного преобразования переменных может быть приведена к диагональному виду
\end{theorem}

\begin{proof}
	\textbf{Индукция} по числу переменных $ n $
	\begin{itemize}
		\item \textbf{База.} $ n = 1 $ \\
		Устверждение теоремы тривиально: всякая квадратичная форма от одной переменной имеет вид $ ax_1^2 $ и является, следовательно, диагональной (всякая матрица первого порядка диагональна)
		\item \textbf{Переход.} Предположим, что $ n \ge 2 $ и  для форм от $ n - 1 $ переменных теорема уже доказана \\
		Пусть $ f(x_1, x_2, ..., x_n) $ -- квадратичная форма от $ n $ переменных
		\begin{itemize}
			\item Если все коэффициенты формы $ f $ -- нули, то доказывать нечего
			\item Пусть не все её коэффициенты нули \\
			Если $ a_{11} = 0 $ (в обозначениях \eref6), то, согласно лемме \ref{le:2}, можно совершить неособенное линейное преобразование переменных так, чтобы после преобразования формы коэффициент при квадрате первой переменной был отличен от нуля \\
			Поэтому можно считать, что $ a_{11} \ne 0 $ \\
			Выделим в форме \eref6 все члены, содержащие $ x_1 $:
			$$ f = a_{11}x_1^2 + 2a_{12}x_1x_2 + ... + 2a_{1n}x_1x_n + g(x_2, ..., x_n) $$
			Здесь $ g $ является, очевидно, формой от $ n - 1 $ переменных $ x_2, ..., x_n $. Преобразуем теперь выписанную сумму так, чтобы все члены с $ x_1 $ вошли в квадрат линейного выражения:
			\begin{multline*}
				f = a_{11} \bigg( x_1^2 + 2x_1 \big( \frac{a_{12}}{a_{11}}x_2 + ... + \frac{a_{1n}}{a_{11}}x_n \big) \bigg) + g(x_2, ..., x_n) = \\ = a_{11} \bigg( x_1 + \frac{a_{12}}{a_{11}}x_2 + ... + \frac{a_{1n}}{a_{11}}x_n \bigg)^2 - a_{11} \bigg( \frac{a_{12}}{a_{11}}x_2 + ... + \frac{a_{1n}}{a_{11}}x_n \bigg)^2 + g(x_2, ..., x_n) = \\ = a_{11} \bigg( x_1 + \frac{a_{12}}{a_{11}}x_2 + ... + \frac{a_{1n}}{a_{11}}x_n \bigg)^2 + f_1(x_2, ..., x_n)
			\end{multline*}
			Здесь $ f_1 $, как и $ g $, является квадратичной формой от $ n - 1 $ переменных $ x_2, ..., x_n $ \\
			Сделаем преобразование переменных
			\begin{equ}{11}
				\left.
				\begin{aligned}
					y_1 = x_1 + \frac{a_{12}}{a_{11}}x_2 + ... + \frac{a_{1n}}{a_{11}}x_n \\
					y_2 = x_2 \\
					\widedots[5em] \\
					y_n = x_n
				\end{aligned} \right\}
			\end{equ}
			Преобразование \eref{11}, очевидно, неособенное. Найдём для него обратное преобразование:
			\begin{equ}{12}
				\left.
				\begin{aligned}
					x_1 = y_1 - \frac{a_{12}}{a_{11}}y_2 - ... - \frac{a_{1n}}{a_{11}}y_n \\
					x_2 = y_2 \\
					\widedots[5em] \\
					x_n = y_n
				\end{aligned} \right\}
			\end{equ}
			После выполнения преобразования \eref{12} (или \eref{11}) форма примет вид $ f = a_{11}y_1^2 + f_1(y_2, ..., y_n) $ \\
			По индукционному предположению, существует неособенное линейное преобразование
			$$
			\begin{cases}
				y_2 = c_{21}z_2 + ... + c_{2n}z_n \\
				\widedots[10em] \\
				y_n = c_{n1}z_2 + ... + c_{nn}z_n
			\end{cases} $$
			при котором форма $ f_1 $ приводится к диагональному виду $ f_1(y_2, ..., y_n) = a_2z_2^2 + ... + a_nz_n^2 $ \\
			Для исходной формы $ f $ вслед за пробразованием \eref{12} выполним следующее преобразование переменных:
			\begin{equ}{13}
				\begin{rcases}
					y_1 = z_1 \\
					y_2 = c_{22}z_2 + ... + c_{2n}z_n \\
					\widedots \\
					y_n = c_{n2}z_2 + ... + c_{nn}z_n
				\end{rcases}
			\end{equ}
			Определитель этого преобразования:
			$$
			\begin{vmatrix}
				1 & 0 & ... & 0 \\
				0 & c_{22} & ... & c_{n2} \\
				. & . & . & . \\
				0 & c_{n2} & ... & c_{nn}
			\end{vmatrix} =
			\begin{vmatrix}
				c_{22} & ... & c_{2n} \\
				. & . & . \\
				c_{n2} & ... & c_{nn}
			\end{vmatrix} \ne 0 $$
			Следовательно, преобразование \eref{13} неособенное. После его выполнения вслед за преобразованием \eref{12} форма $ f $ приобретает вид $ f(x_1, x_2, ..., x_n) = a_{11}z_1^2 + a_2z_2^2 + ... + a_nz_n^2 $
		\end{itemize}
	\end{itemize}
\end{proof}

\section{Закон инерции квадратичных форм}

\begin{definition}
	Если в квадратичной форме $ f(x_1, x_2, ..., x_n) $ вместо переменных подставить какие-нибудь числовые значения $ x_1^*, x_2^*, ..., x_n^* $ и произвести все необходимые вычисления, то в результате получим некоторое число. Это число называется значением квадратичной формы $ f $ при заданных значениях переменных
\end{definition}

\begin{notation}
	$ f(x_1^*, x_2^*, ..., x_n^*) $
\end{notation}

\begin{remark}
	Пусть для формы $ f(x_1, x_2, ..., x_n) $ выполнено неособенное линейное преобразование переменных
	\begin{equ}{32}
		x_i = \sum_{j = 1}^n c_{ij}y_j \qquad i = 1, 2, ..., n
	\end{equ}
	с матрицей $ C $, в результате которого форма $ f $ перешла в форму $ g(y_1, y_2, ..., y_n) $ \\
	В силу неособенности матрицы $ C $ для произвольных значений $ x_1^*, x_2^*, ..., x_n^* $ существует единственный набор значений $ y_1^*, y_2^*, ..., y_n^* $ такой, что
	$$ x_i^* = \sum_{j = 1}^n c_{ij}y_j^* \qquad i = 1, 2, ..., n $$
	Значения $ y_1^*, y_2^*, ..., y_n^* $ мы будем называть значениями, соответствующими значениям $ x_1^*, x_2^*, ..., x_n^* $ при преобразовании \eref{32} \\
	Утверждается, что
	\begin{equ}{33}
		f(x_1^*, x_2^*, ..., x_n^*) = g(y_1^*, y_2^*, ..., y_n^*)
	\end{equ}
	В самом деле, форма $ g $ определяется равенством
	$$ g(y_1, ..., y_n) = f(\sum c_{1j}y_j, ..., \sum c_{nj}y_j) $$
	Поэтому
	$$ g(y_1^*, ..., y_n^*) = f(\sum c_{1j}y_j^*, ..., \sum c_{nj}y_j^*) = f(x_1^*, ..., x_n^*) $$
	Ясно, что значения $ x_1^*, ..., x_n^* $ и $ y_1^*, ..., y_n^* $ однозначно определяются друг через друга не только преобразованием \eref{32}, но и обратным ему
\end{remark}

\begin{theorem}[Закон инерции квадратичных форм]
	Если вещественная квадратичная форма вещественными неособенными линейными преобразованиями переменных приедена двумя способами к диагональному виду, то в обоих случаях число положительных коэффициентов, число отрицательных коэффициентов и число нулевых коэффициентов при квадратах новых переменных одно и то же
\end{theorem}

\begin{proof}
	Пусть вещественная квадратичная форма $ f(x_1, x_2, ..., x_n) $ при помощи вещественных неособенных линейных преобразований переменных приведена к диагональному виду:
	\begin{equ}{34}
		f(x_1, x_2, ..., x_n) = \alpha_1y_1^2 + ... + \alpha_py_p^2 - \beta_1y_{p + 1}^2 - ... - \beta_qy_{p + q}^2
	\end{equ}
	\begin{equ}{35}
		f(x_1, x_2, ..., x_n) = \gamma_1z_1^2 + ... + \gamma_rz_r^2 - \delta_1z_{r + 1}^2 - ... - \delta_sz_{r + s}^2
	\end{equ}
	Мы считаем, что здесь все $ \alpha_i, \beta_j, \gamma_k, \delta_l $ строго положительны, так что форма \eref{34} при квадратах новых переменных имеет $ p $ положительных коэффициентов, $ q $ отрицательных и $ n - p - q $ нулевых, а форма \eref{35} -- $ r $ положительных, $ s $ отрицательных и $ n - r - s $ нулевых
	Мы должны показать, что
	\begin{equ}{36}
		p = r, \qquad q = s, \qquad n - p - q = n - r - s
	\end{equ}
	Выпишем соответствующие линейные преобразования переменных (вещественные и неособенные), выразив при этом \textbf{новые} переменные через \textbf{старые}:
	\begin{equ}{37}
		y_i = \sum_{j = 1}^n c_{ij}x_j \qquad i = 1, 2, ..., n
	\end{equ}
	\begin{equ}{38}
		z_i = \sum_{j = 1}^n d_{ij}x_j \qquad i = 1, 2, ..., n
	\end{equ}
	\begin{undefthm}{План доказательства}
		Предположим, вопреки утверждению теоремы, что $ p \ne r $ \\
		Для определённости можно, конечно, считать, что $ p < r $ \\
		Мы покажем, что тогда для старых пременных можно найти такие вещественные значения $ x_1^*, x_2^*, ..., x_n^* $, не равные одновременно нулю, что соответствующие им значения новых переменных $ y_1^*, y_2^*, ..., y_n^* $ и $ z_1^*, z_2^*, ..., z_n^* $ будут удовлетворять условиям
		\begin{equ}{39}
			y_1^* = 0, \quad ..., \quad y_p^* = 0
		\end{equ}
		\begin{equ}{310}
			z_{r + 1}^* = 0, \quad ..., \quad z_n^* = 0
		\end{equ}
	\end{undefthm}
	Предположим, что $ p \ne r $. Рассмотрим случай, когда $ p < r $: \\
	Рассмотрим следующую систему линейных однородных уравнений:
	\begin{equ}{311}
		\begin{rcases}
			c_{11}x_1 + c_{12}x_2 + ... + c_{1n}x_n = 0 \\
			\widedots \\
			c_{p1}x_1 + c_{p2}x_2 + ... + c_{pn}x_n = 0 \\
			d_{r + 1, 1}x_1 + d_{r + 1, 2}x_2 + ... + d_{r + 1, n}x_n = 0 \\
			\widedots \\
			d_{n1}x_1 + d_{n2}x_2 + ... + d_{nn}x_n = 0
		\end{rcases}
	\end{equ}
	Система эта получена приравниваем к нулю первых $ p $ линейных выражений справа в преобразовании \eref{37} и последних $ n - r $ линейных выражений в преобразовании \eref{38} \\
	Число уравнений в системе \eref{311} равно $ p + (n - r) = n - (r - p) $, что строго меньше $ n $ (поскольку $ r > p $) \\
	Но раз число уравнений в однородной системе меньше числа неизвестных, то она имеет ненулевое решение \\
	Ввиду вещественности преобразований \eref{37} и \eref{38} коэффициенты системы \eref{311} также вещественны, а значит, ненулевое решение этой системы мы можем выбрать вещественным \\
	Пусть это будет $ x_1^*, x_2^*, ..., x_n^* $ \\
	Этим мы и нашли искомые значения для старых переменных \\
	Если теперь в соответствии с преобразованиями \eref{37} и \eref{37} мы найдём соответствующие значения для новых переменных $ y_1^*, y_2^*, ..., y_n^* $ и $ z_1^*, z_2^*, ..., z_n^* $, то для этих переменных будут выполнены, очевидно, условия \eref{39} и \eref{310} \\
	Согласно равенствам \eref{34}, \eref{35} и формуле \eref{33} мы имеем
	\begin{equ}{312}
		f(x_1^*, ..., x_n^*) = -\beta_1y_{p + 1}^{*2} - ... - \beta_qy_{p + q}^{*2}
	\end{equ}
	\begin{equ}{313}
		f(x_1^*, ..., x_n^*) = \gamma_1z_1^{*2} + ... + \gamma_rz_r^{*2}
	\end{equ}
	Квадрат вещественного числа неотрицателен, поэтому из \eref{312} следует, что
	\begin{equ}{314}
		f(x_1^*, ..., x_n^*) \le 0
	\end{equ}
	\begin{note}
		Знак равенства здесь возможен ввиду того, что в случае $ p + q < n $ ненулевые значения $ y_i^* $ могут оказаться только среди значений $ y_{p + q + 1}^*, ..., y_n^* $
	\end{note}
	С другой стороны, в силу неособенности преобразования \eref{38} не все $ z_i^* $ равны нулю (в противном случае все $ x_i $ были бы равны нулю); поэтому, учитывая \eref{310}, мы заключаем, что хоть одно из значений $ z_1^*, ..., z_r^* $ отлично от нуля, а значит, из \eref{313} вытекает неравенство
	\begin{equ}{315}
		f(x_1^*, ..., x_n^*) > 0
	\end{equ}
	Таким образом, предположение о том, что $ p \ne r $, привело нас противоречащим друг другу неравенствам \eref{314} и \eref{315} \\
	Следовательно, $ p = r $, т. е. число положительных коэффициентов в обеих диагональных формах \eref{34} и \eref{35} одно и то же \\
	Второе из неравенств \eref{36} может быть доказано аналогично \\
	Третье из неравенств \eref{36} является очевидным следствием первых двух
\end{proof}

\section{Положително определённые квадратичные формы: критерии}

\begin{definition}
	Вещественная квадратичная форма называется положительно определённой, если положительны все её значения при вещественных значениях переменных, не равных нулю одновременно
\end{definition}

\begin{theorem}\label{th:42}
	Для того чтобы вещественная квадратичная форма была положительно определённой, необходимо и достаточно, чтобы при приведении её к диагональному виду вещественным неособенным линейным преобразованием переменных все коэффициенты при квадратах новых переменных были положительны
\end{theorem}

\begin{proof}
	Рассмотрим вещественное неособенное преобразование
	\begin{equ}{42}
		x_i = \sum_{j = 1}^n c_{ij}y_j \qquad i = 1, 2, ..., n
	\end{equ}
	Пусть вещественная форма $ f(x_1, ..., x_n) $ преобразованием \eref{42} приведена к диагональному виду $ a_1y_1^2 + a_2y_2^2 $ \\
	Если $ x_1^*, x_2^*, ..., x_n^* $ -- соответствующие им значения новых переменных, то, согласно формуле \eref{33}, мы имеем
	\begin{equ}{416}
		f(x_1^*, ..., x_n^*) = \alpha_1y_1^{*2} + \alpha_2y_2^{*2} + ... + \alpha_ny_n^{*2}
	\end{equ}
	\begin{itemize}
		\item Пусть все коэффициенты $ \alpha_1, \alpha_2, ..., \alpha_n $ положительны \\
		Тогда, если вещественные же значения $ y_1^*, ..., y_n^* $ также не все нули (в силу неособенности преобразования \eref{42}), и поэтому правая часть равенства \eref{416} (а значит, и его левая часть) положительны \\
		Таким образом, при положительных $ \alpha_i $ форма $ f $ положительно определённая
		\item Наоборот, пусть форма $ f $ положительно определённая \\
		Положим
		\begin{equ}{417}
			y_1^* \define 0, ..., y_{i - 1}^* \define 0, \quad y_i^* = 1, \quad y_{i + 1}^* = 0, ..., y_n^* = 0
		\end{equ}
		Соответствующие значения $ x_1^*, ..., x_n^* $ старых переменных, очевидно, вещественны и не равны нулю одновременно, так что $ f(x_1^*, ..., x_n^*) > 0 $ \\
		Но, согласно \eref{416} и \eref{417}, $ f(x_1^*, ..., x_n^*) = \alpha_i $, следовательно, $ \forall i = 1, 2, ..., n \quad \alpha_i > 0 $
	\end{itemize}
\end{proof}

\begin{definition}
	Квадратичную форму $ x_1^2 + x_2^2 + ... + x_n^2 $ мы будем называть чистой суммой квадратов
\end{definition}

\begin{remark}
	Чистая сумма квадратов в качестве своей матрицы имеет, очевидно, единичную матрицу
\end{remark}

\begin{theorem}
	Вещественная квадратичная форма является положительно определённой тогда и только тогда, когда она вещественным неособенным линейным преобразованием переменных может быть приведена к чистой сумме квадратов
\end{theorem}

\begin{proof}
	\hfill
	\begin{itemize}
		\item Если форма приводится к чистой сумме квадратов, то, согласно теореме \ref{th:42}, она положительно определённая
		\item Наоборот, пусть $ f(x_1, ..., x_n) $ -- положительно определённая квадратичная форма \\
		Приведём её вещественным неособенным преобразованием \eref{42} к диагональному виду
		\begin{equ}{418}
			f = \alpha_1y_1^2 + \alpha_2y_2^2 + ... + \alpha_ny_n^2
		\end{equ}
		По теореме \ref{th:42} все коэффициенты $ \alpha_i $ здесь положительны \\
		Сделаем вслед за преобразованием \eref{42} следующее преобразвание переменных:
		\begin{equ}{419}
			\begin{rcases}
				y_1 = \dfrac1{\sqrt{\alpha_i}}z_1 \\
				y_2 = \qquad \dfrac1{\sqrt{\alpha_2}}z_2 \\
				\widedots \\
				y_n = \qquad\qquad\qquad \dfrac1{\sqrt{\alpha_n}}z_n
			\end{rcases}
		\end{equ}
		При выполнении этого вещественного и неособенного преобразования форма \eref{418} приобретает вид $ z_1^2 + z_2^2 + ... + z_n^2 $, т. е. переходит в чистую сумму квадратов \\
		Для завершения доказательства остаётся только заметить, что последовательное выделение преобразований \eref{42} и \eref{419} равносильно одному вещественному неособенному преобразованию
	\end{itemize}
\end{proof}

\section{Векторное пространство. Определение, примеры, простешие \texorpdfstring{\\}{} свойства}

\begin{definition}
	$K$ -- поле, $V$ -- множество. Заданы операции сложения на $V$ ($V \times V \to V$) и умножения на скаляр ($V \times K \to V$) \\
    Множество $V$ называется векторным пространством над $K$, если выполнены следующие свойства:
    \begin{enumerate}
    	\item $V$ -- абелева группа по сложению
        \item Дистрибутивность: $ a(u + v) = au + av, \qquad \forall a \in K, \quad u, v \in V $
        \item Дистрибутивность: $ (a + b)u = au + bu, \qquad \forall a, b \in K, \quad u \in V $
        \item Ассоциативность: $ a(bu) = (ab)u, \qquad \forall a, b \in K, \quad u \in U $
        \item $ 1 \cdot u = u, \qquad 1 \in K, \quad \forall u \in U $
    \end{enumerate}
    Элементы $V$ называют векторами, элементы $K$ -- скалярами
\end{definition}

\begin{exmpls}
	\item Геометрические векторы на плоскости -- векторное пространство над $\R$
    \item $\R^n$ -- векторное пространство над $\R$
    \item $K^n$ -- векторное пространство над $K$, где $K$ -- поле
    \item $M_{m \times n}$ -- векторное пространство над $\R$
    \item $\Co$ -- векторное пространство над $\R$
    \item $K[x]$ -- векторное пространство над $K$
    \item Множество многочленов степени $\bm\le n$ -- векторное пространство над $K$
\end{exmpls}

\begin{props}
    \item $ 0 \cdot u = \vect{0}, \qquad \forall u \in V $
    \begin{proof}
    	$ 0 \cdot u = (0 + 0)u = 0 \cdot u + 0 \cdot u $ \\
        $ 0 = 0 \cdot u = 0 \cdot u + 0 \cdot u $ \\
        $ 0 = 0 \cdot u $
    \end{proof}
    \item $ a \cdot \vect{0} = \vect{0}, \qquad \forall a \in K $
    \item $a \cdot u = 0 \implies a = 0 $ или $ u = \vect{0} $
\end{props}

\section{Линейные комбиинации, линейная зависимость}

\begin{definition}
	Линейной комбинацией векторов $u_1, ..., u_k \in V$ называется вектор
    $$ a_1u_1 + ... + a_ku_k, \quad a_i \in K $$
    $a_i$ -- коэффициенты
\end{definition}

\begin{definition}
	Линейная комбинация называется тривиальной, если все коэффициенты равны нулю
\end{definition}

\begin{definition}
	Векторы $u_i$ называются линейно зависимыми, если существует их нетривиальная линейная комбинация, равная нулю \\
    Иначе -- линейно независимые
\end{definition}

\begin{props}
	\item
    \begin{enumerate}
    	\item Векторы линейно зависимы $\iff$ один из векторов является ЛК остальных
        \begin{proof}
        	\hfill
            \begin{itemize}
            	\item $\impliedby$ \\
                Пусть $u_1$ -- ЛК, то есть $u_1 = a_2u_2 + ... + a_nu_n $ \\
                $ (-1)u_1 + a_2u_2 + ... + a_nu_n = 0 $ -- нетривиальная ЛК
                \item $\implies$ \\
                Пусть $a_1u_1 + ... + a_nu_n = 0 $ -- нетривиальная ЛК \\
                Пусть $a_1 \ne 0$
                $$ a_1 = -\frac{a_2}{u_1}u_2 - ... - \frac{a_n}{u_1}u_n $$
            \end{itemize}
        \end{proof}
        \item Если $u_1, ..., u_n$ ЛНЗ, а $u_1, ..., u_n, v$ ЛЗ, то $v$ является ЛК остальных
        \begin{proof}
            $u_1, ..., u_n, v$ ЛЗ $ \iff \exist a_1, ..., a_n $ (не все нули) $ : a_1u_1 + ... + a_nu_n + a_{n + 1}v = 0 $
            \begin{itemize}
            	\item Если $a_{n + 1} \ne 0 $, то можно выразить $v$
                \item Если $a_{n + 1} = 0 $, то: \\
                Не все $a_i$ равны 0, $a_1u_1 + ... + a_nu_n = 0 $ -- нетривиальная. Противоречие
            \end{itemize}
        \end{proof}
    \end{enumerate}
    \item
    \begin{enumerate}
    	\item Если к ЛЗ добавить несколько векторов, то она останется ЛЗ
        \item Если из ЛНЗ убрать несколько векторов, то она останется ЛНЗ
    \end{enumerate}
    \item
    \begin{enumerate}
        \item \label{en:1} $ c \ne 0 \in K$. \\
        $u_1, ..., u_n$ ЛЗ $\iff cu_1, ..., u_n$ ЛЗ
        \item \label{en:2} $c \in K$ \\
        $u_1, ..., u_n$ ЛЗ $\iff u_1 + cu_2, u_2, ..., u_n $ ЛЗ
    \end{enumerate}
    \begin{proof}
    	$$ u_1' \define
        \begin{cases}
            cu_1 \qquad \eqref{en:1} \\
            u_1 + cu_2 \qquad \eqref{en:2}
        \end{cases} $$
        $$ u_1 =
        \begin{cases}
            \frac1cu_1' \qquad \eqref{en:1} \\
            u_1' + (-c)u_2 \qquad \eqref{en:2}
        \end{cases} $$
        Набор $u_1, ..., u_n$ получается из $u_1', u_2, ..., u_n$ преобразованием того же типа \\
        Достаточно доказать $\implies$
        \begin{enumerate}
        	\item Пусть $a_1u_1 + ... + a_nu_n = 0$, не все $a_i$ равны 0
            $$ \frac{a}c u_1' + a_2u_2 + ... + a_nu_n = 0, \quad \text{не все коэфф. равны 0} $$
            \item $a_1u_1 + a_2u_2 + ... + a_nu_n = 0$, не все $a_i$ равны 0
            $$ a_1u_1' + (a_2 - ca_1)u_2 + ... + a_nu_n = 0 $$
            $$ a_1(u_1 + cu_2) + ... $$
            Пусть $ a_1 = a_2 - ca_1 = a_3 = ... = a_n = 0 $
        \end{enumerate}
    \end{proof}
\end{props}

\begin{theorem}[линейная зависимость линейных комбинаций]
	Пусть $k > m$ и векторы $v_1, ..., v_k$ являются ЛК векторов $u_1, ..., u_m$ \\
    Тогда $v_1, ..., v_k$ ЛЗ
\end{theorem}

\begin{proof}
    \textbf{Индукция} по $m$
    \begin{itemize}
        \item \textbf{База.} $m = 1$ \\
        Есть вектор $u_1$. Все остальные -- его ЛК:
        $$ v_1 = a_1u_1, \qquad v_2 = a_2u_2, ... $$
        \begin{itemize}
        	\item $a_1 = 0 \implies v_1 = 0, \qquad 1 \cdot v_1 + 0 \cdot v_2 + 0 \cdot v_3 + ... = 0 $
            \item $a_1 \ne 0 $
            $$ v_2 = a_2u_1 = a_2 \cdot \frac{v_1}{a_1} $$
            $$ \frac{a_2}{a_1} v_1 + (-1) \cdot v_2 + 0 \cdot v_3 + ... = 0 $$
        \end{itemize}
        \item \textbf{Переход.} $ m - 1 \to m $
        $$ v_1 = a_{11}u_1 + a_{12}u_2 + ... + a_{1m}u_m $$
        $$ \widedots $$
        $$ v_k = a_{k1}u_! + a_{k2}u_2 + ... + a_{km}u_m $$
        Исключим $u_1$ из всех векторов, кроме первого:
        \begin{itemize}
            \item $a_{11} = a_{21} = ... = a_{k1} = 0 $ \\
            Применяем индукционное предположение к $v_1, ..., v_k$ и $u_2, ..., u_m$
            \item Пусть не все $a_{i1}$ равны нулю. НУО считаем, что $a_{i1} \ne 0$ \\
            При $i > 1$ положим $v_i' = v_i - \dfrac{a_{i1}}{a_{11}}v_1 $ \\
            Векторы $v_2', v_3', ..., v_k'$ являются ЛК $u_2, u_3, ..., u_m$ \\
            $ k - 1 > m - 1 $ \\
            По индукционному предположению, $v_2', ..., v_k'$ ЛЗ \\
            Добавим к этому набору $v_1$ (пользуемся свойством 2a) \\
            Воспользуемся свойством 3b: \\
            $ v_1, v_2, ..., v_k $ ЛЗ
        \end{itemize}
    \end{itemize}
\end{proof}

\section{Конечномерное пространство. Порождающие и линейно независимые системы}

\begin{definition}
    Пусть $V$ -- векторное пространство \\
    Множество векторов $ \set{v_i}$ называется порождающим для $V$, если любой вектор $v \in V$ является ЛК некоторого конечного подмножества $\set{v_i}$
\end{definition}

\begin{definition}
	Если у $V$ есть конечная порождаящая система, то $V$ называется конечномерным \\
    Иначе -- бесконечномерным
\end{definition}

\begin{property}
	Пусть $V$ -- конечномерное \\
    Тогда в $V$ \textbf{не} существует сколь угодно больших ЛНЗ систем
\end{property}

\begin{undefthm}{Другая формулировка}
	$ \exist N : \forall k > N \quad \forall v_1, ..., v_k \in V \quad v_1, ..., v_k$ ЛЗ
\end{undefthm}

\begin{proof}
	Пусть $u_1, ..., u_N$ -- конечная порождающая система. По теореме о линейной зависимости линейных комбинаций $v_1, ..., v_k$ ЛЗ
\end{proof}

\begin{theorem}[порождающие и ЛНЗ системы]
	Пусть $V$ -- конечномерное пространство
    \begin{enumerate}
		\item Пусть $u_1, ..., u_n$ -- минимальная по включению\footnotemark порождающая система. Тогда она ЛНЗ
        \begin{proof}
        	Пусть $u_1, ..., u_n$ -- ЛЗ \\
            Тогда некоторый вектор -- ЛК остальных. Пусть это $u_n$
            $$ u_n = c_1u_1 + c_2u_2 + ... + c_{n - 1}u_{n - 1} $$
            Докажем, что $u_1, ..., u_{n - 1}, u_n$ -- не минимальная, то есть, что $u_1, ..., u_{n - 1}$ -- тоже порождающая \\
            Пусть $v \in V, \qquad v = a_1u_1 + ... + a_{n - 1}u_{n - 1} + a_nu_n $
            $$ v = a_1u_1 + ... + a_n \bigg( c_1u_1 + ... \bigg) = (a_1 - a_nc_1)u_1 + ... + (a_{n - 1} + a_nc_{n - 1})u_{n - 1} $$
        \end{proof}
        \item Пусть $u_1, ..., u_n$ -- максимальная по включению ЛНЗ. Тогда она порождающая
        \begin{proof}
        	Пусть $v \in V$ \\
            $u_1, ..., u_n$ -- ЛНЗ, $u_1, ..., u_n, v$ -- ЛЗ (т. к. $u_i$ -- минимальная) \\
            Применяем свойство 1b
        \end{proof}
    \end{enumerate}
\end{theorem}
\footnotetext{Если из неё убрать вектор, она перестанет быть порождающей. Не обязательно минимальная по количеству векторов}

\section{Равносильные определения базиса. Координаты}

\begin{definition}
	Пусть $V$ -- конечномерное векторное пространство \\
    Система векторов называется базисом $V$, если она ЛНЗ и порождающая
\end{definition}

\begin{theorem}[равносильные определения базиса]
	Следующие утверждения равносильны:
    \begin{enumerate}
		\item \label{it:81} $u_1, ..., u_n$ -- базис $V$
        \item \label{it:82} $u_1, ..., u_n$ -- максимальная по включению ЛНЗ
        \item \label{it:83} $u_1, ..., u_n$ -- минимальная по включению порождающая система
        \item \label{it:84} Любой вектор можно единственным образом представить в виде ЛК $u_i$
    \end{enumerate}
\end{theorem}

\begin{proof}
    Уже доказаны: $ \ref{it:82} \implies \ref{it:81} $, $ \ref{it:83} \implies \ref{it:81} $
    \begin{itemize}
        \item $ \ref{it:81} \implies \ref{it:82} $ \\
        $ u_i $ -- ЛНЗ \\
        Докажем, что $ u_1, ..., u_n, v $ -- ЛЗ для $ \forall v $ \\
        $ u_i $ -- порождающая $ \implies \exist a_i : v = a_1u_1 + ... + a_nu_n $ \\
        Оказалось, что $v$ -- ЛК $u_i \implies u_1, ..., u_n, v $ -- ЛЗ
        \item $ \ref{it:81} \implies \ref{it:83} $ \\
        $ u_i $ -- порождающая \\
        Пусть $ u_i $ не минимальная. Пусть $ u_1, ..., u_{n - 1} $ тоже порождающая $ \implies \exist a_i : un = a_nu_1 + ... + a_{n - 1}u_{n - 1} \implies u_i $ -- ЛЗ
        \item $ \ref{it:84} \iff \ref{it:81} $ \\
        Система порождающая и в \ref{it:81}, и в \ref{it:84} \\
        Нужно доказать, что представление единственно $ \iff $ ЛНЗ
        \begin{itemize}
            \item $ \ref{it:84} \implies \ref{it:81} $
            $$ 0 = 0 \cdot u_1 + ... + 0 \cdot u_n $$
            Представление нуля единственно. Значит, система ЛНЗ
            \item $ \ref{it:84} \implies \ref{it:81} $
            $$ v = a_1u_1 + ... + a_nu_n, \quad v = b_1u_1 + ... + b_nu_n $$
            $$ 0 = v - v = (a_1 - b_1)u_1 + ... + (a_n - b_n)u_n $$
            ЛНЗ $ \implies a_i - b_i = 0 $
        \end{itemize}
    \end{itemize}
\end{proof}

\begin{definition}
	Координатами вектора $v$ в базисе $u_1, ..., u_n $ называется такой набор $ a_i, ..., a_n \in K : v = a_1u_1 + ... + a_nu_n $
\end{definition}

\section{Дополнение до базиса. ``Спуск'' к базису. Количество элементов в разных базисах. Размерность}

\begin{properties}[базиса]
	$ V $ -- конечномерное векторное пространство
    \begin{enumerate}
    	\item Дополнение до базиса \\
        Любую ЛНЗ систему можно дополнить до базиса
        \begin{proof}
            $ u_1, ..., u_k $ -- ЛНЗ
            Если это не базис, можно добавить вектор так, что система останется ЛНЗ\footnotemark \\
            Докажем, что процесс когда-нибудь закончится: \\
            Пусть есть порождающая система из $n$ векторов $ \implies $ в любой ЛНЗ системе не более $n$ векторов
        \end{proof}
        \item ``Спуск'' к базису \\
        Из любой порождающей системы можно выбрать базис
        \begin{proof}
        	Если система не минимальна, будем убирать векторы по одному
        \end{proof}
        \item Количество векторов \\
        В любых двух базисах поровну элементов
        \begin{proof}
            Пусть $ u_1, ..., u_k $ и $ w_1, ..., w_m $ -- базисы \\
            Тогда, $u_i$ -- порождающая, и $w_i$ -- ЛНЗ \\
            По теореме о линейной зависимости линейной комбинации, $m \le k $ \\
            Аналогично, $ m \ge k $
        \end{proof}
    \end{enumerate}
\end{properties}
\footnotetext{Если ничего нельзя добавить, то она максимальная, и это базис}

\begin{definition}
	Пусть $V$ конечномерно \\
    Размерностью $V$ называется количество элементов в базисе
    \begin{notation}
        $ \dim V, \quad \dim_K V $
    \end{notation}
    Если $ V = \set{0} $, то $ \dim V = 0 $
\end{definition}

\section{Подспространство. Пересечение и сумма подпространтв. Формула Грассмана}

\begin{definition}
	Пусть $V$ -- векторное пространство над $K$, $U \sub V $ \\
    $U$ называется подпространством, если $U$ -- векторное пространство над $K$ с теми же опреациями
\end{definition}

\begin{definition}
	$ U, W $ -- подпространства $V$ \\
    Их суммой называется множество $ \set{u + w | u \in U, w \in W} $
\end{definition}

\begin{notation}
	$ U + W $
\end{notation}

\begin{definition}
	$ U_1, ..., U_n $ -- подпространства $V$
    $$ U_1 + ... U_n = \set{u_1 + ... + u_n | u_i \in U_i} $$
\end{definition}

\begin{remark}
	$ U_1 + U_2 + U_3 = (U_1 + U_2) + U_3 $
\end{remark}

\begin{props}
	\item Сумма подпространств является подпространством
    \item Пересечение подпространств является подпространством
\end{props}

\begin{theorem}[формула Грассмана]
	Пусть $ U, W $ -- конечномерные подпространства векторного пространства $V$ \\
    Тогда $ \dim (U + W) + \dim (U \cap W) = \dim U + \dim W $
\end{theorem}

\begin{proof}
    Пусть $ l_1, ..., l_k $ -- базис $ U \cap W \implies l_i $ -- ЛНЗ \\
    Дополним их до базиса $U : l_1, ..., l_k, u_1, ..., u_m $ -- базис $U$ \\
    Аналогично, $ l_1, ..., l_k, w_1, ..., w_n $ -- базис $W$ \\
    Достаточно доказать, что $ l_1, ..., l_k, u_1, ..., u_m, w_1, ..., w_n $ -- базис $ U + W $, так как тогда $ (k + m + n) + k = (k + m) + (k + n) $
	\begin{itemize}
		\item Докажем, что это порождающая система: \\
	    Пусть $ v \in U + W, \quad v = u + w $ \\
	    Разложим по базису:
	    $$ u = \sum a_il_i + \sum b_iu_i, \quad w = \sum a_il_i + \sum d_iw_i $$
	    Сложим:
	    $$ v = \sum(a_i + b_i)l_i + \sum b_i u_i + \sum d_i w_i $$
		\item Докажем ЛНЗ: \\
	    Пусть $ \sum a_il_i + \sum b_i u_i + \sum c_i w_i = 0 $
	    $$
	    \begin{rcases}
			\sum b_i u_i \in U \\
	        \sum b_i u_i = - \sum a_i b_i - \sum d_i w_i \in W
	    \end{rcases} \implies \sum b_iu_i \in U \cap W $$
	\end{itemize}
    $ l_1, ..., l_k $ -- базис $ U \cap W $
    $$ \exist c_i : \sum b_iu_i = \sum c_il_i \implies (-c1)l_1 + ... + (-c_k)l_k + b_1u_1 + ... + b_mu_m = 0 \implies c_i = 0, l_i = 0 $$
    $$ \exist a_ib_i + 0 + \sum d_iw_i = 0 \implies a_i = 0, d_i = 0 $$
\end{proof}

\section{Равносильные определения прямой суммы подпространств}

\begin{definition}\label{def:1}
	$ V $ -- векторное пространство, $U, W $ -- подпространства \\
    Сумма $ U + W $ называется прямой, если $ \forall v \in V $ представляется в виде $ u + w, \quad u \in U, w \in W $ единственным образом
\end{definition}

\begin{notation}
	$ U \oplus W $
\end{notation}

\begin{remark}
	Прямая сумма $U_1, ..., U_k$ определяется так же \\
    Если $ V = U_1 \oplus ... \oplus U_k $, то говорят, что $V$ раскладывается в прямую сумму $U_i$
\end{remark}

\begin{theorem}
	Равносильны определения прямой суммы в случае 2 подпространств $U$ и $W$ конечномерного просранства $V$:
    \begin{enumerate}
        \item \label{it:2:1} Сумма $U + W $ прямая (по определению \ref{def:1})
        \item \label{it:2:2} Если $ u + w = 0, \quad u \in U, w \in W $, то $u = 0, w = 0 $
        \item \label{it:2:3} $ U \cap W = \set{0} $
        \item \label{it:2:4} Объединение базисов $U$ и $W$ является базисом $U + W$
    \end{enumerate}
\end{theorem}

\begin{proof}
	\hfill
    \begin{itemize}
        \item $ \ref{it:2:1} \implies \ref{it:2:2} $ очевидно
        \item $ \ref{it:2:2} \implies \ref{it:2:1} $ \\
        Пусть $ u + w = u' + w' \implies (u - u') + (w - w') = 0 \implies u = u', w = w' $
        \item $ \ref{it:2:2} \implies \ref{it:2:3} $ \\
        Пусть $ v \in U \cap W \implies -v \in U \cap W $
        $$ v + (-v) = 0 \implies v = 0 $$
        \item $ \ref{it:2:3} \implies \ref{it:2:2} $ \\
        Пусть $ u + w = 0, \quad u \in U, w \in W $ \\
        $ \underset{\in U}u = -\underset{\in W}w \implies u \in U \cap W \implies u = 0 \implies w = 0 $
        \item $ \ref{it:2:3} \iff \ref{it:2:4} $
        $$ \dim U + \dim W = \dim (U + W) + \dim (U \cap W) $$
        $$ \ref{it:2:4} \iff \dim U + \dim W = \dim (U + W) \iff \dim (U \cap W) = 0 \iff U \cap W = 0 $$
    \end{itemize}
\end{proof}

\begin{theorem}
	Пусть $V$ --- конечномерное пространство, $U_1, ..., U_k $ -- подпространства \\
    Тогда следующие условия равносильны:
    \begin{enumerate}
        \item \label{it:3:1} Сумма $ U_1 + ... + U_k $ прямая
        \item \label{it:3:2} Если $ u_1 + ... u_k, \quad u_i \in U $, то $u_i = 0 $
        \item \label{it:3:3} $ \forall i \quad U_i \cap (U_1 + ... + U_{i - 1} + U_{i + 1} + ... + U_k) = \set{0} $
        \item \label{it:3:4} $ U_1 \cap U_2 = \set{0}, \quad (U_1 + U_2) \cap U_3 = \set{0}, \quad \widedots[3em] $
        \item \label{it:3:5} Объединение любых базисов $u_i$ является базисом $u_1 + ... + u_k $
    \end{enumerate}
\end{theorem}

\begin{proof}
    \hfill
    \begin{itemize}
        \item $ \ref{it:3:1} \implies \ref{it:3:2} $ очевидно
        \item $ \ref{it:3:2} \implies \ref{it:3:1} $ \\
        Пусть $ v = u_1 + ... + u_k = u_1' + ... + u_k' $
        $$ v - v = (u_1 - u_1') + ... + (u_k - u_k') = 0 \implies u_i = u_i' $$
        \item $ \ref{it:3:2} \implies \ref{it:3:3} $ \\
        Пусть $v \in U_1 \cap (U_2 + ... + U_k) $
        $$ v = u_1 + ... + u_k, \quad u_i \in U_i $$
        $$ v \in i \implies -v \in U_i $$
        $$ 0 = (-v) + u_2 + ... + u_k \implies v = 0, \quad u_2 = ... = u_k = 0 $$
        \item $ \ref{it:3:3} \implies \ref{it:3:4} $ \\
        Докажем, что $ (U_1 + ... + U_{i - 1}) \cap U_i = 0 $ \\
        Заметим, что $ U_1 + ... + U_{i - 1} \sub U_1 + ... + U_{i - 1} + U_{i + 1} + ... + U_k \implies (U_1 + ... + U_{i - 1}) \cap U_i \sub (U_1 + ... + U_{i - 1} + U_{i + 1} + ... + U_k) \cap U_i = \set{0} $
        \item $ \ref{it:3:4} \implies \ref{it:3:2} $ \\
        Пусть $ u_1 + ... + u_k = 0, \quad u_i \in U_i $ \\
        Пусть не все $ u_1, ..., u_k $ равны 0 \\
        Положим $ i \define \max\set{s | u_s \ne 0} $
        $$ u_1 + ... + u_{i - 1} + \underset{\ne 0}{u_i} = 0 \implies u_i = -u_1 - ... - u_{i - 1} \in U_1 + ... + U_{i - 1} \implies u_i \in (U_1 + ... + U_{i - 1}) \cap U_i = \set{0} $$
        \item $ \ref{it:3:4} \iff \ref{it:3:5} $ \\
        Пусть $n_i = \dim U_i $ \\
        Пусть $B$ -- объединение базисов $U_i$ \\
        Тогда $B$ -- порождающая система $U_1 + ... + U_k $ \\
        $B$ -- базис $ \iff B $ -- минимальная порождающая система $ \iff |B| = \dim (U_1 + ... + U_k) $ \\
		Положим $ W_i = (U_i + ... + U_{i - 1}) \cap U_i \qquad \forall i = 2, ..., k $
		\begin{multline*}
			\dim (U_1 + ... + U_{k - 1} + U_k) = \dim (U_1 + ... + U_{k - 1}) + \dim U_k - \dim W_k = \\ = \dim (U_1 + ... + U_{k - 2}) + \dim U_{k - 1} - \dim W_{k - 1} + \dim U_k - \dim W_k = \widedots[6em] = \\ = (\dim U_1 + ... + \dim U_k) - (\dim W_2 + ... + \dim W_k)
		\end{multline*}
		$$ \dim(U_1 + ... + U_k) = \dim U_1 + ... + \dim U_k \iff \forall i \quad \dim W_i = 0 \iff W_i = 0 $$
    \end{itemize}
\end{proof}

\begin{implication}
	Если $ V $ раскладывается в прямую сумму $ U_1, ..., U_k $, то $ \dim V = \dim U_1 + ... + \dim U_k $
\end{implication}

\section{Матрица перехода между базисами. Связь координат вектора в разных базисах}

\begin{definition}
	Пусть $ e_1, ..., e_n $ и $ e_1', ..., e_n' $ -- базисы векторного пространства $ V $ \\
	Матрицей перехода от $ e_i $ к $ e_i'$ называется матрица, у которой в $ i $-м столбце записаны координаты $ e_i' $ в базисе $ e_1, ..., e_k $
\end{definition}

\begin{notation}
	$ C_{e_i, e_i'}, \quad C_{e_i \to e_i'} $
\end{notation}

\begin{properties}
	$ e_i, e_i' $ -- базисы векторного пространства $ V $
	\begin{enumerate}
		\item Если $ X $ и $ X' $ -- столбцы одного и того же вектора в базисах $ e_i $ и $ e_i' $, то $ X = C_{e_i \to e_i'}X' $
		\begin{proof}
			Пусть $ X \define
			\begin{pmatrix}
				x_1 \\
				. \\
				. \\
				. \\
				x_n
			\end{pmatrix}, \quad X' \define
			\begin{pmatrix}
				x_1' \\
				. \\
				. \\
				. \\
				x_n'
			\end{pmatrix} $
			\begin{multline*}
				x_1e_1 + ... + x_ne_n = x_1'e_1' + ... + x_n'e_n' = \\ = x_1'(c_{11}e_1 + ... + c_{n1}e_1) + \widedots[3em] + x_i'(c_{1i}e_1 + ... + c_{ni}e_n) + \widedots[3em] + x_n'(c_{1n}e_1 + ... + c_{nn}e_n) = \\ = (c_{11}x_1' + ... + c_{1i}x_i' + ... + c_{1n}x_n')e_1 + \widedots[3em] + (c_{n1}x_1' + ... + c_{ni}x_i' + ... + c_{nn}x_n')e_n
			\end{multline*}
			$$
			\begin{cases}
				x_1 = c_{11}x_1' + ... + c_{nn}x_n' \\
				\widedots[10em] \\
				x_n = c_{n1}x_1' + ... + c_{nn}x_n'
			\end{cases} $$
		\end{proof}
		\item Если для любого вектора выполнено $ X = CX' $, то $ C $ -- матрица перехода
		\begin{proof}
			Пусть $ v \define e_i' $ -- базис, $ X, X' $ -- координаты в базисах $ e_i, e_i' $
			$$ X' =
			\begin{pmatrix}
				0 \\
				. \\
				. \\
				1 \\
				. \\
				. \\
				0
			\end{pmatrix} \cdot i, \qquad X =
			\begin{pmatrix}
				c_{11} & ... & c_{1n} \\
				. & . & . \\
				c_{n1} & ... & c_{nn}
			\end{pmatrix} \cdot
			\begin{pmatrix}
				0 \\
				. \\
				. \\
				1 \\
				. \\
				. \\
				0
			\end{pmatrix} \cdot i =
			\begin{pmatrix}
				c_{1i} \\
				. \\
				. \\
				. \\
				c_{ni}
			\end{pmatrix} $$
			$ e_i' = v = c_{1i}e_1 + ... + c_{ni}e_n \implies c_{n1}, ..., c_{ni} $ -- координаты $ e_i' $ в базисе $ e_1, ..., e_n $
		\end{proof}
		\item Матрица перехода обратима. Обратная к ней -- матрица перехода в другую сторону
		\begin{proof}
			Нужно доказать, что $ C_{e_i \to e_i'} \cdot C_{e_i' \to e_i} = E $ \\
			Пусть $ v \in V, \quad X, X' $ -- столбцы координат
			$$ C_{e_i \to e_i'} \cdot C_{e_i' \to e_i} \cdot X = C_{e_i \to e_i'} \cdot X' = X $$
		\end{proof}
	\end{enumerate}
\end{properties}

\section{Приведение матрицы к трапецевидной}

\begin{definition}
	Трапецевидная матрица:
	$$
	\begin{pmatrix}
		a_{11} & a_{12} & a_{13} & ... & a_{1n} \\
		0 & a_{22} & a_{23} & ... & a_{2n} \\
		. & . & . & . & . \\
		0 & ... & a_{rr} & ... & a_{rn} \\
		0 & . & . & . & 0 \\
		. & . & . & . & . \\
		0 & . & . & . & 0
	\end{pmatrix}, \qquad \exist r : a_{ii} \ne 0 \text{ при } i < r + 1 $$
\end{definition}

\begin{theorem}[приведение матрицы к трапецевидной]
	Любую матрицу можно превратить в трапецевидную элементарными преобразованиями строк и перестановками столбцов
\end{theorem}

\begin{proof}
	Пусть $ A $ -- матрица $ m \times n $ \\
	\textbf{Индукция} по $ m $
	\begin{itemize}
		\item \textbf{База.} $ m = 1 $
		$$ A =
		\begin{pmatrix}
			a_{11} & ... & a_{1n}
		\end{pmatrix} $$
		\begin{itemize}
			\item Если $ a_{11} = ... = a_{1n} = 0 $, то матрица трапецевидная
			\item Если нет, то переставим столбцы так, что $ a_{11} \ne 0 $. Получится трапецевидная матрица
		\end{itemize}
		\item \textbf{Переход.} $ m - 1 \to m $
		\begin{itemize}
			\item Если матрица нулевая, то она трапецевидная
			\item Иначе переставляем столбцы так, что $ a_{11} \ne 0 $
			$$
			\begin{pmatrix}
				a_{11} & a_{12} & ... \\
				a_{21} & . & . \\
				. & . & . \\
				a_{n1} & . & .
			\end{pmatrix} $$
			Вычтем из $ i $-й строки первую, умноженную на $ \dfrac{a_{i1}}{a_{11}} $. Получим нули в первом столбце:
			$$ A =
			\begin{pmatrix}
				a_{11} & a_{12} & ... & a_{1n} \\
				0 & & & \\
				. & & A' & \\
				0 & & &
			\end{pmatrix} $$
			Применяем индукционное предположение к $ A' $ \\
			При этом в первой строчке $ A $ элементы могут поменяться местами, а в первом столбце остаются нули
		\end{itemize}
	\end{itemize}
\end{proof}

\section{Ранг матрицы и элементарные преобразования}

\begin{definition}
	Ранг матрицы -- размер наибольшего ненулевого минора
\end{definition}

\begin{notation}
	$ r_A, \quad \rk A, \quad \operatorname{rank} A $
\end{notation}

\begin{theorem}[ранг и элементарные преобразования]
	При элементарных преобразованиях строк и столбцов ранг матрицы не меняется
\end{theorem}

\begin{proof}
	Пусть мы получили $ B $ из $ A $ одним преобразованием \\
	Достаточно доказать, что $ \rk B \le \rk A $ (т. к. $ B \to A $ аналогично) \\
	Достаточно рассматривать преобразования строк \\
	Пусть $ k > \rk A $, то есть все миноры $ A $ поряка $ k $ равны нулю \\
	Докажем, что минор $ B $ порядка $ k $ равен нулю (из этого будет следовать, что $ k > \rk B $):
	\begin{itemize}
		\item Перестановки строк: \\
		Миноры $ B $ -- миноры $ A $ с точностью до перестановки строк \\
		При перестановке строк определитель не меняется
		\item Умножение строки на число:
		$$ A =
		\begin{pmatrix}
			... \\
			a_i \\
			...
		\end{pmatrix} \to B =
		\begin{pmatrix}
			... \\
			ta_i \\
			...
		\end{pmatrix} $$
		\begin{itemize}
			\item Если минор не содержит $ i $-ю строку, он не изменится
			\item Если содержит, он умножится на $ k $ ($ 0 \cdot k = 0 $)
		\end{itemize}
		\item Прибавление строки, умноженной на число:
		$$ A =
		\begin{pmatrix}
			... \\
			s_i \\
			.. \\
			s_j \\
			...
		\end{pmatrix} \to B =
		\begin{pmatrix}
			.. \\
			s_i + s_j \\
			... \\
			s_j
		\end{pmatrix} $$
		\begin{itemize}
			\item Если минор \textbf{не} содержит $ i $-ю строку -- он не изменится
			\item Если минор содержит $ i $-ю и $ j $-ю строки -- он не изменится
			\item Если минор содержит $ i $-ю строку и \textbf{не} содержит $ j $-ю:
			$$
			\underset{\text{Минор } B}{
			\begin{vmatrix}
				. & . & . \\
				... & x_i + tx_j & ... \\
				. & . & .
			\end{vmatrix}} =
			\underset{\text{Минор } A}{
			\begin{vmatrix}
				. & . & . \\
				... & x_i & ... \\
				. & . & .
			\end{vmatrix}} + t \cdot \underset{\text{Минор } A \text{ с переставленными строками}}{
			\begin{vmatrix}
				. & . & . \\
				... & x_j & ... \\
				. & . & .
			\end{vmatrix}} $$
			Все миноры $ A $ равны нулю, значит их сумма равна нулю
		\end{itemize}
	\end{itemize}
\end{proof}

\section{Ранг как размерность}

\begin{theorem}[ранг как размерность]
	Пусть $ A $ -- матрица $ m \times n $ с коэффициентами из поля $ K $ \\
	Тогда
	\begin{enumerate}
		\item Размерность подпространства пространства $ K^n $, порождённого строками $ A $, равна $ \rk A $
		\item Размерность подпространства пространства $ K^m $, порождённого столбцами $ A $, равна $ \rk A $
	\end{enumerate}
\end{theorem}

\begin{proof}
	Достаточно доказать для строк (т. к. ранг при транспонировании не меняется) \\
	Приведём $ A $ к трапецевидной форме элементарными преобразованиями строк и перестановками столбцов \\
	Пусть $ A \to A' $ \\
	По теореме о ранге и элем. преобр., $ \rk A = \rk A' $ \\
	Пусть $ U $ -- пространство строк $ A $, $ U' $ -- пространство строк $ A' $ \\
	Докажем, что $ \dim U = \dim U' $:
	\begin{itemize}
		\item Если строки ЛНЗ, то при элементарных преобразованиях получаются ЛНЗ
		\item Если строки ЛЗ, то они остаются ЛЗ
	\end{itemize}
	Значит, при элементарных преобразованиях строк размерность не меняется \\
	Пусть $ u_i \define (x_1^{(i)}, ..., x_n^{(n)}), \quad u_i' = (x_{\sigma(1)}^{(i)}, ..., x_{\sigma(n)}^{(i)}) $, где $ \sigma $ -- перестановка \\
	Рассмотрим ЛК $ \sum c_iu_i $ и $ \sum c_iu_i' $
	$$ c_1u_1 + c_2u_2 + ... = (..., c_1x_k^{(1)} + c_1x_k^{(2)}, ...) $$
	$$ c_1u_1' + c_2u_2' + ... = (..., c_1x_{\sigma(k)}^{(1)} + c_1x_{\sigma(k)}^{(2)}, ...) $$
	$ \sum c_iu_i $ и $ \sum c_iu_i' $ отличаются перестановкой координат, значит, ЛЗ и ЛНЗ наборы соотвестсвуют друг другу и размерность не меняется \\
	Достаточно доказать теорему для трапецевидной матрицы $ A' $ \\
	$ \rk A = r $ из определения трапецевидной матрицы (т. к. во всех б\'{о}льших минорах будет нулевая строка) \\
	Нужно доказать, что $ \dim U = r $ \\
	Для этого нужно найти $ r $ ЛНЗ строк \\
	Очевидно, что это будут первые $ r $ строк. Докажем, что они ЛНЗ: \\
	Пусть $ u_i $ -- $ i $-я строка
	$$ (0, 0, ..., 0) = c_1u_1 + c_2u_2 + ... + c_nu_n = (c_1a_{11}, c_1a_{12} + c_2a_{22}, ..., c_1a_{1r} + c_2a_{2r} + ... + c_ra_{rr}, ...) $$
	$$
	\begin{rcases}
		c_1a_{11} = 0 \\
		a_{11} \ne 0
	\end{rcases} \implies c_1 = 0, \qquad
	\begin{rcases}
		c_1a_{12} + c_2a_{22} = 0 \\
		c_2a_{22} = 0
	\end{rcases} \implies c_2 = 0, \quad \widedots[1em] \quad
	\begin{rcases}
		c_1a_{1r} + ... + c_ra_{rr} = 0 \\
		c_r = 0
	\end{rcases} \implies c_r = 0 $$
\end{proof}

\section{Теорема Кронекера-Капелли}

\begin{theorem}[Кронекера-Капелли]
	Система линейных уравнений совместна тогда и только тогда, когда ранг матрицы системы равен рангу расширенной матрицы системы
\end{theorem}

\begin{proof}
	Приведём матрицу системы к трапецевидной элементарными преобразованиями \\ строк и перестановками столбцов. Очевидно, что система перейдёт в равносильную
	$$
	\begin{pmatrix}
		a_{11} & a_{12} & ... & a_{1n} & | & b_1 \\
		0 & . & . & . & | & . \\
		. & a_{rr} & ... & a_{rn} & | & b_r \\
		0 & . & . & 0 & | & b_{r + 1} \\
		. & . & . & . & | & .
	\end{pmatrix} $$
	Система совместна $ \iff \forall i > r \quad b_i = 0 $ \\
	Пусть $ A $ -- матрица системы, $ A' $ -- расширенная матрица системы
	$$ \rk A = r $$
	\begin{itemize}
		\item Если $ \forall i > r \quad b_i = 0 $, то $ A' $ -- трапецевидная, и $ \rk A' = r $
		\item Если $ b_s \ne 0 $ при некотором $ s > r $: \\
		Возьмём минор:
		\begin{itemize}
			\item строки: 1, 2, ..., r, s
			\item столбцы: 1, 2, ..., r, n + 1
		\end{itemize}
		Его определитель не равен 0, значит $ \rk A' > r $
	\end{itemize}
\end{proof}

\section{Матричная запись линейного отображения. Изменение матрицы при замене базиса}

\begin{definition}
	Пусть $ U, V $ -- векторные пространства над $ K $. Отображение $ f : U \to V $ называется линейным, если:
	\begin{enumerate}
		\item $ \forall u_1, u_2 \in U \quad f(u_1 + u_2) = f(u_1) + f(u_2) $
		\item $ \forall u \in U, ~ k \in K \quad f(ku) = kf(u) $
	\end{enumerate}
\end{definition}

\begin{remark}
	Линейное отображение $ f : U \to U $ иногда называют линейным преобразованием
\end{remark}

\begin{definition}
	Пусть $ U, V $ -- конечномерные, $ \qquad e_1, ..., e_n $ -- базис $ U $, $ \qquad g_1, ..., g_m $ -- базис $ V $ \\
	$ f $ -- линейное отображение $ U \to V $ \\
    Матрицей $ f $ в данных базисах называется матрица, в $ i $-м столбце которой записаны координаты $ f(e_i) $ в базисе $ g_1, ..., g_m $, то есть
    $$
    \begin{pmatrix}
		a_{11} & ... & a_{1m} \\
        . & . & . \\
        a_{m1} & ... & a_{mn}
    \end{pmatrix} $$
    $$ f(e_i) = \sum_{k = 1}^{m} a_{ki}g_k $$
\end{definition}

\begin{lemma}[матричная запись линейного оторажения]
	$ U, V $ -- конечномерные, $ \qquad e_1, ..., e_n $ -- базис $ U $ \\
	$ g_1, ..., g_m $ -- базис $ V $, $ \qquad f : U \to V $ -- линейное
    \begin{enumerate}
    	\item Пусть $ A $ -- матрица $ f $ в данных базисах, $ \qquad u \in U, \quad v \in V $, такие, что $ f(u) = v $, \\
        $ X $ -- столбец координат $ u $ в базисе $ e_1, ..., e_n $ \\
        $ Y $ -- столбец координат $ v $ в базисе $ g_1, ..., g_m $ \\
        Тогда $ Y = AX $
        \begin{proof}
        	Пусть $ X =
            \begin{pmatrix}
            	x_1 \\
                . \\
                . \\
                . \\
                x_n
            \end{pmatrix}, \qquad Y =
            \begin{pmatrix}
            	y_1 \\
                . \\
                . \\
                . \\
                y_m
            \end{pmatrix} $
            \begin{multline*}
				v = f(u) = f(x_1e_1 + ... + x_ne_n) \underset{\text{линейность}}= x_1f(e_1) + ... + x_nf(e_n) \underset{\text{по опр. матрицы отображения}}= \\ = x_1(a_{11}g_1 + ... + a_{m1}g_m) + ... + x_i(a_{1i}g_1 + ... + a_{mi}g_m) + ... + x_n(a_{1n}g_1 + ... + a_{mn}g_m) \underset{\text{перегруппируем}}= \\ = (a_{11}x_1 + ... + a_{i1}x_i + ... + a_{1m}x_n)g_1 + ... + (a_{mn}x_1 + ... + a_{mi}x_i + ... + a_{mn}x_n)g_m
            \end{multline*}
			С другой стороны, $ v = y_1g_1 + ... + y_ng_n $
			$$ y_1 = a_{11}x_1 + ... + a_{1n}x_n + ... + a_{1n}x_m, \qquad y_m = a_{m1}x_1 + ... + a_{mn}x_n $$
            $$ Y = AX =
            \begin{pmatrix}
                a_{11} & ... & a_{1n} \\
                . & . & . \\
                a_{m1} & ... & a_{mn}
            \end{pmatrix} \cdot
            \begin{pmatrix}
            	x_1 \\
                . \\
                . \\
                . \\
                x_n
            \end{pmatrix} =
            \begin{pmatrix}
                a_{11}x_1 + ... + a_{1n}x_n \\
                \widedots[7em] \\
                a_{mi}x_1 + ... + a_{mn}x_n
            \end{pmatrix} $$
        \end{proof}
        \item Пусть $ A $ -- такая матрица, что $ \forall u, v : f(u) = v $, и их столбцов координат $ X, Y $ выполнено $ Y = AX $, то $ A $ -- матрица $ f $ в этих базисах
        \begin{proof}
        	Будем рассматривать вместо $ u $ базисные векторы:
			$$ u \define e_i, \qquad X =
			\begin{pmatrix}
				0 \\
				. \\
				1 \\
				. \\
				0
			\end{pmatrix} $$
			$$ Y = AX =
			\begin{pmatrix}
				... & a_{i1} & ... \\
				. & . & . \\
				... & a_{in} & ...
			\end{pmatrix} \cdot
			\begin{pmatrix}
				0 \\
				. \\
				1 \\
				. \\
				0
			\end{pmatrix} =
			\begin{pmatrix}
				a_{i1} \\
				. \\
				. \\
				. \\
				a_{in}
			\end{pmatrix} $$
			Получили, что$ Y $ -- $ i $-й столбец $ A $ \\
			С другой стороны, $ Y $ -- столбец координат $ f(u) = f(e_i) $ в базисе $ g_1, ..., g_m $
        \end{proof}
    \end{enumerate}
\end{lemma}

\begin{theorem}[Изменение матрицы при замене базисов]
	$ U, V $ -- конечномерные \\
	$ f : U \to V $ -- линейное, $ \qquad e_i, e_i' $ -- базисы $ U $, $ \qquad g_i, g_i' $ -- базисы $ V $ \\
	$ \qquad A $ -- матрица $f$ в базисах $ e_i, g_i, \qquad A' $ -- матрица $ f $ в базисах $ e_i', g_i' $ \\
    Тогда $ A' = C_{g_i \to g_i'}^{-1} \cdot A \cdot C_{e_i \to e_i'} $
\end{theorem}

\begin{proof}
	Пусть $ u \in U, \quad v \in V, \qquad f(u) = v $ \\
    $ X, X' $ -- столбцы коордиинат $ u $ в $ e_i, e_i' $ \\
    $ Y, Y' $ -- столбцы координат $ v $ в $ g_i, g_i' $
	\begin{equ}{171}
		X = C_{e_i \to e_i'} X'
	\end{equ}
	\begin{equ}{172}
		Y' = C_{g_i' \to g_i} Y = C_{g_i \to g_i'}^{-1} Y
	\end{equ}
	По первому утверждению леммы:
	\begin{equ}{173}
		Y = AX
	\end{equ}
	Проверим, что $ Y' = \bigg( C_{g_i \to g_i'}^{-1} \cdot A \cdot C_{e_i \to e_i'} \bigg)X' $. Из этого, по второму утверждению леммы, будет следовать утверждение теоремы
	$$ C_{g_i \to g_i'}^{-1} \cdot A \cdot C_{e_i \to e_i'} \cdot X' \underset{\eref{171}}= C_{g_i \to g_i'}^{-1} \cdot A \cdot X \underset{\eref{173}}= C_{g_i \to g_i'}^{-1} \cdot Y \underset{\eref{172}}= Y' $$
\end{proof}

\section{Свойства изоморфизма}

\begin{definition}
	$ f : U \to V $ называется изоморфизмом, если
    \begin{enumerate}
    	\item $f$ линейно
        \item $f$ -- биекция
    \end{enumerate}
    Если существует изоморфизм $ f : U \to V $, то пространства называются изоморфными
    \begin{notation}
    	$ U \cong V $ ($U \simeq V, U \sim V $)
    \end{notation}
\end{definition}

\begin{props}
	\item Если $ f $ -- изоморфизм, то $ \exist f^{-1} $ и $ f^{-1} $ -- изоморфизм
	\begin{proof}
		$ f^{-1} $ существует, так как $ f^{-1} $ -- биекция \\
		Докажем линейность: \\
		Возьмём $ u_1, u_2 \in U, \quad v_1 \define f(u_1), \quad v_2 \define f(u_2) $
		$$ v_1 + v_2 = f(u_1) + f(u_2) = f(u_1 + u_2) $$
		$$ f^{-1}(v_1 + v_2) = f^{-1}(f(u_1) + f(u_2)) = f^{-1}(f(u_1 + u_2)) = u_1 + u_2 = f^{-1}(v_1) + f^{-1}(v_2) $$
		Умножение на число -- аналогично
	\end{proof}
	\item Если $ f : U \to V $ и $ g : V \to W $ -- изоморфизмы, то $ g \circ f : U \to W $ -- изоморфизм
	\begin{proof}
		Композиция биекций -- биекция \\
		Композиция линейных -- линейное
	\end{proof}
\end{props}

\begin{properties}
	$ f : U \to V $ -- линейное отображение. Тогда:
	\begin{enumerate}
		\item \label{it:181} $ f $ -- инъекция $ \iff \ker f = \set{0} $
		\begin{proof}
			\hfill
			\begin{itemize}
				\item $ \implies $
				$$
				\begin{rcases}
					f(0) = 0 \\
					f \text{ -- инъекция}
				\end{rcases} \implies \forall u \ne 0 \quad f(u) \ne 0 $$
				\item $ \impliedby $
				Пусть $ f(u_1) = f(u_2) \implies f(u_1 - u_2) = f(u_1) - f(u_2) = 0 \implies u_1 - u_2 \in \ker f = \set{0} \implies u_1 - u_2 = 0 \implies u_1 = u_2 $
			\end{itemize}
		\end{proof}
		\item $ \ker f = \set{0} \implies f $ -- изоморфизм $ U \to \Img f $
		\begin{proof}
			$ f $ -- инъекция (по пункту \ref{it:181}) \\
			$ f $ -- сюръекция (по определению $ \Img f $)
			Значит, $ f $ -- биекция \\
			$ f $ линейно. Значит, $ f $ -- изоморфизм
		\end{proof}
	\end{enumerate}
\end{properties}

\begin{properties}
	Пусть $ f : U \to V $ -- изоморфизм. Тогда:
	\begin{enumerate}
		\item \label{it:182} $ e_1, ..., e_k $ -- ЛНЗ $ \iff f(e_1), ..., f(e_k) $ -- ЛНЗ
		\begin{proof}
			$ f^{-1} $ -- изоморфизм. Значит, дсотаточно доказать $ \impliedby $ \\
			То есть, что, если $ e_1, ..., e_k $ ЛЗ, то $ f(e_1), ..., f(e_k) $ ЛЗ \\
			Пусть $ a_1e_1 + ... + a_ke_k = 0 $ и не все $ a_i $ равны нулю \\
			Тогда $ a_1f(e_1) + ... + a_kf(e_k) = f(0) = 0 \implies f(e_i) $ -- ЛЗ
		\end{proof}
		\item \label{it:183} $ e_1, ..., e_k $ -- базис $ U \implies f(e_1), ..., f(e_k) $ -- базис $ V $
		\begin{proof}
			Базис -- максимальный ЛНЗ \\
			По пункту \ref{it:182}, $ f(e_1), ..., f(e_k) $ тоже ЛНЗ \\
			По пункту \ref{it:182}, любой б\'{о}льший набор будет ЛЗ \\
		\end{proof}
		\item $ \dim U = \dim V $
		\begin{proof}
			Очевидным образом следует из пункта \ref{it:183}
		\end{proof}
	\end{enumerate}
\end{properties}

\section{Лемма о выделении ядра прямым слагаемым. Размерности ядра и образа линейного отображения}

\begin{lemma}[выделение ядра прямым сложением]
	Пусть $ U, V $ -- конечномерны, $ \qquad f : U \to V $ линейно \\
    Тогда $ \exist W $ -- подпространство $ U $, такое что:
    \begin{enumerate}
        \item $ W \cong \Img f, \qquad f \clamp{W} \to \Img f $ -- изоморфизм
        \item $ \ker f \oplus W = U $
    \end{enumerate}
\end{lemma}

\begin{proof}
	Пусть $ g_1, ..., g_k \in V, \quad g_1, ..., g_k $ -- базис $ \Img f $
    $$ g_i \in \Img f \implies \exist e_i : f(e_i) = g_i, \quad e_i \in U $$
    Положим $ W = \langle e_1, ..., e_k \rangle $ \\
    Докажем, что $ W $ подходит:
    \begin{enumerate}
        \item Пусть $ f_1 : W \to \Img f, \quad f_1 = f \clamp{W} $. Докажем, что $ f_1 $ -- изоморфизм:
        \begin{itemize}
        	\item Проверим сюръективность: \\
            Пусть $ v \in \Img f \implies \exist a_i : v = a_1g_1 + ... + a_kg_k \implies v = a_1f(e_1) + ... + a_kf(e_k) = f_1(a_1e_1 + ... + a_ke_k) $
            \item Проверим инъективность: \\
            Достаточно проверить, что в 0 переходит только 0 \\
            Пусть $ w \in W, \quad f_1(w) = 0 $
            $$ w = a_1e_1 + ... + a_ke_k $$
            $$ f_1(w) = a_1f(e_1) + ... + a_kf(e_k) = a_1g_1 + ... + a_kg_k \underimp{g_i \text{ ЛНЗ}} \forall i \quad a_i = 0 \implies w = 0 \cdot e_1 + ... + 0 \cdot e_k = 0 $$
        \end{itemize}
        \item Проверим, что $ \ker f + W = U $: \\
        Пусть $ u \in U $ \\
        Пусть $ f(u) = v \in \Img f $ \\
        Пусть $ x \in W : f(x) = v $ (такой $x$ существует, так как $ f \clamp{W} $ -- изоморфизм) \\
        Положим $ y = u - x $ \\
        Тогда $ f(y) = f(u) - f(x) = v - v = 0 \implies y \in \ker f $
        $$
        \begin{rcases}
        	u = y + x \\
            y \in \ker f \\
            x \in W
        \end{rcases} \implies u \in \ker f + W $$
        \item Докажем, что $ U = \ker f \oplus W $: \\
        Достаточно доказать, что $ \underset{
            \begin{subarray}{c}
            	x \in \ker f \\
                y \in W
            \end{subarray}}{x + y = 0} \implies x = y = 0 $
        $$ x \in \ker f \implies f(y) = f(-x) = -f(x) = 0 $$
        $$
        \begin{rcases}
            f \clamp{W} \text{ -- инъекция} \\
            f(y) = 0
        \end{rcases} \implies y = 0 \implies x = 0 $$
    \end{enumerate}
\end{proof}

\begin{theorem}[размерность ядра и образа]
	Пусть $ U $ конечномерно, $ f : U \to V $ линейно \\
    Тогда $ \dim \ker f + \dim \Img f = \dim U $
\end{theorem}

\begin{proof}
    Положим $ W : W \cong \Img f, \quad U = \ker f \oplus W $ \\
    По свойству прямой суммы, $ \dim U = \dim \ker f + \dim W \implies \dim U = \dim \ker f + \dim \Img f $
\end{proof}

\section{Каноническая форма матрицы отображения. Образ линейного отображения и ранг матрицы}

\begin{theorem}[каноническая форма матрицы линейного отображения]
	Пусть $ U, V $ конечномерны, $ f : U \to V $ линейно \\
    Тогда существуют базисы $ u, v $, в которых матрица $ f $ имеет вид
    $$
    \begin{pmatrix}
        E & 0 \\
        0 & 0
    \end{pmatrix} =
    \begin{pmatrix}
    	1 & 0 & 0 & ... & 0 \\
        0 & 1 & 0 & ... & 0 \\
        . & . & . & . & 0 \\
        0 & 0 & 0 & 0 & 0
    \end{pmatrix} $$
\end{theorem}

\begin{proof}
    $ U = \ker f \oplus W, \qquad f \clamp{W} $ -- изоморфизм из $ W $ в $ \Img f $ \\
    Пусть $ e_1, ..., e_k $ -- базис $ W, \qquad e_{k + 1}, ..., e_n $ -- базис $ \ker f $ \\
    Тогда $ e_1, ..., e_n $ -- базис $ U $ (по свойству прямой суммы) \\
    $ f(e_1), ..., f(e_k) $ -- базис $ \Img f $ (по свойству изоморфизма) \\
    $ f(e_1), ..., f(e_k) $ ЛНЗ \\
    Положим $ g_1 = f(e_1), ..., g_k = f(e_k) $ \\
    Дополним $ g_1, ..., g_k $ до базиса $ V $ \\
    Пусть $ g_1, ..., g_m $ -- базис $ V $ \\
    Докажем, что базисы $ e_1, ..., e_n $ и $ g_1, ..., g_m $ подходят
    \begin{itemize}
    	\item Пусть $ i \le k $
        $$ f(e_1) = g_i = 0 \cdot g_1 + ... + 1 \cdot g_i + ... + 0 \cdot g_k + ... $$
        \item Пусть $ i > k $
        $$ e_i \in \ker f \implies f(e_i) = 0 = 0 \cdot g_1 + ... + 0 \cdot g_m $$
    \end{itemize}
\end{proof}

\begin{implication}
	Пусть $ A $ -- матрица $ n \times n $ с коэффициентами из поля $ K $ \\
    Тогда $ \exist C, D $ -- обратимые матрицы $ n \times n $, такие, что
    $$ C^{-1} A D =
    \begin{pmatrix}
    	E & 0 \\
        0 & 0
    \end{pmatrix} $$
\end{implication}

\begin{proof}
	Пусть $ U = K^n, \qquad e_1, ..., e_n $ -- базис $ U, \qquad f : A $ -- матрица $ f $ в $ e_1, ..., e_n $ \\
    Пусть $ e_1', ..., e_n', \quad e_1'', ..., e_n'' $ -- базисы $ U $, в которых $ f $ имеет матрицу
    $$
    \begin{pmatrix}
    	E & 0 \\
        0 & 0
    \end{pmatrix} \implies C^{-1} A D =
    \begin{pmatrix}
    	E & 0 \\
        0 & 0
    \end{pmatrix} $$
    где $ C, D $ -- матрицы перехода
\end{proof}

\begin{theorem}[линейное отображение и ранг матрицы]
	Пусть $ U, V $ конечномерны, $ f : U \to V $ линейно, $ A $ -- матрица $ f $ в некоторых базисах \\
    Тогда $ \dim \Img f = \rk A $
\end{theorem}

\begin{proof}
	Пусть $ e_1, ..., e_n $ -- базис $ U $, $ g_1, ..., g_m $ -- базис $ V $ \\
    Пусть $ w_i = f(e_i) $ \\
    Тогда $ \Img f = \langle w_1, ..., w_n \rangle $, т. к.
    $$
    \begin{rcases}
    	\forall v \in \Img f ~ \exist u \in U : f(u) = v \\
        \exist a_i : u = a_1e_1 + ... + a_ke_k
    \end{rcases} \implies v = a_1f(e_1) + ... + a_kf(e_k) = a_1w_1 + ... + a_kw_k $$
    Пусть $ X_j =
    \begin{pmatrix}
        a_{1j} \\
        . \\
        . \\
        . \\
        a_{mj}
    \end{pmatrix} $ -- $j$-й столбец матрицы $ f $ \\
    Тогда $ w_j = a_{1j}g_1 + ... + a_{mj}g_m $
    $$ \rk A = \dim \langle X_1, ..., X_n \rangle, \qquad \dim f = \dim \langle w_1, ..., w_n \rangle $$
    Из любой порождающей системы можно выбрать базис $ \implies \dim \langle w_1, ..., w_n \rangle $ равна максимальному количеству ЛНЗ векторов из $ w_1, ..., w_n $ \\
    Аналогично для $ X_1, ..., X_n $ \\
    Пусть $ v = c_1w_1 + ... + c_nw_n, \quad X $ -- столбец координат базиса \\
    Тогда $ X = c_1X_1 + ... + c_nX_n $
    \begin{multline*}
        v = c_1w_1 + ... + c_nw_n = c_1(a_{11}g_1 + ... + a_{i1}g_1 + ... + a_{m1}g_m) + ... + c_n(a_{1n}g_1 + ... + a_{in}g_i + ... + a_{mn}g_m) = \\ = (c_1a_{11} + ... + c_na_{1n}g_1 + ... + (c_1a_{i1} + ... + c_na_{1n})g_i + ...
    \end{multline*}
    $$ v = 0 \iff x = 0 $$
    $$ c_1w_1 = ... + c_nw_n = 0 \iff c_1x_1 + ... + c_nx_n = 0 $$
\end{proof}

\section{Действия над линейными отображениями, матрицы полученных линейных отображений}

\begin{definition}
	Пусть $ f, g : U \to V, \quad k $ -- скаляр \\
    Отображением $ f + g $ называется такое отображение, что $ (f + g)(u) = f(u) + g(u) $ \\
    Отображением $ kf $ называется такое отображение, что $ (kf)(u) = k \cdot f(u) $
\end{definition}

\begin{remark}
	$ f + g, ~ kf $ линейны
\end{remark}

\begin{definition}
	Произведением $ f : V \to W $ и $ g : U \to V $ называется $ fg = f \circ g : U \to V $ \\
    В частности, $ f^n = \underbrace{f \circ f \circ ... \circ f}_n : U \to U $
\end{definition}

\begin{remark}
	$ fg, ~ f^n $ линейны
\end{remark}

\begin{lemma}[действия над отображением и матрицей]
	\hfill
    \begin{enumerate}
    	\item Пусть $ U, V $ конечномерны, $ e_i, e_i' $ -- их базисы, $f, g : U \to V $ линейны, $ A, B $ -- матрицы $ f $ и $ g $, $ a, b $ -- скаляры \\
        Тогда $ aA + bB $ -- матрица $ af + bg $
        \begin{proof}
        	Пусть $ u \in U, \quad X $ -- столбец координат $ u $ в $ e_i, \quad Y_1, Y_2 $ -- столбцы координат $ f(u), g(u) $ в $ e_i \implies Y_1 = AX, ~ Y_2 = BX \implies aY_1 + bY_2 = aAX + bBX = (aA + bB)X $
            $$ (af + bg)(u) = af(u) + bg(u) \implies \text{ столбец координат } (af + bg)(u) = aY_1 + bY_2 = (aA + bB)X $$
        \end{proof}
        \item Пусть $ U, V, W $ конечномерны, $ e_i, e_i', e_i'' $ -- их базисы, $ f : V \to W, g : U \to V $ линейны, $ A, B $ -- матрицы $ f, g $ \\
        Тогда $ AB $ -- матрица $ fg $
        \begin{proof}
        	$ u \in U, w \in W : (fg)(u) = w $ \\
            $ X, Z $ -- столбцы координат \\
            Пусть $ v = g(u), \quad Y $ -- столбец координат $ V \implies Y = BX, ~ Z = AY \implies Z = A(BX) = (AB)X $
        \end{proof}
    \end{enumerate}
\end{lemma}

\section{Пространство линейных отображений}

\begin{theorem}[пространство линейных отображений]
	$ U, V $ -- векторные пространства над полем $ K $. Тогда:
    \begin{enumerate}
    	\item Множество линейных отображений образует векторное пространство над $ K $
        \item Если $ \dim U = m, ~ \dim V = n $, то пространство линейных отображений изоморфна пространству матриц размера $ m \times n $, его размерность равна $ mn $
    \end{enumerate}
\end{theorem}
