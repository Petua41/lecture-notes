\chapter{}

\section{Связность}

\begin{definition}
    $ X $ -- топологическое пространство. $ X $ называется несвязным, если
    $$ \underset{U_i \ne \O}{\exist U_1, U_2 \in \Omega_X} :
    \begin{cases}
    	U_1 \cap U_2 = \O \\
        U_1 \cup U_2 = X
    \end{cases} $$
    Иначе -- связное
\end{definition}

\begin{remark}
	Несвязность означает, что $ U_1, U_2 $ одновременно и открытые, и замкнутые \\
    Связность означает, что нет нетривиальных открыто-замкнутых подмножеств
\end{remark}

\begin{definition}
	$ (X, \Omega) $ -- топологическое пространство, $ A \sub X $. $ A $ называется связным, если оно связно в индуцированной топологии, т. е.
    $$ \forall U_1, U_2 \in \Omega_X \quad
    \begin{rcases}
    	U_1 \cup U_2 \supset A \\
        U_1 \cap U_2 \cap A = \O
    \end{rcases} \implies \left[
    \begin{aligned}
    	U_1 \cap A = \O \\
        U_2 \cap A = \O
    \end{aligned} \right. $$
\end{definition}

\begin{exmpls}
    \item Дискретное пространство более чем из одной точки \textbf{несвязно} (любое множество будет открытым и замкутым одновременно)
    \item Аннидискретное пространство \textbf{связно} (всего два открытых множества -- $ \O $ и $ X $)
    \item Стрелка \textbf{связна} (любые два непустых открытых множества пересекаются)
    \item Зариского:
    \begin{itemize}
        \item Если пространство бесконечно -- \textbf{связно}
        \item Если пространство конечно -- \textbf{несвязно} (конечное пространство Зариского -- то же самое, что дискретное пространство)
    \end{itemize}
    \item Стандартная топология:
    \begin{itemize}
        \item $ \R $ \textbf{связно}
        \item $ [a, b] $ \textbf{связно}
        \item $ \Q $ \textbf{несвязно}
        \begin{proof}
        	$ U_1 = (-\infty, \sqrt2) \cap \Q, \quad U_2 = (\sqrt2, +\infty) \cap Q $ -- граница не принадлежит ни $ U_1 $, ни $ U_2 $
        \end{proof}
        \begin{remark}
            Отсюда видно, что любое подмножество $ \Q $ \textbf{несвязно}
        \end{remark}
    \end{itemize}
\end{exmpls}

\begin{theorem}
	$ A $ связно, $ \quad A \sub B \sub \Cl A \implies B $ связно
\end{theorem}

\begin{proof}
	Пусть $ B $ несвязно, т. е.
    $$ \exist U_1, U_2 \in \Omega_X :
    \begin{cases}
    	U_1 \cup U_2 \supset B \\
        U_1 \cap U_2 \cap B = \O \\
        U_1 \cap B \ne \O \\
        U_2 \cap B \ne \O
    \end{cases} $$
    $$ A \sub B \implies
    \begin{cases}
        U_1 \cup U_2 \supset A \\
        U_1 \cap U_2 \cap A = \O
    \end{cases} \underimp{(A \text{ связно})} \text{ считаем } A \cap U_1 = \O \implies U_2 \supset A $$
    Возьмём $ F \define \Cl A \setminus U_1 $ \\
    $ F $ замкнуто, $ F \supset A $ -- \contra (замыкание не наименьшее)
\end{proof}

\begin{implication}
	$ A $ связно $ \implies \Cl A $ связно
\end{implication}

\begin{theorem}
	$ A, B $ связны, $ A \cap B \ne \O \implies A \cup B $ связно
\end{theorem}

\begin{proof}
	Пусть $ U_1, U_2 $ открытые, $ \quad U_1 \cup U_2 \supset A \cup B, \quad U_1 \cap U_2 \cap (A \cup B) = \O, $ \\
    $ U_1 \cap (A \cup B) \ne \O, \quad U_2 \cap (A \cup B) \ne \O $ \\
    Возьмём $ x_0 \in A \cup B $ \\
    НУО считаем, что $
    \begin{cases}
    	x_0 \in U_1 \\
        x_0 \notin U_2
    \end{cases} $
    $$
    \begin{rcases}
    	U_1 \cup U_2 \supset A \\
        U_1 \cap U_2 \cap A = \O
    \end{rcases} \underimp{A \text{ связно}} \left[
    \begin{aligned}
        U_1 \cap A = \O \\
        U_2 \cap A = \O
    \end{aligned} \right. $$
    $$
    \begin{rcases}
    	U_1 \cup U_2 \supset B \\
        U_1 \cap U_2 \cap B = \O
    \end{rcases} \underimp{B \text{ связно}} \left[
    \begin{aligned}
        U_1 \cap B = \O \\
        U_2 \cap B = \O
    \end{aligned} \right. $$
\end{proof}

\begin{implication}
	$ A_1, A_2, ..., A_n $ -- связные \\
    Построим граф с вершинами из $ \set{A_n} $ \\
    Если $ A_i \cap A_j \ne \O $, то проводим ребро \\
    Если граф связный, то $ \bigcup_{i = 1}^n A_i $ связно
\end{implication}

\begin{theorem}
	$ (0, 1) $ связен
\end{theorem}

\begin{proof}
	Пусть $ (0, 1) $ несвязен $ \implies \exist $ открытые $ U_1, U_2 :
    \begin{cases}
    	U_1 \cup U_2 \supset (0, 1) \\
        U_1 \cap U_2 \cap (0, 1) = \O
    \end{cases} $ \\
    Возьмём $ a \in U_1 \cap (0, 1), \quad b \in U_2 \cap (0, 1) $. Считаем $ a < b $ \\
    Положим $ x_* \define \sup\set{x \in U_1 | x < b} $
    $$  \left[
    \begin{aligned}
    	& x_* \in U_1 \implies \exist \veps > 0 : (x_* - \veps, x_* + \veps) \sub U_1 \implies b > x_* + \veps \quad \contra \quad \sup \\
        & x_* \in U_2 \implies \exist \veps > 0 : (x_* - \veps, x_* + \veps) \sub U_2 \implies x_* \text{ не точная верхняя граница (} x_* - \veps \text{ точнее)}
    \end{aligned} \right. $$
\end{proof}

\begin{implication}
	$ [0, 1] $ связен (как замыкание $ [0, 1] $)
\end{implication}

\begin{theorem}
	$ f : X \to Y $ -- непрерывная, $ A \sub X $ -- связно $ \implies f(A) $ связно в $ Y $
\end{theorem}

\begin{proof}
	Пусть $ f(A) $ несвязно, т. е.
    $$
    \begin{cases}
    	f(A) \sub U_1 \cup U_2 \\
        U_1 \cap U_2 \cap f(A) = \O \\
        U_1 \cap f(A) \ni y_1 \\
        U_2 \cap f(A) \ni y_2
    \end{cases} $$
    Положим $ V_1 \define f^{-1}(U_1), \quad V_2 \define f^{-1}(U_2) $
    $$ V_1 \cup V_2 = f^{-1}(U_1) \cup f^{-1}(U_2) = f^{-1}(U_1 \cup U_2) \supset f^{-1} \big( f(A) \big) \supset V_1 \cup V_2 \cup A = \O \supset A $$
\end{proof}

\begin{implication}
	$ X \simeq Y, \quad X $ связно $ \implies Y $ связно
\end{implication}

\begin{theorem}
	$ X \times Y $ связно $ \iff X, Y $ связны
\end{theorem}

\begin{proof}
	\hfill
    \begin{itemize}
    	\item $ \implies $ \\
        Возьмём отображения $
        \begin{cases}
        	p_X : X \times Y \to X \quad p_X(x, y) = x \\
            p_Y : X \times Y \to Y \quad p_Y(x, y) = y
        \end{cases} $
        \item $ \impliedby $ \\
        $ X, Y $ несвязные $ \implies
        \begin{cases}
            X \times Y = U_1 \cup U_2 \\
            U_1 \cap U_2 = \O
        \end{cases} $ \\
        Возьмём $ (x_1, y_1) \in U_1, \quad (x_2, y_2) \in U_2 $ \\
        НУО считаем $ (x_1, y_2) \in U_1 $ \\
        $ X \times \set{y_2} \simeq X $ -- связная, а $ U_1, U_2 $ разбивают $ X \times \set{y_2} $ -- \contra
    \end{itemize}
\end{proof}

\subsection{Компоненты связности}

\begin{definition}[компонента связности точки]
    $$ K_a \define \bigcup_{
        \begin{subarray}{c}
        	a \in A \\
            A \text{ связно}
        \end{subarray}}A \text{ -- связно} $$
\end{definition}

\begin{statement}
	$ K_a = K_b $ или $ K_a \cap K_b = \O $
\end{statement}

\begin{proof}
	Если $ K_a \ne K_b, K_a \cap K_b \ne \O \implies K_a \cup K_b $ связно
\end{proof}

\begin{statement}
	$ K_a $ замкнута
\end{statement}
