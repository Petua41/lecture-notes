\section{Ещё немного про натуральное уравнение}

\begin{quest}
	Верно ли что для всяких $ k $ и $ \text\ae $ существует кривая, задаваемая ими?
\end{quest}

\begin{answer}
	Только для положительных $ k $
\end{answer}

\begin{proof}
	Составляем дифур, доказываем, что решение существует
\end{proof}

\begin{quest}
	Можно ли нарезать болты как-то, кроме (классической) винтовой спирали?
\end{quest}

\begin{answer}
	Нельзя
\end{answer}

\begin{proof}
	Резьба болта и резьба гайки должны совмещаться в любой точке \\
	Это означает постоянство кривизны и кручения \\
	А мы доказали, что для заданных кривизны и кручения существует только одна кривая (с точностью до положения в пространстве)
\end{proof}

\chapter{Дифференциальная геометрия поверхностей}

\begin{undefthm}{Немного истории}
	Изучение поверхностей началось с Гаусса: \\
	Военные обратились к Гауссу с вопросом ``почему карта искажает расстояния и можно ли нарисовать карту без искажений'' \\
	Гаусс доказал, что нельзя
\end{undefthm}

\begin{note}
	Гаусс доказал, что сфера не изоморфна плоскости
\end{note}

\section{Поверхности}

\begin{definition}
	$ D \sub \R^2 $ -- область (открытое + связное) \\
	$ \vec{r} : D \to \R^3 $ -- вектор-функция (гладкая) \\
	$ \vec{r} $ называется параметризацией
\end{definition}

\begin{definition}
	Диффеоморфизм -- гладкий гомеоморфизм, обратное тоже гладкое
\end{definition}

\begin{note}
	Гдакость обозначает \textbf{нужное количество} непрерывных производных \\
	Нужное количество включает в себя ``нужное для равенства интересующих нас частных производных''
\end{note}

\begin{definition}
	$ \rho : \vawe{D} \to \R^3 $ -- другая вектор-функция
	$$ r(D) = \rho(\vawe{D}) \implies \exist \underbrace{\rho^{-1} \circ r}_{\text{диффеоморфизм}} : D \to \vawe{D} $$
	Тогда $ r $ и $ \rho $ -- разные параметризации одной поверхности
\end{definition}

\begin{definition}
	Поверхностью будем называть класс эквивалентности соответствующих функций
\end{definition}
Будем предполагать, что задана координатная сетка \\
Её внутренние координаты -- $ u $ и $ v $

\begin{definition}
	Поверхность назвыается регуляной, если $ \pder{r}u \nparallel \pder{r}v $
\end{definition}

\begin{remark}
	Это -- касательные векторы к координатным прямым \\
	Это означает, что координатная сетка нигде не имеет нулевого угла между координатными линиями
\end{remark}

Везде, где не оговорено особо, считаем что поверхность регулярна

\begin{eg}[нерегулярной поверхности]
	Есть какая-нибудь кривая \\
	К каждой её точке проводим касательую прямую \\
	Если кривая не плоская, то эти касательные будут образовывать некоторую поверхность \\
	Угол между этими касательными везде будет нулевой
\end{eg}

\begin{exmpls}[как поверхность задавать]
	\item В явном виде: $ z = f(x, y) $ \\
	Поверхность является графиком такой функции
	\item Неявно: $ F(x, y, z) = 0 $ \\
	По теореме о неявной функции, в определённых случаях это можно превратить в явную функцию в некоторых окрестностях
	\item Параметрически:
	$$
	\begin{cases}
		x = x(u, v) \\
		y = y(u, v) \\
		z = z(u, v)
	\end{cases} $$
\end{exmpls}

\section{Касательная плоскость}

\begin{undefthm}[как задаётся кривая на поверхности]
	Любая точка кривой лежит на поверхности \\
	Тогда мы можем применить к ней $ r^{-1} $ и ``спустить'' её на $ D $ \\
	Получим координаты $ u(t), v(t) $ \\
	Они называются внутренними координатами кривой на поверхности
\end{undefthm}

\begin{definition}
	Пусть $ u(t), v(t) $ -- внутренние координаты кривой на поверхности \\
	$ \vec{r} \bigg( u(t), v(t) \bigg) $ -- кривая на поверхности \\
	$ \dfrac{\di \vec{r}}{\di t}\clamp{t = t_0} $ -- касательный вектор
\end{definition}

\begin{definition}
	Касательная плоскость к поверхности -- множество касательных векторов в данной точке
\end{definition}

\begin{statement}
	Это плоскость с базисом $ \pder{r}u $ и $ \pder{r}v $
\end{statement}

\begin{notation}
	$ r_u \define \pder{r}u $ ($ = r_u' $)
\end{notation}

\begin{proof}
	Распишем и всё получится: \\
	Касательный вектор -- $ \dfrac{\di \vec{r}}{\di t}\clamp{t = t_0} $
	$$ \dfrac{\di \vec{r}}{\di t}\clamp{t = t_0} = \pder{r}{u} \cdot \frac{\di u}{\di t} + \pder{r}v \cdot \frac{\di v}{\di t} $$
	Это линейная комбинация $ \pder{t}u $ и $ \pder{r}v $ \\
	Верно ли, что $ \alpha r_u + \beta r_v $ является касательным вектором? \\
	Верно, для кривой
	$$
	\begin{cases}
		u = \alpha(t + t_0) \\
		v = \beta(t + t_1)
	\end{cases} \qquad
	\begin{cases}
		u_t = \alpha \\
		v_t = \beta
	\end{cases} $$
\end{proof}

$$ r_u \times r_v \ne 0 $$
$ n \define \frac{r_u \times r_v}{|r_u \times r_v|} $ -- вектор нормали \\
$ (r_u, r_v, n) $ -- правая тройка

\begin{exmpls}
	\item Явное задание:
	$$
	\begin{cases}
		x = u \\
		y = v \\
		z = z(u, v)
	\end{cases} $$
	Это частный случай параметрического задания
	\item Неявное задание:
	$$ \nabla f(x, y, z) \parallel \vec{n} $$
	Нормальная прямая будет иметь уравнение:
	$$ \frac{x - x_0}{f_x\clamp{x_0, y_0, z_0}} = \frac{y - y_0}{f_y\clamp{x_0, y_0, z_0}} = \frac{z - z_0}{f_z\clamp{x_0, y_0, z_0}} $$
	Касательная плоскость:
	$$ f_x\clamp{x_0, y_0, z_0}(x - x_0) + f_y\clamp{x_0, y_0, z_0}(y - y_0) + f_z\clamp{x_0, y_0, z_0}(z - z_0) = 0 $$
	\item Параметрическое задание:
	$$ r_u = (x_u, y_u, z_u), \qquad r_v = (x_v, y_v, z_v) $$
	$$ n \parallel
	\begin{vmatrix}
		i & j & k \\
		x_u & y_u & z_u \\
		x_v & y_v & z_v
	\end{vmatrix} $$
	Нормальная прямая:
	$$ \frac{x - x_0}{
		\begin{vmatrix}
			y_u & z_u \\
			y_v & z_v
		\end{vmatrix}} = \frac{y - y_0}{
		\begin{vmatrix}
			z_u & x_u \\
			z_v & x_v
		\end{vmatrix}} = \frac{z - z_0}{
		\begin{vmatrix}
			x_u & y_u \\
			x_v & y_v
		\end{vmatrix}} $$
	Касательная плоскость:
	$$
	\begin{vmatrix}
		y_u & z_u \\
		y_v & z_v
	\end{vmatrix}(x - x_0) +
	\begin{vmatrix}
		z_u & x_u \\
		z_v & x_v
	\end{vmatrix}(y - y_0) +
	\begin{vmatrix}
		x_u & y_u \\
		x_v & y_v
	\end{vmatrix}(z - z_0) = 0 $$
	\begin{remark}
		Мы переставили столбцы, чтобы избавиться от знаков
	\end{remark}
\end{exmpls}

\section{Длина кривой на поверхности}

\begin{definition}
	Коэффициенты \rom1 квадратичной формы поверхности:
	$$
	\begin{cases}
		E \define |r_u|^2 = (r_u, r_u) \\
		F \define (r_u, r_v) \\
		G \define (r_v, r_v) = |r_v|^2
	\end{cases} $$
\end{definition}

$$ \vec{r} = \bigg( x(u, v), y(u, v), z(u, v) \bigg) $$
$$ u = u(t), \qquad v = v(t) $$
\begin{multline*}
	l = \dint[t]ab{\bigg| \frac{\di r}{\di t} \bigg|} = \dint[t]ab{|r_uu_t + r_vv_t|} = \dint[t]ab{\sqrt{(r_uu_t + r_vv_t; r_uu_t + r_vv_t)}} = \\
	= \dint[t]ab{\sqrt{(r_u, r_u)u_t^2 + 2(r_u, r_v)u_tv_t + (r_v, r_v)v_t^2}} = \dint[t]ab{\sqrt{Eu_t^2 + 2Fu_tv_t + Gv_t^2}}
\end{multline*}
\rom1 форма:
$$ Eu_t^2 + 2Fu_tv_t + Gv_t^2, \qquad E, F, G \text{ -- функции от } u, v $$

\begin{remark}
	\hfill \\
	$
	\begin{rcases}
		f = 0 \\
		E, G > 0
	\end{rcases} \iff $ координатная сетка ортогональна \\
	$ E = 1 \iff $ координатные линии $ v = \const $, т. е. находятся в натуральной параметризации
\end{remark}

\begin{eg}[как параметризовать сферу]
	Координаты на сфере обычно называются широта и долгота \\
	Долгота меяется от какого-то нулевого меридиана \\
	Долгота -- угол $ \vphi $, широта -- угол $ \psi $ \\
	Сферическая система координат:
	$$
	\begin{cases}
		x = R\cos\vphi\cos\psi \\
		y = R\sin\vphi\cos\psi \\
		z = R\sin\psi
	\end{cases} $$
\end{eg}
