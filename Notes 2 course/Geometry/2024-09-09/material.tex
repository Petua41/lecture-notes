\section{Длина кривой}

Разделим кривую на кусочки:
$$ a = t_0 < t_1 < ... < t_n = b $$
Найдём длину ломаной:
$$ \sum_{i = 1}^n |\vect{f}(t_i) -\vect{f}(t_{i - 1})| \text{ -- интегральная сумма} $$
Длина кривой -- предел интегральных сумм

\begin{definition}
	$$ \int \define \liml{\max{t_i - t_{i - 1}} \to 0} \sum_{i = 1}^n |\vect{f}(t_i) - \vect{f}(t_{i - 1}) $$
	$$ \int \define \sup \sum_{i = 1}^n |\vect{f}(t_i) - \vect{f}(t_{i - 1}) $$
\end{definition}

\begin{remark}
	Определения равносильны (в предположении, что предел существует)
\end{remark}

\begin{iproof}
	\item $ \sup \ge \lim $ -- очевидно
	\item $ \lim \ge \sup $ \\
	$ \sup $ -- тоже какой-то частичный предел \\
	Пусть есть последовательность разбиений таких, что длина ломаной стремится к $ \sup $ \\
	Разобьём максимальную (по длине) $ t_i $ на $ n $ равных частей \\
	При этом, по неравенству треугольника, длина ломаной не уменьшилась, а $ \max|t_i - t_{i - 1}| $ теперь стремится к нулю
\end{iproof}

\begin{definition}
	Кривая называется спрямляемой, если $ \int < \infty $
\end{definition}

\begin{remark}
	Тогда $ \sup $ можно заменить на $ \limsup $ \\
	Для верхнего предела верно рассуждение о равносильности
\end{remark}

\begin{eg}[неспрямляемой кривой]
	$$ y = \sin \frac1x, \qquad x \in (0, 1) $$
\end{eg}

\begin{figure}[!ht]
	\begin{tikzpicture}[domain=0:1.2]
		\draw[->] (-0.2, 0) -- (4.2, 0) node[right] {$ x $};
		\draw[->] (0, -1.2) -- (0, 1.2) node[above] {$ f(x) $};

		\draw[color=blue, samples=1000] plot[id=x] function{sin(1/x)} node[right] {$ f(x) = \sin \frac1x $};
	\end{tikzpicture}
\end{figure}

\begin{theorem}
	$ f $ непр. дифференц. на $ [a, b] \implies f $ спрямляема и
	$$ \int = \dint[t]ab{|f'(t)|} $$
\end{theorem}

\begin{proof}
	\begin{equ}1
		\bigg| \dint[t]ab{|\vect{f}'(t)|} - \sum_{i = 1}^n |\vect{f}(t_i) - \vect{f}(t_{i - 1})| \bigg| \define A
	\end{equ}
	Хотим доказать, что это меньше $ 2\veps $
	\begin{remind}
		Неравенство треугольника:
		$$ |\vect{a}| + |\vect{b}| \ge |\vect{a} + \vect{b}|, \qquad |\vect{a}| - |\vect{b}| \le |\vect{a} + \vect{b}| $$
	\end{remind}
	Обозначим
	$$ \Delta_i t \define t_i - t_{i - 1}, \qquad \Delta_i f \define f(t_i) - f(t_{i - 1}), \qquad a_i \in [t_{i - 1}, t_i] $$
	\begin{multline*}
		\eref1 = \bigg| \dint[t]ab{|f'(t)|} - \sum_{i - 1}^n |f'(\tau_i)| \Delta_i t + \sum_{i = 1}^n |f'(\tau_i)| \Delta_i t - \sum_{i = 1}^n |f(t_i) - f(t_{i - 1})| \bigg| \trile \\
		\le \bigg| \underbrace{\dint[t]ab{|f'(t)} - \sum_{i = 1}^n |f'(\tau_i) \Delta_i t}_{\define \rom1} \bigg| + \bigg| \underbrace{\sum_{i = 1}^n \big( f'(\tau_i) \Delta_i t - |\Delta_i f| \big)}_{\define \rom2} \bigg|
	\end{multline*}
	\begin{itemize}
		\item $ \rom1 < \veps $ (фиксируем $ \veps $ и по нему подбираем мелкость разбиения)
		\item
		\begin{multline*}
			\rom2 = \sum_{i = 1}^n \bigg( \big| f'(\tau_i) \big| \Delta_i t - |\Delta_if| \bigg) \le \sum_{i = 1}^n \bigg| f'(\tau_i) \Delta_it - \Delta_if \bigg| = \\
			= \sum_{i = 1}^n \bigg| \dint[t]{t_{i - 1}}{t_i}{f'(\tau_i)} - \dint[t]{t_{i - 1}}{t_i}{f'(t)} \bigg| \le \sum_{i = 1}^n \dint[t]{t_{i - 1}}{t_i}{\bigg| f'(\tau_i) - f'(t) \bigg| } \underset{\eref2}< \sum_{i = 1}^n \dint[t]{t_{i - 1}}{t_i}\veps = \sum \veps \cdot \Delta_it = \\
			= \veps(b - a)
		\end{multline*}
		\begin{equ}2
			\bigg| f'(\tau_i) - f'(t) \bigg| < \veps \text{ при достаточно мелком разбиении}
		\end{equ}
		Воспользовались равномерной непрерывностью $ f'(t) $:
		$$ \bigg| \int f(t) \di t \bigg| \le \int |f(t)| \di t $$
	\end{itemize}
\end{proof}

\begin{undefthm}{Способы задания кривой}
	\begin{enumerate}
		\item Явно: \\
		$ y = f(x) $ (плоская кривая)
		$$ \int = \dint{x_0}{x_1}{\sqrt{1 + f'^2(x)}} $$
		\item Явно в полярных координатах:
		$$ r = r(\vphi) $$
		$$ x = r(\vphi)\cos \vphi, \qquad y = r(\vphi)\sin \vphi $$
		$$ x_\vphi' = r'\cos \vphi - r\sin \vphi, \qquad y_\vphi' = r'\sin \vphi + r\cos\vphi $$
		\begin{multline*}
			\sqrt{x'^2 + y'^2} = \sqrt{(r'\cos\vphi - r\sin\vphi)^2 + (r'\sin\vphi + r\cos\vphi)^2} = \\
			= \sqrt{r'^2\cos^2 - \cancel{2r'r\cos\vphi\sin\vphi} + r^2\sin^2\vphi + r'^2\sin^2\vphi + \cancel{2r'r\sin\vphi\cos\vphi} + r^2\cos^2\vphi} = \sqrt{r'^2 + r^2}
		\end{multline*}
		$$ \int = \dint[\vphi]{\vphi_0}{\vphi_1}{\sqrt{r'^2 + r^2}} $$
		\item Неявно:
		$$ x^2 + y^2 - 1 = 0 $$
		$$ F(x, y) = 0 $$
		Воспользуемся теоремой \ref{th:1}:
		$$ y = \pm \sqrt{1 - x^2}, \qquad \text{кроме } x = \pm 1, ~ y = 0 $$
		$$ \int = \dint{x_0}{x_1}{\sqrt{1 + f'^2(x)}} $$
		$$ f'(x) = y_x' = -\frac{F_x'}{F_y'} $$
		$$ \int = \dint{x_0}{x_1}{\sqrt{1 + \frac{F_x'^2}{F_y'^2}}} $$
	\end{enumerate}
\end{undefthm}

\begin{theorem}[о неявной функции]\label{th:1}
	Если $ \dfrac{\partial F}{\partial y} \ne 0 $ в окрестности какой-то точки, то
	$$ \text{в нек. } (x_0 - \veps, x_0 + \veps) \quad \exist f : F(x, f(x)) = 0 $$
	То есть $ F(x, y) = 0 \iff y = f(x) $
\end{theorem}

$$ f(t) = \bigg( x(t), y(t), z(t) \bigg) $$
$$ \int = \dint[t]{t_0}{t_1}{\sqrt{x'^2 + y'^2 + z'^2}} $$

\section{Натуральная параметризация}

\begin{definition}
	Параметризация $ \vect{f}(t) $ называется натуральной, если $ |f'(t)| = 1 $
\end{definition}

\begin{theorem}
	Натуральная параметризация существует и единственна (с точностью начального момента времени и направления обхода кривой)
\end{theorem}

\begin{iproof}
	\item Существование
	$$ s(t) \define \dint[\tau]{t_0}{\bm{t}}{|f'(\tau)} \implies s'(t) = |f'(t)| $$
	Пусть $ t(s) \define \vphi(s), \qquad \vphi = s^{-1} $
	$$ t'(s) = \vphi'(s) = \frac1{s'(t)} = \frac1{|f'(t)|} $$
	Пусть $ \vect{g}(s) \define \vect{f}(\vphi(s)) $
	$$ g'(s) = f'(\vphi(s)) \cdot \vphi'(s) = \frac{f'(\vphi(s))}{|f'(t)|} = \frac{f'(\vphi(s))}{|f'(\vphi(s))|} $$
	$$ |g'(s)| = \bigg| ... \bigg| = 1 \implies s \text{ -- натур. параметр }, \quad g(s) \text{ -- натур. параметризация} $$
	\item Единственность \\
	Пусть $ s $ и $ t $ -- натуральные параметры
	$$ g(s) = f \big( \vphi(s) \big), \qquad t = \vphi(s) $$
	$$ 1 = |g'(s)| = |f'(t) \cdot \phi'(s)| = |f'(t)| \cdot |\vphi'(s)| $$
	При этом, $ |f'(t)| = 1 $, т. к. $ t $ -- натур.
	$$ \implies |\vphi'(s)| = 1 \implies \vphi'(s) = \pm 1 \implies \vphi(s) = \pm s + \const $$
\end{iproof}

\section{Реп\'{е}р Френе}

Репер $ \equiv $ правая тройка ортонормированных векторов (зависит от $ t $) \\
Пусть $ s $ -- натуральный параметр
$$ \vect{v} \define f'(s), \qquad |\vect{v}(s)| \equiv 1 $$
\begin{lemma}
	Если ветор постоянной длины, то $ v'(s) \perp v(s) $
\end{lemma}
$$ \vect{n} \define \frac{v'}{|v'|} \text{ -- единичный} $$
$$ \vect{b} \define v \times n $$

\begin{definition}
	Репер Френе: ($ v $, $ n $, $ b $) (зависит от $ t $ или от $ s $)
	\begin{itemize}
		\item $ \vect{v} $ -- касательный вектор
		\item $ \vect{n} $ -- вектор главной нормали
		\item $ \vect{b} $ -- вектор бинормали
	\end{itemize}
\end{definition}
