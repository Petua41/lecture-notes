\chapter(R\textasciicircum{}n){$ \R^n $}

\section{Норма линейного оператора}

\begin{definition}
	$ A : \R^{m \ge 1} \to \R^{n \ge 1} $
	$$ \norm{A} \define \sup\limits_{
		\begin{subarray}{c}
			X \in \R^m \\
			\norm{X}_m \le 1
		\end{subarray}} \norm{AX}_n $$
\end{definition}

\begin{props}
	\item $ \norm{A} \ge 0 $
	$$ \norm{A} = 0 \iff A \equiv \On, \qquad \text{ т. е. } \On[m] X = \On \quad \forall X \in \R_m $$
	\begin{proof}
		Первое утверждение очевидно из определения и аналогичного свойства нормы вектора. Докажем второе:
		$$ AX = A_{n \times m}X $$
		$$ A_{n \times m} =
		\begin{bmatrix}
			a_{11} & . & a_{1m} \\
			. & . & . \\
			a_{m1} & . & a_{nm}
		\end{bmatrix} $$
		Возьмём $ 1 \le i \le n $ и $ 1 \le j \le m $ \\
		Обозначим
		$$ e_i \define [\overbrace{0, ..., \underset{i}1, ..., 0}^n], \qquad f_j \define \left.
		\begin{bmatrix}
			0 \\
			. \\
			1 \\
			. \\
			0
		\end{bmatrix} \right\} m $$
		$$ \norm{A} = 0 \implies \norm{AX}_n = 0 \implies AX = \On \quad \forall \underset{\norm{X} = 1}{X \in \R^m} $$
		Иначе супремум был бы положительным \\
		То есть, $ A_{n \times m} X = \On $. Значит,
		\begin{equ}{31}
			e_i(A_{n \times m}f_j) = e_i \column00 = 0
		\end{equ}
		При этом,
		$$ A_{n \times m} f_j =
		\begin{bmatrix}
			a_{11} & . & a_{1m} \\
			. & . & . \\
			a_{m1} & . & a_{nm}
		\end{bmatrix}
		\begin{bmatrix}
			0 \\
			. \\
			1 \\
			. \\
			0
		\end{bmatrix} = \column{a_{1j}}{a_{nj}} $$
		$$ e_i \column{a_{1j}}{a_{nj}} = a_{ij} \undereq{\eref{31}} 0 \implies A_{n \times m} = \On[n \times m] $$
	\end{proof}
	\item $ c \in \R $
	$$ \norm{cA} = |c| \cdot \norm{A} $$
	\begin{proof}
		\begin{multline*}
			\norm{cA} = \sup\limits_{
				\begin{subarray}{c}
					X \in \R^m \\
					\norm{X}_m \le 1
				\end{subarray}} \norm{(cA)X}_n \undereq{\text{линейность}} \sup \norm{c(AX)}_n = \sup |c| \cdot \norm{AX}_n = \\
			= |c| \sup \norm{AX}_n = |c| \cdot \norm{A}
		\end{multline*}
	\end{proof}
	\item $ A, B : \R^m \to \R^n $
	$$ \norm{A + B} \le \norm{A} + \norm{B} $$
	\begin{proof}
		\begin{multline*}
			\norm{A + B} = \sup \norm{(A + B)X}_n \undereq{\text{линейность}} \sup \norm{AX + BX} \le \sup(\norm{AX} + \norm{BX}) \le \\
			\le \sup \norm{AX} + \sup \norm{BX} \bdefeq{\norm{A}, \norm{B}} \norm{A} + \norm{B}
		\end{multline*}
	\end{proof}
	\item\label{prop:4} $ \norm{AX} \le \norm{A} \cdot \norm{X} \quad \forall X \in \R^m $
	\begin{iproof}
		\item Если $ X = \On[m] $, то это очевидно
		\item Пусть $ X \ne \On[m] $ \\
		Тогда $ t \define \norm{X}_m > 0 $ \\
		Рассмотрим $ Y \define \frac1t X $
		$$ \norm{Y}_m = \norm{\frac1t X} \underset{t > 0}= \frac1t \norm{X} \bdefeq{t} 1 $$
		$$ \norm{AY}_n \le \sup_{
			\begin{subarray}{c}
				U \in \R^m \\
				\norm{U} \le 1
			\end{subarray}} \norm{AU}_n = \norm{A} $$
		$$ X = tY \implies \norm{AX}_n = \norm{A(tY)}_n = t \cdot \norm{AY} \le t \norm{A} \underset{t > 0}\le \norm{A} $$
	\end{iproof}
	\item\label{prop:1.5} $ c_0 > 0 $
	\begin{equ}4
		\forall X \in \R^m \quad \norm{AX}_n \le c_0 \cdot \norm{X}_m
	\end{equ}
	$$ \implies \norm{A} \le c_0 $$
	\begin{proof}
		Возьмём $ \forall X \in \R^m $, такое, что $ \norm{X}_m \le 1 $
		\begin{equ}5
			\norm{AX}_n \underset{\eref4}\le c_0 \cdot \norm{X}_m \bydef[\le] c_0
		\end{equ}
		$$ \eref5 \bdef[\iff]\sup \sup\limits_{\norm{X} \le 1} \norm{AX}_n \le c_0 $$
	\end{proof}
	\item $ A =
	\begin{bmatrix}
		a_{11} & . & a_{1m} \\
		. & . & . \\
		a_{n1} & . & a_{nm}
	\end{bmatrix} $
	$$ \implies \norm{A} \le \bigg( \sum_{i = 1}^n \sum_{j = 1}^m a_{ij}^2 \bigg)^{\faktor12} $$
	\begin{proof}
		Пусть
		$$ X \define \column{x_1}{x_m}, \qquad \norm{X}_m \le 1 $$
		Тогда
		\begin{equ}7
			AX = \column{a_{11}x_1 + ... + a_{1m}x_m}{a_{n1}x_1 + ... + a_{nm}x_m}
		\end{equ}
		\begin{remind}
			Неравенство КБШ:
			$$ |(A, X)| \le |a_1| \cdot |x_1| + ... + |a_n| \cdot |x_n| \le \sqrt{a_1^2 + ... + a_n^2} \cdot \sqrt{x_1^2 + ... + x_n^2} = \norm{Y} \cdot \norm{X} $$
		\end{remind}
		$$ \norm{AX}_n^2 \undereq{\eref7} \sum_{i = 1}^n \bigg( \sum_{j = 1}^m a_{ij}x_j \bigg)^2 \underset{\text{КБШ}}\le \sum_{i = 1}^n \bigg( \sum_{j = 1}^m a_{ij}^2 \bigg) \bigg( \sum_{j = 1}^m \underbrace{x_j^2}_{\bydef[\le] 1} \bigg) \le \sum_{i = 1}^n \sum_{j = 1}^m a_{ij}^2 $$
	\end{proof}
	\item $ \R^{n \ge 1}, \R^{m \ge 1}, \R^{k \ge 1}, \qquad A : \R^m \to \R^n, \quad B : \R^m \to \R^k, \qquad BA : \R^m \to \R^k $
	$$ \implies \norm{BA} \le \norm{B} \cdot \norm{A} $$
	\begin{proof}
		Возьмём $ X \in \R^m $, такой, что $ \norm{X}_m \le 1 $ \\
		Пусть $ Y = AX \in \R^n $ \\
		Тогда $ BA(X) \bdefeq{BA} B(AX) \bdefeq{X} BY $
		\begin{equ}9
			\norm{BA(X)}_k = \norm{BY}_k \underset{\text{св-во \ref{prop:4}}}\le \norm{B} \cdot \norm{Y}_n
		\end{equ}
		\begin{equ}{10}
			\norm{Y}_n \bdefeq{Y} \norm{AX}_n \underset{\text{св-во \ref{prop:4}}}\le \norm{A} \cdot \norm{X}_m \underset{\norm{X} \le 1}\le \norm{A}
		\end{equ}
		$$ \eref9, \eref{10} \implies \norm{BA(X)}_k \le \underbrace{\norm{B} \cdot \norm{A}}_{\define c_0} $$
		Применяя свойство \ref{prop:1.5}, получаем нужное утверждение
	\end{proof}
\end{props}

\section{Частные производные второго и последующих порядков}

Рассматриваем вектор-столбцы, но, для удобства, иногда будем записывать их в строчку

\begin{definition}
	$ \Omega \sub \R^{n \ge 2} $ -- открытое, $ \Omega \ne \O, \qquad f : \Omega \to \R, \qquad 1 \le i \le n, \qquad \exist f_{x_i}'(X) \quad \forall X \in \Omega $ \\
	$ X_0 \in \Omega, \qquad 1 \le j \le n $ \\
	Получается новая функция $ f_{x_i}' : \Omega \to \R $ \\
	Пусть $ \exist (f_{x_i}')_{x_j}'(X_0) $ \\
	Говорят, что существует частная производная второго порядка
	$$ f_{x_i,x_j}''(X_0) \define (f_{x_i}')_{x_j}'(X_0) $$
\end{definition}

\begin{definition}
	Пусть $ \forall X \in \Omega \quad \exist f_{x_ix_j}''(X) $ \\
	Рассмотрим $ 1 \le k \le n $ и $ X_0 \in \Omega $ \\
	Пусть $ \exist (f_{x_ix_j}'')_{x_k}'(X_0) $ \\
	Будем говорить, что существует частная производная третьего порядка
	$$ f_{x_i,x_j,x_k}'''(X_0) = (f_{x_ix_j}'')_{x_k}'(X_0) $$
	$$ \widedots $$
	Пусть для $ l \ge 3 $ определено понятие $ f_{\underbrace{x_i, x_j, ..., x_s}_l}^{(l)}(X_0) $ \\
	Пусть $ \forall X \in \Omega \exist f_{x_i, ..., x_s}^{(l)}(X) $ \\
	Возьмём $ 1 \le t \le n $ \\
	Предположим, что $ \exist (f_{x_i, ..., x_s}^{(l)})_{x_t}'(X_0) $ \\
	Такую частную производную будем назвывать частной производной порядка $ l + 1 $
	$$ f_{x_i, ..., x_s, x_t}^{(l + 1)}(X_0) \define (f_{x_i, ..., x_s}^{(l)})_{x_t}'(X_0) $$
\end{definition}

\begin{notation}
	Исторически более распространено обозначение
	$$ \frac{\partial^l f(x)}{\partial x_s, ..., \partial x_i} $$
	В знаменателе $ x_j $ расположены в обратном порядке, т. к.
	$$ \frac{\partial^2f(x)}{\partial x_j \partial x_i} = \frac{\partial \bigg( \dfrac{\partial f}{\partial x_i} \bigg)}{\partial x_j} = \frac\partial{\partial x_j} \frac{\partial f}{\partial x_i} $$
\end{notation}

В следующей теореме (и следствии к ней) считаем, что $ \R^n $ -- пространство вектор-строк

\begin{theorem}[о смешанных производных]
	$ G = B_r(x_1^0, x_2^0), \qquad X_0 =
	\begin{bmatrix}
		x_1^0 \\
		x_2^0
	\end{bmatrix}, \qquad f : G \to \R, \qquad f \in \Cont{G} $ \\
	$ \forall X \in G \quad \exist f_{x_1}'(X), f_{x_2}'(X) \in \Cont{G}, \qquad \forall X \in G \quad \exist f_{x_1x_2}''(X), f_{x_2x_1}''(X) \text{ непрерывные в } X_0 $
	$$ \implies f_{x_1x_2}''(X_0) = f_{x_2x_1}''(X_0) $$
\end{theorem}

\begin{proof}
	Возьмём $ 0 < h < \frac{r}{\sqrt2} $ \\
	Тогда $ (x_1^0 + h, x_2^0 + h) \in G $ \\
	Рассмотрим функцию
	$$ g(h) \define \frac{f(x_1^0 + h, x_2^0 + h) - f(x_1^0 + h, x_2^0) - f(x_1^0, x_2^0 - h) + f(x_1^0, x_2^0)}{h^2} $$
	Определим
	$$ \vphi(x_2) \define \frac{f(x_1^0 + h, x_2) - f(x_1^0, x_2)}h, \qquad x_2 \in [x_2^0, x_2^0 + h] $$
	Рассмотрим выражение
	\begin{equ}{24}
		\frac{\vphi(x_2^0 + h) - \vphi(x_2^0)}h \bdefeq\vphi \frac{f(x_1^0 + h, x_2^0 + h) - f(x_1^0, x_2^0 + h) - f(x_1^0 + h, x_2^0) + f(x_1^0, x_2^0)}{h^2} = g(h)
	\end{equ}
	\begin{equ}{25}
		\forall x_2 \in [x_2^0, x_2^0 + h] \quad \exist \vphi'(x_2) \bdefeq\vphi \frac{f_{x_2}'(x_1^0 + h, x_2) - f_{x_2}'(x_1^0, x_2)}h
	\end{equ}
	Применим к $ \vphi $ теорему Лагранжа:
	\begin{multline}\lbl{27'}
		\exist 0 < h_2 < h : \vphi(x_2^0 + h) - \vphi(x_2^0) = \vphi'(x_2^0 + h_2) \cdot h \implies \\
		\implies \frac{\vphi(x_2^0 + h) - \vphi(x_2^0)}h = \vphi'(x_2^0 + h_2) \underset{\eref{25}}= \frac{f_{x_2}'(x_1^0 + h_2, x_2^0 + h_2) - f_{x_2}'(x_1^0, x_2^0 + h_2)}h
	\end{multline}
	Рассмотрим отдельно выражение $ f_{x_2}'(x_1^0 + h, x_2^0 + h_2) - f_{x_2}'(x_1^0, x_2^0 + h_2) $ \\
	Рассмотрим функцию $ l(x_1) \define f_{x_2}'(x_1, x_2^0 + h_2) $ \\
	По условию, наложенному на первые производные, она непрерывна при $ x \in [x_1^0, x_1^0 + h] $ \\
	По условию,
	\begin{equ}{27}
		\forall x_1 \in [x_1^0, x_1^0 + h] \quad \exist l'(x_1) = f_{x_2x_1}''(x_1, x_2^0 + h_2)
	\end{equ}
	Применим теорему Лагранжа к $ l $:
	\begin{multline}\lbl{29}
		\exist 0 < h_1 < h : l(x_1^0 + h) - l(x_1^0) = l'(x_1^0 + h_1) \cdot h \implies \\
		\implies \frac{l(x_1^0 + h) - l(x_1^0)}h \bydef \frac{f_{x_2}'(x_1^0 + h, x_2^0 + h_2) - f_{x_2}'(x_1^0, x_2^0 + h_2)}h = \\
		= l'(x_1^0 + h_1) \underset{\eref{27}}= f_{x_2x_1}''(x_1^0 + h_1, x_2^0 + h_2)
	\end{multline}
	\begin{multline}\lbl{210}
		g(h) \underset{\eref{24}}= \frac{\vphi(x_2^0 + h) - \vphi(x_2^0)}h \underset{\eref{27'}}= \frac{f_{x_2}'(x_1^0 + h, x_2^0 + h_2) - f_{x_2}'(x_1^0, x_2^0 + h_2)}h \underset{\eref{29}}= \\
		= f_{x_2x_1}''(x_1^0 + h_1, x_2^0 + h_2), \qquad 0 < h_1, h_2 < h
	\end{multline}
	Рассмотрим функцию
	$$ \psi(x_1) \define \frac{f(x_1, x_2^0 + h) - f(x_1, x_2^0)}h $$
	\begin{equ}{212}
		\frac{\psi(x_1^0 + h) - \psi(x_1^0)}h = \frac{f(x_1^0 + h, x_2^0 + h) - f(x_1^0 + h, x_2^0) - f(x_1^0, x_2^0 + h) + f(x_1^0, x_2^0)}{h^2} = g(h)
	\end{equ}
	\begin{equ}{213}
		\forall x_1 \in [x_1^0, x_1^0 + h] \quad \exist \psi'(x_1) = \frac{f_{x_1}'(x_1, x_2^0 + h) - f_{x_1}'(x_1, x_2^0)}h
	\end{equ}
	По теореме Лагранжа,
	\begin{multline}\lbl{214}
		\exist 0 < \overline{h_1}, \overline{h_2} < h : \frac{\psi(x_1^0 + h) - \psi(x_1^0)}h = \psi'(x_1 + \overline{h_1}) \underset{\eref{213}}= \frac{f_{x_1^0}'(x_1^0 + \overline{h_1}, x_2^0 + h) - f_{x_1}'(x_1^0 + \overline{h_1}, x_2^0)}h = \\
		= f_{x_1x_2}''(x_1^0 + \overline{h_1}, x_2^0 + \overline{h_2})
	\end{multline}
	\begin{equ}{215}
		g(h) \underset{\eref{212}}= \frac{\psi(x_1^0 + h) - \psi(x_1^0)}h \underset{\eref{214}}= f_{x_1x_2}''(x_1^0 + \overline{h_1}, x_2^0 + \overline{h_2})
	\end{equ}
	\begin{equ}{216}
		\eref{210}, \eref{215} \implies f_{x_2x_1}''(x_1^0 + h_1, x_2^0 + h_2) = f_{x_1x_2}''(x_1^0 + \overline{h_1}, x_2^0 + \overline{h_2})
	\end{equ}
	Устремим $ h $ к нулю справа и слева \\
	По условию теоремы,
	$$ f_{x_2x_1}''(x_1^0 + h_1, x_2^0 + h_2) \underarr{h \to +0} f_{x_2x_1}''(x_1^0, x_2^0) $$
	$$ f_{x_1x_2}''(x_1^0 + \overline{h_1}, x_2^0 + \overline{h_2}) \underarr{h \to -0} f_{x_1x_2}''(x_1^0, x_2^0) $$
	По соотношению \eref{216}, это одна и та же функция, а значит, она имеет единственный предел
\end{proof}

\begin{implication}[для $ n > 2 $]
	$ X_0 \in \R^{n \ge 3}, \qquad X_0 = (x_1^0, ..., x_i^0, ..., x_j^0, ..., x_n^0) $ \\
	$ f : B_r(X_0) \to \R, \qquad f \in \Cont{B_r(X_0)}, \qquad \forall X \in B_r(X_0) \quad \exist f_{x_i}'(X), f_{x_j}'(X) \in \Cont{B_r(X_0)} $ \\
	$ \forall X \in B_r(X_0) \quad \exist f_{x_ix_j}''(X), f_{x_jx_i}''(X) $ -- непр. в $ X_0 $
	$$ \implies f_{x_ix_j}''(X_0) = f_{x_jx_i}''(X_0) $$
\end{implication}

\begin{proof}
	$$ F(x_i, x_j) \define f(x_1^0, ..., x_i, ..., x_j, ..., x_n^0) $$
	$$ F_{x_ix_j}''(X_i, x_j) = f_{x_ix_j}''(x_1^0, ..., x_i, ..., x_j, ..., x_n^0) $$
\end{proof}

\begin{statement}
	$ \Omega \sub \R^{n \ge 2}, \qquad i \ne j, \qquad f \in \Cont{\Omega}, \qquad \forall X \in \Omega \quad f_{x_i}'(X), f_{x_j}'(X) \in \Cont{\Omega} $ \\
	$ \forall X \in \Omega \quad \exist f_{x_ix_j}''(X), f_{x_jx_i}''(X) \in \Cont{\Omega} $ \\
	По следствию,
	$$ \forall X \in \Omega \quad f_{x_ix_j}''(X) = f_{x_jx_i}''(X) $$
\end{statement}

\begin{remark}
	Есть примеры, которые показывают, что если не требовать непрерывности вторых производных в точке $ X_0 $, то они могут не совпадать
\end{remark}

\begin{statement}[для будущего определения]
	$ \Omega \sub \R^{n \ge 2}, \qquad i \ne j, \quad k $ \\
	Рассмотрим $ f_{x_ix_jx_k}'''(X), \quad f_{x_jx_ix_k}'''(X), \quad f_{x_ix_kx_j}'''(X) $ \\
	Пусть они все непрерывны на $ \Omega $ \\
	Все производные первого и второго порядков существуют и непрерывны на $ \Omega $ \\
	Тогда, по следствию
	$$ f_{x_ix_j}'' = f_{x_jx_i}'' \implies (f_{x_ix_j}'')_{x_k}' = (f_{x_jx_i}'')_{x_k}' $$
	$$ (f_{x_i}')_{x_kx_j}'' = (f_{x_i}')_{x_jx_k}'' $$
	Тем самым мы доказали, что у такой функции все частные производные третьего порядка совпадают
\end{statement}
