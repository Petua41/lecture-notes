\section{Формула Тейлора для функции \texorpdfstring{$ n $}n переменных с остатком в форме Пеано}

\begin{theorem}
	$ f : E \to \R^, \qquad E \sub \R^{n \ge 2}, \qquad X_0 \in E, \qquad X_0 \in \omega \sub E, \qquad f \in \Cont[r \ge 1]{\omega} $
	\begin{equ}1
		\implies f(X_0 + H) = f(X_0) + \sum_{k = 1}^r \sum_{|\alpha| = k} \frac1{\alpha!} \partial^\alpha f(X_0)H^\alpha + \rho(H)
	\end{equ}
	где
	\begin{equ}2
		\frac{\rho(H)}{\norm{H}^r} \underarr{H \to \On} 0
	\end{equ}
\end{theorem}

\begin{iproof}
	\item $ r = 1 $ \\
	По достаточному условию дифференцируемости, $ f $ дифф. в $ X_0 $, что, по определению, означает, что
	\begin{equ}3
		f(X_0 + H) = f(X_0) + \sum_{\nu = 1}^n f_{x_\nu}'(X_0) h_\nu + \rho(H)
	\end{equ}
	где
	\begin{equ}4
		\frac{\rho(H)}{\norm{H}} \underarr{H \to \On} 0, \qquad H = \column{h_1}{h_n}
	\end{equ}
	Перепишем эту сумму, используя мультииндексы. Возьмём $ \alpha : |\alpha| = 1 $, то есть $ \alpha = (0, ..., \underset\nu1, ..., 0) $
	$$ C_1^\alpha = \frac{0!...1!...0!}{1!} = 1 $$
	$$ \sum_{n = 1}^n f_{x_\nu}'(X_0)h_\nu = \sum_{|\alpha| = 1}C_1^\alpha \partial^\alpha f(X_0) H^\alpha $$
	Значит, ранее введённое определение дифференцируемости соотносится с обозначениями через мультииндексы
	\item $ r \ge 2 $ \\
	Применим к функции $ f $ формулу Тейлора с остатком в форме Лагранжа для $ r - 1 $:
	\begin{multline}\lbl6
		\exist c \in (0, 1) : f(X_0 + H) = \\
		= f(X_0) + \sum_{k = 1}^{r - 1} \sum_{|\alpha| = k} \frac1{\alpha!} \partial^\alpha f(X_0)H^\alpha + \sum_{|\alpha| = r} \frac1{\alpha!} \partial^\alpha f(X_0 + cH)H^\alpha \underset{\pm \sum \frac1{\alpha!} \partial^\alpha f(X_0)H^\alpha}= \\
		= f(X_0) + \sum_{k = 1}^{r - 1} \sum_{|\alpha| = k} \frac1{\alpha!} \partial^\alpha f(X_0) H^\alpha + \sum_{|\alpha| = r} \frac1{\alpha!} \partial^\alpha f(X_0) H^\alpha + \underbrace{\sum_{|\alpha| = r} \frac1{\alpha!} \bigg( \partial^\alpha f(X_0 + cH) - \partial^\alpha f(X_0) \bigg)H^\alpha}_{\rho(h)}
	\end{multline}
	Получили соотношение \eref1. Осталось доказать \eref2
	\begin{equ}7
		f \in \Cont[r]\omega \bydef[\implies] \partial^\alpha f(X_0 + H) - \partial^\alpha f(X_0) \underarr{H \to \On} 0 \qquad \forall \alpha : |\alpha| = r
	\end{equ}
	$$ H^\alpha \bydef h_1^{\alpha_1}...h_n^{\alpha_n} \implies |H^\alpha| \le \norm{H}^{\alpha_1}..\norm{H}^{\alpha_n} = \norm{H}^{|\alpha|} = \norm{H}^r $$
	\begin{equ}8
		\bigg| \frac{\rho(H)}{\norm{H}^r} \bigg| \underset{\eref6}= \bigg| \frac{\bigg( \partial^\alpha f(X_0 + H) - \partial^\alpha f(X_0) \bigg)H^\alpha}{\norm{H}^r} \bigg| \le \bigg| \partial^\alpha f(X_0 + cH) - \partial^\alpha f(X_0) \bigg| \underarr{H \to \On} 0
	\end{equ}
\end{iproof}

\section{Дифференциалы второго и последующих порядков}

Будем иметь дело с некоторым открытым $ \omega \sub \R^{n \ge 1} $

\begin{definition}
	$ f \in \Cont[1]\omega, \qquad X \in \omega, \qquad H \in \R^n, \qquad H = \column{h_1}{h_n} $
	$$ \di[1] f(X, H) \define \di f(X, H) \bydef \sum_{k = 1}^n f_{x_k}'(X)h_k $$
	$$ \widedots $$
	Пусть для некоторого $ r \ge 1 $ для функции $ f \in \Cont[r]\omega $ для любых $ X $ и $ \omega $ определена функция
	\begin{equ}9
		\di[r]f(X, H) = \sum_{|\alpha| = r} A_{r,\alpha} \partial^\alpha f(X)H^\alpha, \qquad A_{r,\alpha} \text{ -- некоторые \textbf{определённые} коэффициенты}
	\end{equ}
	$$ A_{1,\alpha} = 1 \quad \forall \alpha : |\alpha| = 1 $$
	Пусть $ f \in \Cont[r]\omega $. Определим дифференциал порядка $ r + 1 $:
	\begin{equ}{10}
		\di[r + 1]f(X, H) \define \sum_{|\alpha| = r} A_{r,\alpha} \di \bigg( \partial^\alpha f(X), H \bigg)H^\alpha = \sum_{|\alpha| = r + 1}A_{r + 1,\alpha} \partial^\alpha f(X)H^\alpha
	\end{equ}
\end{definition}

Дальше функции предполагаются достаточно гладкими, если не оговорено особо

\begin{eg}[переход от дифференциала пордяка 1 к дифференциалу порядка 2]
	Воспользуемся тем, что $ C_1^\alpha = 1 \quad \forall \alpha : |\alpha| = 1 $ \\
	Выпишем дифференциал перого порядка:
	$$ \di[1] f(X, H) = \sum_{k = 1}^n f_{x_k}'(X)h_k $$
	\begin{multline}\lbl{10'}
		\di[2] f(X, H) \bydef \sum_{k = 1}^n \di \bigg( f_{x_k}'(X), H \bigg)h_k \bydef \sum_{k = 1}^n \bigg( \sum_{l = 1}^n f_{x_kx_l}''(X)h_l \bigg)h_k = \sum_{k = 1}^n \sum_{l = 1}^n f_{x_kx_l}''(X)h_lh_k \underset{f_{x_kx_l}'' = f_{x_lx_k}''}= \\
		\sum_{k = 1}^n f_{x_kx_k}''(X)h_k^2 + 2\sum_{k < l} f_{x_kx_l}''(X)h_kh_l
	\end{multline}
	Возьмём $ \alpha : |\alpha| = 2 $:
	\begin{itemize}
		\item $ \alpha = (0, ..., \underset{k}2, ..., 0) $
		$$ \partial^\alpha f(X) = f_{x_kx_l}'' $$
		$$ C_2^\alpha \bydef \frac{2!}{0!...2!...0!} = 1 $$
		\item $ \alpha = (0, ..., \underset{k}1, ..., \underset{l}1, ..., 0) $
		$$ \partial^\alpha f = f_{x_kx_l}'' $$
		$$ C_2^\alpha \bydef \frac{2!}{0!...1!...1!...0!} = 2 $$
	\end{itemize}
	Теперь можно записать, что
	$$ \eref{10'} = \sum_{|\alpha| = 2} C_2^\alpha \partial^\alpha f(X) H^\alpha $$
	То есть, $ A_{2,\alpha} = C_2^\alpha $
\end{eg}

\begin{theorem}
	\begin{equ}{11}
		A_{r,\alpha} = C_r^\alpha
	\end{equ}
\end{theorem}

\begin{proof}
	Будем доказывать \textbf{индукцией} по $ r $:
	\begin{itemize}
		\item \textbf{База.} $ r = 1, 2 $ -- только что проверили
		\item \textbf{Переход.} Пусть это верно для $ r \ge 2 $. Докажем для $ r + 1 $: \\
		По \textbf{предположению индукции},
		\begin{equ}{12}
			\di[r + 1] f(X, H) \bydef \sum_{|\alpha = 1}C_r^\alpha \di \bigg( \partial^\alpha f(X), H \bigg) H^\alpha = \sum_{|\alpha| = r} C_r^\alpha \bigg( \sum_{\nu = 1}^n (\partial^\alpha f)_{x_\nu}'(X) h_\nu \bigg) H^\alpha
		\end{equ}
		В доказательстве формулы для производной порядка $ r $ (в предыдущей лекции, формулы для $ g^{(r + 1)}(Y + tH) $) было доказано, что
		$$ \eref{12} = \sum_{|\alpha| = r + 1} C_{r + 1}^\alpha \partial^\alpha f(X) H^\alpha $$
	\end{itemize}
\end{proof}

\begin{statements}
	Теперь можно переписать формулы Тейлора:
	\begin{enumerate}
		\item с остатком в форме Лагранжа:
		$$ f(X_0 + H) = f(X_0) + \sum_{k = 1}^r \frac1{k!} \di[k] f(X_0, H) + \frac1{(r + 1)!} \di[r + 1]f(X_0 + cH, H) $$
		\item с остатком в форме Пеано:
		$$ f(X_0 + H) = f(X_0) + \sum_{k = 1}^r \frac1{k!}\di[k] f(X_0, H) + \rho(H) $$
		$$ \frac{\rho(H)}{\norm{H}^r} \underarr{H \to \On} 0 $$
	\end{enumerate}
\end{statements}

Применим формулу Тейлора с остатком в форме Лагранжа при $ r = 2 $:
\begin{equ}{13}
	f(X_0 + H) = f(X_0) + \di f(X_0, H) + \half \di[2]f(X_0, H) + \rho(H)
\end{equ}
где
\begin{equ}{14}
	\frac{\rho(H)}{\norm{H}^2} \underarr{H \to \On} 0
\end{equ}
Перепишем \eref{13} через двойные суммы (как мы это делали при переходе к дифференциалу порядка 2):
\begin{equ}{15}
	f(X_0 + H) = f(X_0) + \sum_{k = 1}^n f_{x_k}'(X_0)h_k + \half \sum_{k = 1}^n \sum_{l = 1}^n f_{x_kx_l}''(X_0)h_kh_l + \rho(H)
\end{equ}
Рассмотрим квадратичную форму
$$ A(H) = \sum_{k = 1}^n \sum_{l = 1}^n a_{kl}h_kh_l, \qquad a_{kl} = \half f_{x_kx_l}''(X_0), \quad a_{kl} = a_{lk} $$

\begin{remind}[классификация квадратичных форм]
	Квадратичная форма называется:
	\begin{enumerate}
		\item положительно определённой, если $ \forall H \ne \On \quad A(H) > 0 $
		\item отрицательно определённой, если $ \forall H \ne \On \quad A(H) < 0 $
		\item неопределённой, если $ \exist H_1, H_2 \ne \On \quad A(H_1) > 0, \quad A(H_2) < 0 $
	\end{enumerate}
\end{remind}

\begin{theorem}
	Если квадратичная форма $ A $ положительно определена, то
	\begin{equ}{21}
		\exist m_1 > 0 \quad A(H) \ge m_1\norm{H}^2
	\end{equ}
	Если квадратичная форма отрицательно определена, то
	\begin{equ}{22}
		\exist m_2 > 0 \quad A(H) \le -m_2\norm{H}^2
	\end{equ}
\end{theorem}

\begin{proof}
	Достаточно доказать \eref{21}, т. к. для полож. определённой $ B $, форма $ -B(H) $ отрицательно определена \\
	Рассмотрим единичную сферу $ S \define \set{H \in \R^n | \norm{H} = 1} $ \\
	$ S $ -- компакт в пространстве $ \R^n $ (это доказывалось, когда рассматривались компакты в $ \R^n $)
	\begin{intuition}
		$ A(H) \in \Cont{\R^n} $
	\end{intuition}
	Значит, по второй теореме Вейерштрасса, $ A $ достигает своего минимального и максимального значений, т. е.
	\begin{equ}{23}
		\exist H_0 \in S : \forall H \in S \quad A(H) \ge A(H_0)
	\end{equ}
	Обозначим $ m_1 \define A(H_0) $ \\
	Т. к. квадратичная форма положительно определена, $ m_1 > 0 $ \\
	Рассмотрим $ \forall H \ne \On $ \\
	Пусть $ t \define \norm{H} > 0 $ (т. к. $ H \ne \On $) \\
	Рассмотрим $ H^* = \frac1t H $
	$$ \norm{H*} = \norm{\frac1tH} = \frac1t\norm{H} = \frac{t}t = 1 $$
	То есть, $ H^* \in S $ \\
	Тогда, в силу выбора $ H_0 $, получаем, что $ A(H^*) \ge m_1 $
	$$ A(H^*) \bydef A(\frac1tH) \underset{\text{по определению } A}= \frac1{t^2}A(H) \ge m_1 $$
	$$ A(H) \ge m_1t^2 \bydef m_1 \norm{H}^2 $$
\end{proof}

\section{Достаточное условие экстремума для функции \texorpdfstring{$ n $}n переменных}

\begin{theorem}
	$ \omega \sub \R^n $ -- открытое, $ \qquad X_0 \in \omega, \qquad f \in \Cont[2]\omega $ \\
	Выполнено необходимое условие локального экстремума, то есть $ f_{x_j}'(X_0) = 0, \quad 1 \le j \le n $
	\begin{remark}
		$ \implies \di f(X_0, H) = 0 \quad \forall H $
	\end{remark}
	Тогда
	\begin{enumerate}
		\item\label{th:4} если $ \di[2] f(X_0, H) $ положтельно определён, то $ X_0 $ -- строгий локальный минимум
		\begin{proof}
			Вспомним формулу Тейлора с остатком в форме Пеано:
			\begin{equ}{27}
				f(X_0 + H) = f(X_0) + \di f(X_0, H) + \half \di[2] f(X_0, H) + \rho(H)
			\end{equ}
			где
			\begin{equ}{28}
				\frac{\rho(H)}{\norm{H}^2} \underarr{H \to \On} 0
			\end{equ}
			Перепишем, применив замечание:
			\begin{equ}{27'}
				f(X_0 + H) = f(X_0) + \half \di[2] f(X_0, H) + \rho(H)
			\end{equ}
			Т. к. второй дифференциал положительно определён, то, по предыдущей теореме,
			\begin{equ}{29}
				\exist m_1 > 0 : \di[2] f(X_0, H) \ge m_1 \norm{H}^2
			\end{equ}
			Соотношение \eref{28} означает, что
			\begin{equ}{210}
				\exist \delta > 0 : \forall 0 < \norm{H} < \delta \quad \bigg| \frac{\rho(H)}{\norm{H}^2} \bigg| < \frac{m_1}4
			\end{equ}
			\begin{equ}{211}
				\eref{210} \iff |\rho(H)| < \frac{m_1}4 \norm{H}^2
			\end{equ}
			\begin{multline*}
				\eref{27'} \implies f(X_0 + H) \ge f(X_0) + \half \di[2] f(X_0, H) - |\rho(H)| \underset{\eref{29}, \eref{211}}> \\
				> f(X_0) + \half[m_1]\norm{H}^2 - \frac{m_1}4\norm{H}^2 = f(X_0) + \frac{m_1}4\norm{H}^2 > f(X_0)
			\end{multline*}
		\end{proof}
		\item\label{th:5} если $ \di[2] f(X_0, H) $ отрицательно определён, то $ X_0 $ -- строгий локальный максимум
		\begin{proof}
			Аналогично
		\end{proof}
		\item\label{th:6} если $ \di[2] f(X_0, H) $ неопределён, то нет локального экстремума
		\begin{proof}
			$ \di[2] f(X_0, H) $ неопределён означает, что
			$$ A(H_1) > 0, \qquad A(H_2) < 0 $$
			Рассмотрим $ H_1^* = \frac1{\norm{H_1}}H_1 $. Очевидно, что $ H_1^* \in S $
			$$ A(H_1^*) = \frac1{\norm{H}^2}A(H_1) \define p_1 > 0 $$
			Рассмотрим $ H_2^8 = \frac1{\norm{H_2}}H_2 $. Очевидно, что $ H_2^* \in S $
			$$ A(H_2) = \frac1{\norm{H_2}^2}A(H_2) \define p_2 > 0 $$
			Возьмём $ t > 0 $
			\begin{equ}{212}
				A(tH_2^*) = t^2A(H_2^*) = -p_2t^2
			\end{equ}
			\begin{equ}{213}
				A(tH_1^*) = t^2A(H_1^*) = p_1t^2
			\end{equ}
			Это было верно для любой квадратичной форме. Вернёмся к $ A(H) = \di[2] f(X_0, H) $ \\
			Выберем $ \delta_1 > 0 $, такое что
			\begin{equ}{214}
				\forall 0 < \norm{H} < \delta_1 \quad |\rho(H)| < \frac14 \min\set{p_1, p_2} \cdot \norm{H}^2
			\end{equ}
			Пусть $ 0 < t < \delta_1 $
			$$ \norm{tH_1^*} = \norm{tH_2^*} = t $$
			Рассмотрим
			\begin{multline*}
				f(X_0 + tH_1^*) = f(X_0) + \half \di[2] f(X_0, tH_1^*) + \rho(tH_1^*) \ge f(X_0) + \half t^2 \di[2] f(X_0, H_1^*) - |\rho(tH_1^*)| > \\
				> f(X_0) + \half p_1t^2 - \frac14 p_1t^2 = f(X_0) + \frac14 p_1t^2 > f(X_0)
			\end{multline*}
			При этом, $ X_0 + tH_1^* $ лежит в любой окрестности $ X_0 $ \\
			Рассмотрим
			$$ f(X_0 + H_2^*) \underset{\eref{212}}\le f(X_0) - \half[p_2]t^2 + |\rho(H)| < f(X_0) - \half[p_2]t^2 + \frac{p_2}4t^2 = f(X_0) - \frac{p_2}4t^2 < f(X_0) $$
			При этом, $ X_0 + tH_2^* $ тоже лежит в любой окрестности $ X_0 $ \\
			Значит, локального экстремума нет (по определениб локального экстремума)
		\end{proof}
	\end{enumerate}
\end{theorem}
