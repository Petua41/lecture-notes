Необыкновенные дифференциальные уравнения -- относительно частных производных \\
Первого порядка -- относительно первой производной

\begin{definition}
	Область -- непустое открытое связное подмножество
\end{definition}

Рассмотрим обыкновенное дифференциальное уравнение первого порядка, разрешённое относительно производной:
\begin{equ}{11}
	\frac{\di y(x)}{\di x} = f(x, y(x)), \qquad \text{или в краткой записи } y' = f(x, y)
\end{equ}
где $ x $ -- это независимая переменная, $ y = y(x) $ -- искомая функция, а $ f(x, y) $, если не оговорено противное, -- вещественная функция, определённая и непрерывная на множестве $ \vawe{G} = G \cup \hat{G} $ \\
$ G $ -- область в $ \R^2 $ \\
$ \hat{G} $ -- та часть (возможно, пустая) $ \partial G $ (границы $ G $), где функция $ f(x, y) $ определена и непрерывна \\
К ней же относим те точки, в которых функция $ f(x, y) $ может быть доопределена с сохранением непрерывности \\
Чаще всего будем рассматривать композиции элементарных функций

\begin{definition}
	Функцией называется пара объектов $ (X, f) $, в которой $ X $ -- это любое числовое множество, а $ f $ -- правило, по которому каждому числу из множества $ X $ сопоставляется единственное число
\end{definition}

Под непрерывностью будем понимать непрерывность по совокупности переменных

\begin{notation}
	Символ $ \langle $ подразумевает одну из скобок: $ ( $ или $ [ $, а символ $ \rangle $ -- скобку $ ) $ или $ ] $
\end{notation}

\begin{definition}
	Функция $ y = \vphi(x) $, заданная на некотором промежутке $ \braket{a, b} $, называется решением дифференциального уравнения \eref{11}, если для всякого $ x \in \braket{a, b} $ выполняются следующие три условия:
	\begin{enumerate}
		\item функция $ \vphi(x) $ дифференцируема в точке $ x $
		\item точка $ \big( x, \vphi(x) \big) \in \vawe{G} $
		\item $ \vphi'(x) = f \big( x, \vphi(x) \big) $
	\end{enumerate}
\end{definition}

\begin{remark}
	Фактически решение уравнения \eref{11} -- это пара: промежуток $ \braket{a, b} $ и определённая на нём функция $ \vphi(x) $
\end{remark}

\begin{remark}
	Первые два условия вспомогательные -- они позволяют подставить $ y = \vphi(x) $ в обе части \eref{11}
\end{remark}

\begin{remark}
	Любое решение $ y = \vphi(x) $ является функцией не просто дифференцируемой по условию 1, а непрерывно дифференцируемой или гладкой на $ \braket{a, b} $, т. е. $ \vphi(x) \in C^1(\braket{a, b}) $
\end{remark}

\begin{proof}
	Функция $ \vphi(x) $ дифференцируема, а значит, непрерывна в любой точке $ x \in \braket{a, b} $, поэтому $ f \big( x, \vphi(x) \big) $ непрерывна как композиция непрерывных функций, что влечёт непрерывность $ \vphi'(x) $ \\
	При этом, если решение задано на отрезке $ [a, b] $, то на его концах существуют и непрерывны односторонние производные
\end{proof}

\begin{definition}
	Решение $ y = \vphi(x) $ уравнения \eref{11}, заданное на промежутке $ \braket{a, b} $ будем называть:
	\begin{enumerate}
		\item внутренним, если $ \big( x, \vphi(x) \big) \in G $ для любого $ x \in \braket{a, b} $
		\item граничным, если $ \big( x, \vphi(x) \big) \in \hat{G} $ для любого $ x \in \braket{a, b} $
		\item смешанным, если найдутся такие $ x_1, x_2 \in \braket{a, b} $, что точка $ \big( x_1, \vphi(x_1) \big) \in G $, а точка $ \big( x_2, \vphi(x_2) \big) \in \hat{G} $
	\end{enumerate}
\end{definition}

Чтобы узнать, является ли конкретная функция решением, достаточно её подставить

\begin{lemma}[о записи решения в интегральном виде]
	Для того чтобы определённая на промежутке $ \braket{a, b} $ функция $ y = \vphi(y) $ была решением дифференциального уравнения \eref{11}, необходимо и достаточно, чтобы функция $ \vphi(x) $ была непрерывна на $ \braket{a, b} $, её график лежал в $ \vawe{G} $ и при некотором $ x_0 \in \braket{a, b} $ выполнялось тождество
	\begin{equ}{12}
		\vphi(x) \overset{\braket{a, b}}\equiv \vphi(x_0) + \dint[s]{x_0}x{f \big( s, \vphi(x) \big)}
	\end{equ}
\end{lemma}

\begin{iproof}
	\item Необходимость \\
	Пусть функция $ y = \vphi(x) $ на $ \braket{a, b} $ является решением уравнения \eref{11}, тогда по определению справедливо тождество $ f \big( x, \vphi(x) \big) \overset{\braket{a, b}}\equiv \vphi'(x) $ \\
	Интегрируя его при любом фиксированном $ x_0 \in \braket{a, b} $ по $ s $ от $ x_0 $ до $ x $ и перенося $ \vphi(x_0) $ в правую часть, получаем тождество \eref{12} \\
	В самом деле,
	$$ \dint[s]{x_0}x{f \big( s, \vphi(s) \big)} \overset{\braket{a, b}}\equiv \dint[s]{x_0}x{\vphi'(s)} = \vphi(x) - \vphi(x_0) $$
	\item Достаточность \\
	Пусть непрерывная на промежутке $ \braket{a, b} $ функция $ y = \vphi(x) $ удовлетворяет тождеству \eref{12}, тогда $ \vphi(x) $ нерпрывно дифференцируема на $ \braket{a, b} $, поскольку в правой части \eref{12} стоит интеграл с переменным верхним пределом от композиции непрерывных функций. \\
	Дифференцируя \eref{12}, заключаем, что выполняется и третье условие из определения решения уравнения \eref{11}
\end{iproof}

\begin{problem}[Коши]
	Для любой точки $ (x_0, y_0) \in \vawe{G} $ задача Коши с начальными данными $ x_0, y_0 $ заключается в том, чтобы найти все решения $ y = \vphi(x) $ уравнения \eref{11}, заданные на промежутках $ \braket{a, b} \ni x_0 $, в том числе внутренние, граничные или смешанные, такие, что $ \vphi(x_0) = y_0 $. \\
	При этом говорят, что задача Коши поставлена в точке $ (x_0, y_0) $, а найденные решения -- это решения поставленной задачи Коши
\end{problem}

\begin{notation}
	ЗК($ x_0, y_0 $)
\end{notation}
