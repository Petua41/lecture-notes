\begin{eg}[недиагонализуемый оператор]
	$ \mathcal{A} : \R^2 \to \R^2 $ \\
	Нужно, чтобы ар. кратность была 2, а геометрическая -- 1
	$$ \mathcal{A}
	\begin{pmatrix}
		x \\
		y
	\end{pmatrix} = A
	\begin{pmatrix}
		x \\
		y
	\end{pmatrix}, \qquad A =
	\begin{pmatrix}
		1 & 0 \\
		1 & 1
	\end{pmatrix} $$
	$$ \chi(t) =
	\begin{vmatrix}
		1 - t & 0 \\
		1 & 1 - t
	\end{vmatrix} = (1 - t)^2 $$
	$ \lambda = 1, \qquad $ ар. кратность -- 2 \\
	Найдём $ \dim V_1 $
	$$
	\begin{pmatrix}
		1 & 0 \\
		1 & 1
	\end{pmatrix}
	\begin{pmatrix}
		x \\
		y
	\end{pmatrix} = 1
	\begin{pmatrix}
		x \\
		y
	\end{pmatrix} $$
	$$
	\begin{cases}
		x = x \\
		x + y = y
	\end{cases} \quad \iff \quad x = 0 $$
	$ \dim V_1 = 1 \qquad $ геом. кратность -- 1
\end{eg}

\begin{remark}[Возведение в степень диагонализуемого оператора]
	$ A $ -- диагонализуемый \\
	$ e_1, ..., e_n $ -- базис из с. в. \\
	$ \lambda_1, ..., \lambda_n $ -- соответствующие с. ч.
	$$ \mathcal{A}^k(e_i) = \lambda_i^ke_i $$
	Пусть $ v \in V, \qquad v = a_1e_1 + ... + a_ne_n $
	$$ \implies \mathcal{A}^kv = a_1\lambda_1^ke_1 + ... + a_n\lambda_n^ke_n $$
	Пусть $ A $ -- матрица в стандартном базисе
	$$ C^{-1}AC =
	\begin{pmatrix}
		\lambda_1 & . & 0 \\
		. & . & . \\
		0 & . & \lambda_n
	\end{pmatrix}, \qquad C^{-1}A^kC =
	\begin{pmatrix}
		\lambda_1^k & . & 0 \\
		. & . & . \\
		0 & . & \lambda_n^k
	\end{pmatrix} $$
\end{remark}

\begin{definition}
	Блочной матрицей называется матрица вида
	$$ A =
	\begin{pmatrix}
		A_{11} & A_{12} & ... & A_{1n} \\
		. & . & . & . \\
		A_{m1} & A_{m2} & ... & A_{mn}
	\end{pmatrix} $$
	где $ \forall i \quad A_{ix} $ имеют поровну строк и $ \forall j \quad A_{xj} $ имеют поровну столбцов
\end{definition}

\begin{eg}
	$$ \left(
	\begin{tabular}{c c | c}
		. & . & . \\
		\hline
		. & . & . \\
		. & . & .
	\end{tabular} \right) $$
\end{eg}

\begin{definition}
	Блочно-диагональной матрицей называется матрица вида
	$$
	\begin{pmatrix}
		A_1 & 0 & . & 0 \\
		0 & A_2 & . & 0 \\
		. & . & . & . \\
		0 & . & 0 & A_n
	\end{pmatrix} $$
	где $ A_i $ -- квадратные
\end{definition}

\begin{remark}
	Блочно-диагональная матрица всегда квадратная
\end{remark}

\begin{eg}
	$$
	\begin{pmatrix}
		1 & 2 & 0 \\
		3 & 4 & 0 \\
		0 & 0 & 5
	\end{pmatrix} $$
\end{eg}

\begin{definition}
	$ U $ -- инвариантное подпространство оператора $ \mathcal{A} $ \\
	Через $ \mathcal{A}\clamp{U} $ обозначим сужение $ \mathcal{A} $ на $ U $, т. е.
	$$ \mathcal{A}\clamp{U} : U \to U, \qquad \mathcal{A}\clamp{U}(x) = \mathcal{A}(x) \quad \forall x \in U $$
\end{definition}

\begin{eg}
	$ \mathcal{A} : \R^3 \to \R^3 $
	$$ \mathcal{A}
	\begin{pmatrix}
		x \\
		y \\
		z
	\end{pmatrix} =
	\begin{pmatrix}
		x - y \\
		x + y \\
		2z
	\end{pmatrix} $$
	Выпишем его инвариантные пространства:
	$$ W = \left\langle
	\begin{pmatrix}
		0 \\
		0 \\
		1
	\end{pmatrix} \right\rangle, \qquad U = \left\langle
	\begin{pmatrix}
		1 \\
		0 \\
		0
	\end{pmatrix},
	\begin{pmatrix}
		0 \\
		1 \\
		0
	\end{pmatrix} \right\rangle $$
	$$ \underset{\in W}{
	\begin{pmatrix}
		0 \\
		0 \\
		z
	\end{pmatrix}} \mapsto \underset{\in W}{
	\begin{pmatrix}
		0 \\
		0 \\
		2z
	\end{pmatrix}}, \qquad \underset{\in U}{
	\begin{pmatrix}
		x \\
		y \\
		0
	\end{pmatrix}} \mapsto \underset{\in U}{
	\begin{pmatrix}
		x - y \\
		x + y \\
		0
	\end{pmatrix}} $$
	$$ \chi_{\mathcal{A}} = \chi_A =
	\begin{vmatrix}
		1 - t & 1 & 0 \\
		1 & 1 - t & 0 \\
		0 & 0 & 2 - t
	\end{vmatrix} = (2 - t) \bigg( (1 - t)^2 + 1 \bigg) = (2 - t)(t^2 - 2t + 2) $$
	$$
	\begin{pmatrix}
		1 \\
		0 \\
		0
	\end{pmatrix} \overarr{\mathcal{A}}
	\begin{pmatrix}
		1 - 0 \\
		1 + 0 \\
		2 \cdot 0
	\end{pmatrix} $$
	Рассмотрим $ U $:
	$$ e_1 =
	\begin{pmatrix}
		1 \\
		0 \\
		0
	\end{pmatrix}, \qquad e_2 =
	\begin{pmatrix}
		0 \\
		1 \\
		0
	\end{pmatrix} $$
	Запишем матрицу $ \mathcal{A} $ в базисе $ e_1, e_2 $:
	$$ \mathcal{A}(e_1) =
	\begin{pmatrix}
		1 \\
		1 \\
		0
	\end{pmatrix} = e_1 + e_2, \qquad \mathcal{A}(e_2) =
	\begin{pmatrix}
		1 \\
		1 \\
		0
	\end{pmatrix} = -e_1 + e_2 $$
	$$ A_U =
	\begin{pmatrix}
		1 & -1 \\
		1 & 1
	\end{pmatrix} $$
	$$ \chi_{\mathcal{A}\clamp{U}} = (1 - t)^2 + 1 = t^2 - 2t + 2 $$
	$$ \chi_{\mathcal{A}\clamp{W}} = 2t $$
\end{eg}

\begin{theorem}[блочные матрицы и инвариантыне подпространства]
	$ \mathcal{A} $ -- оператор на конечномерном пространстве $ V $
	\begin{enumerate}
		\item $ U $ -- инвариантное пространство $ \mathcal{A}, \qquad e_1, ..., e_s $ -- базис $ U, \qquad e_1, .., e_s, ..., e_n $ -- базис $ V $ \\
		$ A_U, A $ -- матрицы $ \mathcal{A} $ на $ U, V $ в этих базисах
		$$ \implies A =
		\begin{pmatrix}
			\bm{A_U} & B \\
			\bm0 & C
		\end{pmatrix} \quad \text{для некоторых } B, C $$
		\begin{proof}
			Пусть
			$$ A_U =
			\begin{pmatrix}
				a_{11} & ... & a_{1s} \\
				. & . & . \\
				a_{s1} & ... & a_{ss}
			\end{pmatrix} $$
			Возьмём $ 1 \le i \le s $ \\
			Посмотрим, как $ \mathcal{A} $ действует на $ e_i $:
			$$ \mathcal{A}(e_i) = a_{1i}e_1 + ... + a_{si}e_s = a_{1i}e_1 + ... + a_{si}e_s + ... + 0 \cdot e_{s + 1} + ... + 0 \cdot e_n $$
			Получили разложение $ \mathcal{A}(e_i) $ по базису $ V $, то есть, столбец матрицы оператора в базисе $ e_1, ..., e_s, ..., e_n $:
			$$
			\begin{pmatrix}
				a_{1i} \\
				. \\
				. \\
				e_{si} \\
				0 \\
				. \\
				. \\
				0
			\end{pmatrix} \text{ -- } i \text{-й столбец } A $$
			$$ \implies A =
			\begin{pmatrix}
				a_{11} & a_{21} & ... & a_{s1} & * & * \\
				. & . & . & . & * & * \\
				a_{s1} & a_{s2} & ... & a_{ss} & . & . \\
				0 & 0 & ... & 0 & . & . \\
				. & . & . & . & . & . \\
				0 & 0 & ... & 0 & * & *
			\end{pmatrix} $$
		\end{proof}
		\item $ V = U_1 \oplus ... \oplus U_k $, где $ U_i $ -- инвар. для $ \mathcal{A} $ \\
		$ A_1, ..., A_k $ -- матрицы $ \mathcal{A} $ на $ U_1, ..., U_k $ в некоторых базисах \\
		$ A $ -- матрица $ \mathcal{A} $ на $ V $ в базисе, являющемся объединением базисов $ U_i $ (в естественном порядке: базис $ U_1 $, базис $ U_2 $, ...)
		$$ \implies A =
		\begin{pmatrix}
			A_1 & 0 & . & 0 \\
			0 & A_2 & . & 0 \\
			. & . & . & . \\
			0 & . & 0 & A_k
		\end{pmatrix} $$
		Так как $ A_1, ..., A_K $ -- квадратные, то $ A $ -- блочно-диагональная
		\begin{proof}
			Пусть $ \dim U_1 = d_1, \quad \dim U_2 = d_2, ... $ \\
			Рассмотрим столбец матрицы $ A $ с номером $ d_1 + d_2 + d_{i - 1} + t $, где $ 1 \le t \le d_i $ \\
			Обозначим элементы базисов:
			$$ U_1 : e_1^{(1)}, ..., e_{d_1}^{(1)} $$
			$$ U_2 : e_2^{(2)}, ..., e_{d_2}^{(2)} $$
			$$ \widedots[6em] $$
			В этом столбце записаны координаты вектора $ e_t^{(i)} $ в базисе $ V $ \\
			Разложим его по базису подпространства $ U_i $:
			$$ e_t^{(i)} = a_1e_1^{(i)} + ... + a_{d_i}e_{d_i}^{(i)} $$
			Дополним нулями:
			$$ \underbrace{0 \cdot e_1^{(1)} + ... + 0 \cdot d_1^{(1)}}_{d_1} + \underbrace{0 \cdot e_1^{(2)} + ...}_{d_2} + \widedots[5em] + \underbrace{a_1e_1^{(i)} + ... + a_{d_i}e_d^{(i)}}_{d_i} + 0 \cdot e_1^{(i + 1)} \widedots[5em] $$
			Получили разложение $ e_r^{(i)} $ по базису $ V $ \\
			$ (d_1 + d_2 + ... + d_{i - 1} + t) $-й столбец равен
			$$
			\begin{pmatrix}
				0 \\
				. \\
				. \\
				0 \\
				a_1 \\
				. \\
				. \\
				a_{d_i} \\
				0 \\
				. \\
				. \\
				0
			\end{pmatrix} $$
		\end{proof}
	\end{enumerate}
\end{theorem}

\begin{implication}[делители характеристического многочлена]
	$ \mathcal{A} $ -- оператор на конечномерном пространстве $ V, \qquad \chi(t) $ -- его характ. многочлен
	\begin{enumerate}
		\item $ U $ -- инвариантное подпространство, $ \chi_U(t) $ -- характ. многочлен $ \mathcal{A}\clamp{U} $
		$$ \implies \chi(t) \divby \chi_U(t) $$
		\item $ V = U_1 \oplus ... \oplus U_k $, где $ U_i $ -- инвариатные \\
		$ \chi_i(t) $ -- характ. многочлен $ \mathcal{A}\clamp{U_i} $
		$$ \chi(t) = \chi_1(t) \cdot ... \cdot \chi_k(t) $$
	\end{enumerate}
\end{implication}

\begin{proof}
	Рассмотрим базисы как в теореме
	\begin{enumerate}
		\item
		\begin{multline*}
			\chi_{\mathcal{A}}(t) = \chi_A(t) = \left|
			\begin{pmatrix}
				A_U & B \\
				0 & C
			\end{pmatrix} - t E_n \right| =
			\begin{vmatrix}
				A_U - t E_s & B \\
				0 & C - t E_{n - s}
			\end{vmatrix} = |A_U - t E_s | \cdot | C - t E_{n -s} | = \\
			= \chi_{A_U}(t) \cdot \chi_C(t) = \chi_U(t) \cdot \chi_C(t)
		\end{multline*}
		\item
		$$ \chi_A(t) = |A - tE| =
		\begin{vmatrix}
			A_1 - tE & 0 & . & 0 \\
			0 & A_2 - tE & . & 0 \\
			. & . & . & . \\
			0 & . & 0 & A_k - tE
		\end{vmatrix} = |A_1 - tE| \cdot |A_2 - tE| \cdot \widedots[4em] = \chi_1(t) \cdot \chi_2(t) \cdot ... $$
	\end{enumerate}
\end{proof}

\begin{lemma}[ранг блочно-диагональной матрицы]
	$ A $ -- блочно-диагональная
	$$ A =
	\begin{pmatrix}
		A_1 & 0 & . & 0 \\
		. & . & . & . \\
		0 & . & 0 & A_k
	\end{pmatrix} $$
	$$ \implies \rk A = \rk A_1 + ... + \rk A_k $$
\end{lemma}

\begin{proof}
	Воспользуемся тем, что ранг -- это количество ЛНЗ строк \\
	Пусть для каждой матрицы $ A_i $ выбран набор строк $ s_1^{(i)}, s_2^{(i)}, ..., s_n^{(i)} $ \\
	Для строки $ s_j^{(i)} $ обозначим через $ \vawe{s_j}^{(i)} $ соответствующую строку матрицы $ A $ \\
	Достаточно доказать, что
	$$ \text{набор } \vawe{s_1}^{(1)}, ..., \vawe{s_{r_1}}^{(1)}, \vawe{s_1}^{(2)}, ..., \vawe{s_{r_2}}^{(2)}, \widedots[5em] \text{ ЛНЗ } \iff \text{ все наборы }
	\begin{cases}
		s_1^{(1)}, ..., s_{r_1}^{(1)} \\
		s_1^{(2)}, ..., s_{r_2}^{(2)} \\
		\widedots[4em]
	\end{cases} \text{ ЛНЗ} $$
	\begin{itemize}
		\item $ \implies $ \\
		Докажем \textbf{от противного}: \\
		Предположим, что $ s_1^{(i)}, ..., s_{r_i}^{(i)} $ ЛЗ \\
		То есть, $ \exist a_1, ..., a_{r_i} $, не все равные нулю, такие, что $ a_1s_1^{(i)} + ... + a_{r_i}s_{r_i}^{(i)} = 0 $ \\
		Дополним нулями:
		$$ a_1\vawe{s_1}^{(i)} + ... + a_{r_i}\vawe{s_{r_i}}^{(i)} = 0 $$
		То есть, $ \vawe{s_1}^{(i)}, ..., \vawe{s_{r_i}}^{(i)} $ ЛЗ \\
		А значит, и всеь набор ЛЗ -- \contra
		\item $ \impliedby $ \\
		Докажем \textbf{от противного}: \\
		Пусть все наборы $ s_1^{(i)}, ..., s_{r_i}^{(i)} $ ЛНЗ, а $ \vawe{s_1}^{(1)}, ..., s_{r_i}^{(1)}, \widedots[4em], \vawe{s_{r_k}}^{(k)} $ ЛЗ, то есть
		$$ \sum_{i, j} a_j^{(i)}\vawe{s_j}^{(i)} = 0, \qquad \text{не все } a_j^{(i)} \text{ равны нулю} $$
		Положим
		$$ T_i \define a_1^{(i)}s_1^{(i)} + ... + a_{r_i}^{(i)}s_{r_i}^{(i)} $$
		$$ \vawe{T_i} \define a_1^{(i)}\vawe{s_1}^{(i)} + ... + a_{r_i}^{(i)}\vawe{s_{r_i}}^{(i)} $$
		$$ \vawe{T_1} + \vawe{T_2} + ... + \vawe{T_k} = 0 \implies \vawe{T_1} = \vawe{T_2} = ... = 0 $$
		Строки $ \vawe{T_1}, ..., \vawe{T_k} $ не содержат ненулевые элементы в одном столбце
		$$ \implies T_1 = 0, \quad T_2 = 0, \quad T_k = 0 $$
		\begin{remark}
			Эти нули разной длины
		\end{remark}
		$$ \implies \forall i \quad a_1^{(i)}s_1^{(i)} + ... + a_{r_i}s_{r_i}^{(i)} = 0 $$
		$$ s_1^{(i)}, ..., s_{r_i}^{(i)} \text{ ЛНЗ } \implies a_1^{(i)} = ... = a_{r_i}^{(i)} = 0 $$
	\end{itemize}
\end{proof}

\begin{remark}
	На самом деле, блочно-диагональная матрица -- избыточное условие, однако нам понадобится именно такой случай
\end{remark}

\begin{implication}
	$ V = U_1 \oplus ... \oplus U_k, \qquad U-i $ -- инвариантно для $ \mathcal{A} $
	$$ \implies \dim \Img \mathcal{A} = \dim \Img \mathcal{A}\clamp{U_1} + ... + \dim \Img \mathcal{A}\clamp{U_k} $$
\end{implication}

\section{Существование жордановой формы нильпотентного оператора}

\begin{definition}
	Оператор называется нильпотентным, если $ \mathcal{A}^k = 0 $ для некоторого $ k $ \\
	Показатель нильпотентности -- это наименьшее $ k $, для которого $ \mathcal{A}^k = 0 $
\end{definition}

\begin{egs}
	$ \mathcal{A} : \R^3 \to \R^3, \qquad \mathcal{A} : X \mapsto AX $
	\begin{itemize}
		\item $ A =
		\begin{pmatrix}
			1 & 2 & -3 \\
			1 & 2 & -3 \\
			1 & 2 & -3
		\end{pmatrix}, \qquad A^2 =
		\begin{pmatrix}
			0 & 0 & 0 \\
			0 & 0 & 0 \\
			0 & 0 & 0
		\end{pmatrix} $ \\
		Нильпотентный, с показателем 2
		\item $ A =
		\begin{pmatrix}
			0 & 0 & 0 \\
			1 & 0 & 0 \\
			0 & 1 & 0
		\end{pmatrix}, \qquad A^2 =
		\begin{pmatrix}
			0 & 0 & 0 \\
			1 & 0 & 0 \\
			0 & 1 & 0
		\end{pmatrix}
		\begin{pmatrix}
			0 & 0 & 0 \\
			1 & 0 & 0 \\
			0 & 1 & 0
		\end{pmatrix} =
		\begin{pmatrix}
			0 & 0 & 0 \\
			0 & 0 & 0 \\
			1 & 0 & 0
		\end{pmatrix} $
		$$ A^3 = A^2 A =
		\begin{pmatrix}
			0 & 0 & 0 \\
			0 & 0 & 0 \\
			1 & 0 & 0
		\end{pmatrix}
		\begin{pmatrix}
			0 & 0 & 0 \\
			1 & 0 & 0 \\
			0 & 1 & 0
		\end{pmatrix} =
		\begin{pmatrix}
			0 & 0 & 0 \\
			0 & 0 & 0 \\
			0 & 0 & 0
		\end{pmatrix} $$
		Нильпотентный, с показателем 3
		\item $ A =
		\begin{pmatrix}
			0 & 0 & 0 \\
			1 & 0 & 0 \\
			0 & 0 & 2
		\end{pmatrix} $
	\end{itemize}
\end{egs}
