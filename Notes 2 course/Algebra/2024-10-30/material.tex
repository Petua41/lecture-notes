\chapter{Евклидовы и унитарные пространства}

\section{Самосопряжённый оператор}

\begin{lemma}
	$ \mc{A} $ -- самосопряжённый оператор на унитарном пространстве \\
	Тогда $ (\mc{A}x, x) \in \R \quad \forall x $
\end{lemma}

\begin{proof}
	$$ (\mc{A}x, x) \undereq{\text{самосопр.}} (x, \mc{A}^*x) = (x, \mc{A}x) $$
	$$ (\mc{A}x, x) \undereq{\text{полуторалинейность}} \ol{(x, \mc{A}x)} $$
	$$ \implies (x, \mc{A}x) \in \R \implies (\mc{A}x, x) \in \R $$
\end{proof}

\begin{definition}
	Самосопряжённый оператор назвыается положительно определённым, если $ (\mc{A}x, x) > 0 \quad \forall x \ne 0 $
\end{definition}

\begin{notation}
	$ a_i \in \bigodot \quad \iff a_1 = ... = a_n = 0 $
\end{notation}

\begin{theorem}[о собственных числах самосопряжённого оператора]
	$ \mc{A} $ -- оператор на унитарном пространстве
	\begin{enumerate}
		\item $ \mc{A} $ -- нормальный \\
		$ \mc{A} $ самоспряжённый $ \quad \iff \quad $ все с. ч. $ \mc{A} $ вещественные
		\begin{proof}
			Знаем, что существует ОНБ из с. в. \\
			Пусть $ \lambda_i $ -- с. ч. \\
			$ A, A^* $ -- матрицы $ \mc{A} $ и $ \ol{\mc{A}^*} $ в этом базисе $ \quad \implies A^* = A^T $ \\
			$ \mc{A} $ -- самосопряжённый $ \iff A = A^* \iff A = \ol{A^T} \iff $
			$$
			\begin{pmatrix}
				\lambda_1 & & 0 \\
				. & . & . \\
				0 & . & \lambda_n
			\end{pmatrix} =
			\begin{pmatrix}
				\ol{\lambda_1} & & 0 \\
				. & . & . \\
				0 & . & \ol{\lambda_n}
			\end{pmatrix}^T \quad \iff \lambda_i = \ol{\lambda_i} \quad \forall i \quad \iff \lambda_i \in \R $$
		\end{proof}
		\item $ \mc{A} $ -- самосопряжённый \\
		$ \mc{A} $ положительно определён $ \quad \iff \quad $ все с. ч. положительны
		\begin{proof}
			Пусть $ e_i $ -- ОНБ из с. в., $ \qquad \lambda_i $ -- с. ч., $ \qquad \lambda_i \in \R $ (т. к. самосопр. -- частный случай нормального) \\
			Пусть $ x = a_1e_1 + ... + a_ne_n $
			\begin{multline*}
				(\mc{A}x, x) = (a_1\lambda_1e_1 + ... + a_n\lambda_ne_n, \quad a_1e_1 + ... + a_ne_n) = \sum \lambda_ia_i\ol{a_j}\underbrace{(e_i, e_j)}_{0 \text{ или } 1} = \\
				= \sum \lambda_ia_i\ol{a_i} = \sum \lambda_i |a_i|^2 \nimp[\in \R]
			\end{multline*}
			\begin{itemize}
				\item Если $ \lambda_i > 0 \quad \forall i $, то $ \sum \underbrace{\lambda_i}_{> 0} |a_i|^2 \ge 0 $ \\
				Равенство достигается только при $ |a_i|^2 \in \bigodot $, то есть $ a_i \in \bigodot $. Значит, $ x = 0 $
				\item Пусть не все $ \lambda_i > 0, \qquad \lambda_{i_0} \le 0 $ \\
				Для $ x = e_{i_0} \quad x \ne 0, \quad (\mc{A}x, x) = \lambda_{i_0} \le 0 $ -- \contra
			\end{itemize}
		\end{proof}
	\end{enumerate}
\end{theorem}

\subsection{Переход к вещественному случаю}

\begin{lemma}\label{lm:2}
	$ A $ -- эрмитова матрица \\
	Тогда все корни $ \chi_A(t) $ вещественны
\end{lemma}

\begin{proof}
	$ A $ -- матрица порядка $ n $ \\
	Определим оператор $ \mc{A} : \Co^n $ как $ X \mapsto AX $ \\
	Тогда $ A $ -- матрица $ \mc{A} $ в стандартном базисе \\
	$ A $ -- эрмитова; станд. базис является ОНБ $ \quad \implies \mc{A} $ -- самосопряжённый \\
	Все с. ч. $ \mc{A} $ вещественны, это и есть корни $ \chi_A(t) $
\end{proof}

\begin{lemma}[ортогональность с. в.]
	$ \mc{A} $ самосопряжённый на $ \R^n, \qquad \mu, \lambda $ -- различные с. ч., $ \quad x, y $ -- соостветсвующие с. в. \\
	Тогда $ (x, y) = 0 $
\end{lemma}

\begin{remark}
	Доказательство для $ \Co^n $ здесь не пожходит
\end{remark}

\begin{proof}
	$$ \lambda(x, y) \undereq{\text{линейность}} (\lambda x, y) \undereq{\text{с. в.}} (\mc{A}x, y) \bdefeq{\mc{A}^*} (x, \mc{A}^*y) \undereq{\text{самоспр.}} (x, \mc{A}y) \undereq{\text{с. в.}} (x, \mu x) \undereq{\text{линейность}} \mu(x, y) $$
\end{proof}

\begin{theorem}
	$ \mc{A} $ -- самосопряжённый оператор на $ \R^n $ \\
	Тогда
	\begin{enumerate}
		\item $ \chi_A(t) $ раскладывается на линейные множители над $ \R $
		\begin{proof}
			Разложим $ \chi_A(t) $ на линейные множиетли над $ \Co $:
			$$ \chi_A(t) = (-1)^n(t - \lambda_1)...(t - \lambda_n), \qquad \lambda_i \in \Co $$
			Пусть $ A $ -- матрица $ \mc{A} $ в стандартном базисе $ \quad \implies A = A^T \quad \underimp{A \text{ вещ.}} A = \ol{A^T} \quad \implies $ \\
			$ \implies A $ эрмитова $ \quad \underimp{\text{лемма \ref{lm:2}}} \lambda_i \in \R \quad \forall i $
		\end{proof}
		\item Существует ОНБ $ \R^n $, состоящий из с. в. $ \mc{A} $
		\begin{proof}
			$ \chi_A(t) $ раскладывается на линейные множители над $ \R $ \\
			$ \mc{A} $ диагонализуем над $ \Co $. Есть базис из с. в. в $ \Co^n $ \\
			$ \implies $ есть базис из с. в. в $ \R^n $ (т. к. $ \R $ -- поле, там неоткуда взяться комплексным числам, а коэффициенты изначально были вещественные, \textit{надо нормально это записать}) \\
			$ \implies \dim U_{\lambda_i} = r_i $ \\
			Выберем ОНБ в каждом подпространстве $ U_{\lambda_i} $ \\
			По лемме об ортогональный с. в. объединение этих базисов -- ОНБ $ \R^n $
		\end{proof}
	\end{enumerate}
\end{theorem}

\begin{theorem}[корень из самосопряжённого оператора]
	$ \mc{A} $ -- положительно определённый самосопряжённый \\
	Тогда существует положиетльно определённый самосопряжённый $ \mc{B} : \quad \mc{A} = \mc{B}^2 $ \nimp[в смысле композиции]
\end{theorem}

\begin{proof}
	$ \mc{A} $ -- самосопряжённый $ \implies \mc{A} $ -- нормальный $ \implies \exist $ ОНБ из с. в. $ \mc{A} $ \\
	Пусть $ e_1, ..., e_n $ -- ОНБ из с. в., $ \qquad \lambda_1, ... \lambda_n $ -- с. ч. \\
	$ \mc{A} $ -- самоспряжённый $ \implies \lambda_i \in \R $ \\
	$ \mc{A} $ -- полож. опред. $ \implies \lambda_i > 0 $ \\
	Определим $ \mc{B} $ как $ \mc{B}(e_i) = \sqrt{\lambda_i}e_i $ \\
	Проверим, что он подойдёт: \\
	Рассмотрим матрицу $ \mc{B} $ в базисе $ e_1, ..., e_n $:
	$$ B =
	\begin{pmatrix}
		\sqrt{\lambda_1} & . & . \\
		. & . & . \\
		. & . & \sqrt{\lambda_n}
	\end{pmatrix} $$
	Она эрмитова $ \implies \mc{B} $ самоспряжённый \\
	$ \sqrt{\lambda_i} > 0 \implies \mc{B} $ положиетльно определён
	$$ \mc{B} \big( \mc{B}(e_i) \big) = \mc{B}(\sqrt{\lambda_i}e_i) = \lambda_ie_i = \mc{A}(e_i) \quad \forall i \quad \implies \mc{B}^2 = \mc{A} $$
\end{proof}

\begin{lemma}
	$ \mc{A} $ невырожденный \\
	Тогда $ \mc{A}\mc{A}^* $ - самосопряжённый положительно определённый
\end{lemma}

\begin{proof}
	$$ \bigg( \mc{A}\mc{A}^* \bigg)^* = \bigg( \mc{A}^* \bigg)^* \mc{A}^* = \mc{A}\mc{A}^* $$
	$$ \bigg( \mc{A}^*\mc{A}x, x \bigg) = \bigg( \mc{A}^*(\mc{A}x), x \bigg) = \bigg( \mc{A}x, (\mc{A}^*)^*x \bigg) = (\mc{A}x, \mc{A}x) \underset{\mc{A}x \ne 0 \text{, т. к. } \mc{A} \text{ невырожд.}}> 0 $$
\end{proof}

\begin{theorem}[полярное разложение оператора]
	$ \mc{A} $ -- невырожденный \nimp[(обратимый)] оператор на унитарном пространстве \\
	Тогда $ \exist \mc{U}, \mc{B} $ такие, что:
	\begin{enumerate}
		\item $ \mc{U} $ унитарный
		\item $ \mc{B} $ -- самосопряжённый положительно определённый
		\item $ \mc{A} = \mc{U}\mc{B} $
	\end{enumerate}
\end{theorem}

\begin{proof}
	$ \mc{A}\mc{A}^* $ самосопряжённый положительно определённый (по лемме). Значит
	$$ \exist \mc{B} : \quad \mc{B}^2 = \mc{A}^*\mc{A}, \qquad \mc{B} \text{ полож. опр. самосопряж.} $$
	$ \mc{A}^*, \mc{A} $ невырожденные $ \implies \mc{A}^*\mc{A} $ невырожденный $ \implies \mc{B} $ невырожденный $ \implies \exist \mc{B}^{-1} $ \\
	Положим $ \mc{U} = \mc{A}\mc{B}^{-1} $ \\
	Докажем, что эти $ \mc{U}, \mc{B} $ подойдут: \\
	Осталось проверить только унитарность, т. е. что $ \mc{U}^* \iseq \mc{U} $
	$$ \mc{U}^* = \bigg( \mc{A}\mc{B}^{-1} \bigg)^* = \bigg( \mc{B}^{-1} \bigg)^* \mc{A}^* \undereq{\text{видно из матрицы}} \bigg( \mc{B}^* \bigg)^{-1}\mc{A}^* \undereq{\mc{B} \text{ самоспр.}} \mc{B}^{-1} \mc{A}^* $$
	$$ \mc{U}^*\mc{U} = \bigg( \mc{B}^{-1}\mc{A}^* \bigg) \bigg( \mc{A}\mc{B}^{-1} \bigg) = \mc{B}^{-1} \bigg( \mc{A}^*\mc{A} \bigg)\mc{B}^{-1} = \mc{B}^{-1}\mc{B}^2\mc{B}^{-1} = \mc{E} $$
\end{proof}

\begin{implication}[перестановка сомножителей]
	$ \mc{A} $ -- невырожденный оператор \\
	Тогда $ \exist $ уинтарный $ \mc{U} $ и самосопряжённый положиетльно определённый $ \mc{B} $ такие, что $ \mc{A} = \mc{B}\mc{U} $
\end{implication}

\begin{proof}
	Применим теорему у $ \mc{A}^* $: \\
	$ \mc{A}^* = \mc{U}_1\mc{B}, \qquad \mc{U}_1 $ -- унитарный, $ \quad \mc{B} $ -- самосопряжённый пол. опред.
	$$ \mc{A} = \bigg( \mc{A}^* \bigg)^* = \bigg( \mc{B}\mc{U}_1 \bigg)^* = \mc{U}_1^* \mc{B}^* = \mc{U}_1^*\mc{B} $$
	Подойдёт $ \mc{U} = \mc{U}_1 $
\end{proof}

\section{Квадратичные формы}

\subsection{Напоминание}

$ A $ -- симметрическая матрица, $ \quad A = (a_{ij}) $ \\
Соответствующая квадратичная форма $ f(x_1, ..., x_n) = \sum a_{ij}x_ix_j $ \\
Матричная запись: $ f(x_1, ..., x_n) = X^TAX $ \\
Линейное преобразование $ X = CY \quad \implies \quad A \mapsto C^TAC $ \\
Неособое преобразование: $ C $ обратима \\
Диагональный (канонический) вид -- если $ A $ -- диагональна

\begin{theorem}[ортогональное преобразование квадратичной формы]\label{th:quad_form}
	\hfill
	\begin{enumerate}
		\item Вещественная квадратичная форма может быть приведена к диагональному виду ортогональным преобразованием
		\begin{proof}
			$ \mc{A} $ -- оператор на $ \R^n $, матрица в стандартном базисе \textit{что-то} \\
			$ \mc{A} $ самосопряжённый $ \implies \exist $ ОНБ $ T_1, ..., T_n $ из с. в. \\
			Пусть $ \lambda_1, ..., \lambda_n $ -- с. ч. \\
			Матрица $ \mc{A} $ в $ T_1, ..., T_n $ является диагональной \\
			Матрица $ \mc{A} $ в $ T_1, ..., T_n $ равна $ C^{-1}AC $, где $ C $ -- матрица перехода \\
			$ C $ состоит из столбцов $ T_i $, т. к. это матрица перехода от стандартного базиса к $ T_i $ \\
			Значит, $ C $ -- ортогональная матрица \\
			$ C^{-1}AC = C^TAC $, т. к. $ C $ ортогональна
		\end{proof}
		\item Если $ C $ -- ортогональная матрица, $ C^TAC $ -- диагональная, то на диагонали матрицы $ C^TAC $ записаны с. ч. матрицы $ A $
		\begin{proof}
			Пусть $ B = C^TAC, $ она диагональна, $ \qquad \mu_1, ..., \mu_n $ -- числа на диагонали \\
			$ S_1, ..., S_n $ -- столбцы $ B \quad \implies B = C^{-1}AC \implies B $ -- матрица $ \mc{A} $ в ОНБ $ S_1, ..., S_n \implies \mc{A}S_i = \mu_iS_i \implies \mu_i $ -- с. ч.
		\end{proof}
	\end{enumerate}
\end{theorem}

\begin{theorem}[преобразование двух форм]
	\hfill \\
	$ f(x_1, ..., x_n), \quad g(x_1, ..., x_n) $ -- вещественные квадратичные формы, $ \qquad f $ положительно определена \\
	Тогда существует неособое преобразование, при котором обе формы приводятся к диагональному виду
\end{theorem}

\begin{proof}
	Композиция неособенных преобразований -- неособенное преобразование, так что можно сделать несколько шагов:
	\begin{enumerate}
		\item Приведём $ f $ к диагональному виду $ f_1 $:
		$$ f_1(y_1, ..., y_n) = \lambda_1y_1^2 + ... + \lambda_ny_n^2, \qquad \lambda_i > 0 $$
		\item Избавимся от $ \lambda $:
		$$ z_i = \sqrt{\lambda_i}y_i $$
		$$ f_2(z_1, ..., z_n) = z_1^2 + ... + z_n^2 $$
		При этом, $ g_2 $ тоже как-то изменилась:
		$$ g_2(z_1, ..., z_n) $$
		Нужно доказать, что форму $ f_2 = z_1^2 + ... + z_n^2 $ и любую форму $ g_2 $ можно одновеременно привести к диагоналному виду
		\item Приведём $ g_2 $ к диагональному виду ортогональным перобразованием $ C $ \\
		Матрица $ f_2 $ равна $ E $
		$$ E \to C^TEC = C^TC = E $$
		Значит, $ f $ приведена к диагональному виду
	\end{enumerate}
\end{proof}

\begin{eg}[на теорему \ref{th:quad_form}]
	$$ \underbrace{2x_1^2 + x_2^@ - 4x_1x_2 - 4x_2x_3}_{Q} + 12x_1 - 8x_2 + 8x_3 + 6 = 0 $$
	Приведём $ Q $ к диагональному виду:
	$$ C =
	\begin{pmatrix}
		\faktor23 & \faktor23 & \faktor13 \\
		-\faktor23 & \faktor13 & \faktor23 \\
		\faktor13 & -\faktor23 & \faktor23
	\end{pmatrix} $$
	С. ч.: $ \lambda_1 = 4, \quad \lambda_2 = 1, \quad \lambda_3 = -2 $
	$$ 4y_1^2 + 1y_2^2 - 2y_3^2 + 12 \bigg( \frac23 y_1 + \frac23 y_2 + \frac13 y_3 \bigg) - 8 $$
	$$ 4y_1^2 + 16y_1 = 4(y_1 + 2)^2 - 16 $$
	\textit{Этот пример есть в Боревиче. Надо посмотреть}
\end{eg}
