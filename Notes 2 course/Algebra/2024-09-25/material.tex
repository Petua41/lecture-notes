\section{Многочлены от оператора}

\begin{notation}
	$ a_1, .., a_n \ne \bigodot \iff $ \textbf{не} все они равны нулю
\end{notation}

\begin{theorem}[ядро и образ многочлена от оператора]
	$ \mathcal{A} $ -- оператор на $ V $, $ p $ -- многочлен, $ \mathcal{B} = p(\mathcal{A}) $ \\
	Тогда $ \ker \mathcal{B} $ и $ \Img \mathcal{B} $ -- ивариантные подпространства относительно $ \mathcal{A} $
\end{theorem}

\begin{proof}
	По лемме,
	\begin{equ}1
		\mc{A} \circ \mc{B} = \mc{B} \circ \mc{A}
	\end{equ}
	\begin{itemize}
		\item $ \ker \mc{B} $
		$$ v \in \ker \mc{B} \implies \mc{B}(v) = 0 \implies \mc{A} \bigg( \mc{B}(v) \bigg) = 0 \underimp{\eref1} \mc{B} \bigg( \mc{A}(v) \bigg) = 0 \implies \mc{A}(v) \in \ker \mc{B} $$
		\item $ \Img \mc{B} $
		$$ v \in \Img \mc{B} \implies v = \mc{B}(w) \implies \mc{A}(v) = \mc{A} \bigg( \mc{B}(w) \bigg) \underset{\eref1}= \mc{B} \bigg( \mc{A}(w) \bigg) $$
	\end{itemize}
\end{proof}

\begin{definition}
	$ \mc{A} $ -- оператор на $ V $, $ \qquad v \in V $
	\begin{itemize}
		\item Аннулятором $ v $ называется такой многочлен $ p $, что $ p(\mc{A})(v) = 0 $
		\item Минимальным аннулятором $ v $ называется многочлен наименьшей степени среди ненулевых аннуляторов
	\end{itemize}
\end{definition}

\begin{remark}
	Минимальный аннулятор задаётся с точностью до умножения на константу
\end{remark}

\begin{exmpls}
	\item $ v $ -- с. в., соответствующие $ \lambda $ (т. е. $ \mc{A}v = \lambda v $)
	$$ \mc{B} = \mc{A} - \lambda\mc{E} \qquad \mc{B}(v) = \mc{A}(v) - \lambda \mc{E}(v) = \lambda v - \lambda v = 0 $$
	Найдём $ p $, такой что $ \mc{B} = p(\mc{A}) $:
	$$ p(t) = t - \lambda $$
	$ p(t) $ -- минимальный аннулятор
	\item $ \mc{A} : \R^2 \to \R^2 $
	$$ \mc{A} : x \mapsto Ax, \qquad A =
	\begin{pmatrix}
		2 & 0 \\
		1 & 2
	\end{pmatrix} $$
	\begin{enumerate}
		\item Докажем, что $ p(t) = (t - 2)^2 $ -- аннулятор $ \forall v $: \\
		Найдём матрицу оператора $ p(\mc{A}) = (\mc{A} - \mc{E})^2 $:
		$$ (A - 2E)^2 =
		\begin{pmatrix}
			0 & 0 \\
			1 & 0
		\end{pmatrix}^2 =
		\begin{pmatrix}
			0 & 0 \\
			0 & 0
		\end{pmatrix} $$
		\item Возьмём теперь $ Q(t) = t - 2 $ \\
		Найдём $ v $, такие что $ Q(t) $ -- аннулятор $ v $:
		$$ (\mc{A} - 2\mc{E})(v) = 0 $$
		$$ \mc{A}(v) = 2v $$
		$$
		\begin{pmatrix}
			2 & 0 \\
			1 & 2
		\end{pmatrix}
		\begin{pmatrix}
			x \\
			y
		\end{pmatrix} = 2
		\begin{pmatrix}
			x \\
			y
		\end{pmatrix} $$
		$$ v =
		\begin{pmatrix}
			0 \\
			y
		\end{pmatrix} $$
	\end{enumerate}
\end{exmpls}

\begin{props}
	\item $ V $ -- конечномерно. Тогда
	\begin{enumerate}
		\item у любого вектора существует ненулевой аннулятор
		\item если $ P_0 $ -- минимальный аннулятор, то $ \deg P_0 \le \dim V $
	\end{enumerate}
	\begin{proof}
		Пусть $ n \define \dim V $ \\
		Докажем, что $ \exist P : \deg P \le n , $ $ P $ -- аннулятор, $ P \ne 0 $ \\
		Возьмём
		$$ \underbrace{v_1\mc{A}(v), \mc{A}^2(v), ..., \mc{A}^n(v)}_{n + 1 \text{ вектор}} $$
		Они ЛЗ, т. к. их больше, чем размерность пространства. Значит,
		$$ \exist a_i \ne \bigodot : a_0v + a_1\mc{A}(v) + ... + a_n\mc{A}^n(v) = 0 $$
		Подойдёт $ P(t) = a_nt^n + ... + a_1t + a_0 $
	\end{proof}
	\item \label{prop:2} $ P_1, ..., P_k $ -- аннуляторы $ v $ \\
	Тогда
	$$ \forall \text{ многочл. } Q_1, ..., Q_k \quad \text{ многочлен } S(t) = Q_1(t)P_1(t) + ... + Q_k(t)P_k(t) \text{ -- аннулятор} $$
	\begin{proof}
		Пусть $ \mc{B}_i \define P_i(\mc{A}), \qquad \mc{C}_i = Q_i(\mc{A}), \qquad \mc{D} = S(\mc{A}) $
		$$ \mc{D}(v) = \mc{C}_1 \bigg( \underbrace{\mc{B}_1(v)}_{= 0 } \bigg) + ... + \mc{C}_k \bigg( \underbrace{\mc{B}_k(v)}_{= 0} \bigg) = \mc{C}_1(0) + ... + \mc{C}_k(0) = 0 $$
	\end{proof}
	\item $ P_0(t) $ -- минимальный аннулятор. Тогда
	$$ P(t) \text{ -- аннулятор } \iff P(t) \divby P_0(t) $$
	\begin{proof}
		Поделим с остатком:
		$$ P(t) = Q(t)P_0(t) + R(t), \qquad \deg R < \deg P_0 $$
		\begin{itemize}
			\item $ \impliedby $
			$$ R(t) = 0, \qquad P(t) = \underbrace{P_0(t)}_{\text{аннулятор}}Q(t) \text{ -- аннулятор (по (\ref{prop:2}.))} $$
			\item $ \implies $
			$$ R(t) = \underbrace{P(t)}_{\text{аннул.}} - Q(t)\underbrace{P_0(t)}_{\text{аннул.}} \text{ -- аннулятор (по (\ref{prop:2}.))} $$
		\end{itemize}
	\end{proof}
	\item Минимальный аннулятор -- единственный с точностью до ассциированности (умножения на обратимый, т. е. на константу)
	\begin{proof}
		$$ \exist P_1, P_2 \text{ -- мин. аннул. } \implies \underbrace{P_1}_{\text{аннул.}} \divby \underbrace{P_2}_{\text{мин. аннул.}} $$
	\end{proof}
\end{props}

\section{Циклические подпространства}

\begin{definition}
	$ \mc{A} $ -- оператор на $ V $, $ \qquad v \in V $ \\
	Циклическим подпространством, порождённым $ v $ называется минимальное по включению инвариантное подпространство, содержащее $ v $
\end{definition}

\begin{theorem}[базис циклического подпространства]
	$ k \in \N $ такое, что:
	\begin{enumerate}
		\item $ v, \mc{A}(v), ..., \mc{A}^{k - 1}(v) $ ЛНЗ
		\item $ v, \mc{A}(v), ..., \mc{A}^{k - 1}(v), \mc{A}^k(v) $ ЛЗ
	\end{enumerate}
	Тогда первый набор является базисом цикического подпространства, порождённого $ v $
\end{theorem}

\begin{proof}
	Пусть $ U $ -- циклическое, порождённое $ v $
	$$ U \text{ -- инвар. } \implies v \in U \implies \mc{A}v \in U \implies \underbrace{\mc{A}^2v}_{= \mc{A}(\mc{A}(v))} \in U \implies \widedots[5em] $$
	$$ v, \mc{A}v, ..., \mc{A}^{k - 1}v \in U $$
	Они ЛНЗ. Чтобы доказать, что это базис, надо доказать, что они прождают $ U $: \\
	Положим $ W = \braket{v, \mc{A}v, ..., \mc{A}^{k - 1}v} $ \\
	Докажем, что $ W = U $:
	\begin{itemize}
		\item Докажем, что $ W $ -- инвар.: \\
		$ \mc{A}^kv $ -- ЛК $ v, \mc{A}v, ..., \mc{A}^{k - 1}v $
		$$ w \in W, \qquad w = a_0v + ... + a_{k - 1}\mc{A}^{k - 1}v $$
		$$ \mc{A}(w) = a_0\mc{A}v + ... + a_{k - 2}\mc{A}^{k - 1}v + \underbrace{a_{k + 1}\mc{A}^kv}_{\text{ЛК } v, ..., \mc{A}^{k - 1}v} $$
		Значит, $ w $ является ЛК $ v, ..., \mc{A}^{k - 1}v $
		\item Докажем, что $ W $ -- минимальное: \\
		Докажем, что если $ W_1 $ инвариантно и $ v \in W_1 $, то $ W \sub W_1 $:
		\begin{multline*}
			\begin{rcases}
				W_1 \text{ инвар. } \\
				v \in W_1
			\end{rcases} \implies \mc{A}v \in W_1, \qquad
			\begin{rcases}
				W_1 \text{ инвар. } \\
				\mc{A}v \in W_1
			\end{rcases} \implies \mc{A}^2v \in W_1, \quad \widedots[3em], \quad \underbrace{\mc{A}^iv}_{\text{порожд.} W} \in W_1 \implies \\
			\implies W_1 \sub W
		\end{multline*}
	\end{itemize}
\end{proof}

\begin{eg}
	$ \mc{A} : \R^3 \to \R^3 $
	$$ \mc{A}
	\begin{pmatrix}
		x \\
		y \\
		z
	\end{pmatrix} =
	\begin{pmatrix}
		x + y \\
		y + z \\
		z
	\end{pmatrix} $$
	$$ v_1 =
	\begin{pmatrix}
		1 \\
		0 \\
		0
	\end{pmatrix}, \qquad \mc{A}(v_1) = v_1 $$
	Циклическое подпространство -- $ \left\langle
	\begin{pmatrix}
		1 \\
		0 \\
		0
	\end{pmatrix} \right\rangle $
	$$ v_2 =
	\begin{pmatrix}
		0 \\
		1 \\
		0
	\end{pmatrix}, \qquad \mc{A}(v_2) =
	\begin{pmatrix}
		1 \\
		1 \\
		0
	\end{pmatrix}, \qquad \mc{A}^2(v_2) =
	\begin{pmatrix}
		2 \\
		1 \\
		0
	\end{pmatrix} $$
	Они все лежат в плоскости $ X, Y, 0 $, а их три штуки. Значит, они ЛЗ \\
	Циклическое подпространство -- $ \braket{v_2, \mc{A}(v_2)} = \left\langle
	\begin{pmatrix}
		x \\
		y \\
		0
	\end{pmatrix} \right\rangle $
	$$ v_3 =
	\begin{pmatrix}
		0 \\
		0 \\
		1
	\end{pmatrix}, \qquad \mc{A}(v_3) =
	\begin{pmatrix}
		0 \\
		1 \\
		1
	\end{pmatrix}, \qquad \mc{A}^2(v_3) =
	\begin{pmatrix}
		1 \\
		2 \\
		1
	\end{pmatrix} $$
	Они ЛНЗ. Размерность нашего пространства -- 3, значит, если добавить четвёртый вектор, они будут ЛЗ \\
	Циклическое подпространство -- $ \R^3 $
\end{eg}

\begin{theorem}[циклическое подпространство и минимальный аннулятор]
	$ V $ -- конечномерное \\
	$ \mc{A} $ -- оператор на $ V $, $ \qquad v \in V $, $ \qquad U $ -- цикл. подпр-во, порождённое $ v $ \\
	$ \chi $ -- хар. многочлен $ \mc{A} $ на $ U $ \\
	Тогда $ \chi $ -- минимальный аннулятор $ v $
\end{theorem}

\begin{proof}
	Пусть $ k $ такое, что
	\begin{enumerate}
		\item $ v, \mc{A}v, ..., \mc{A}^{k - 1}v $ ЛНЗ
		\item $ v, \mc{A}v, ..., \mc{A}^{k - 1}v, \mc{A}^kv $ ЛЗ
	\end{enumerate}
	Путь $ a_i $, не все равные нулю, такие, что
	$$ a_0v + a_1\mc{A}v + ... + a_{k - 1}\mc{A}^{k - 1}v + a_k\mc{A}^kv = 0 $$
	Значит, $ a_k \ne 0 $ (т. к. $ v ..., \mc{A}^{k - 1}v $ ЛНЗ) \\
	Делим на $ a_k $, НУО считаея что $ a_k = 1 $:
	$$ \mc{A}^kv + ... + a_1\mc{A}v + a_0 v = 0 $$
	Положим $ P(t) \define t^k + a_{k - 1}t^{k - 1} + ... + a_1t + a_0 \implies P(t) $ -- аннулятор \\
	Докажем, что $ P(t) $ -- минимальныйю \textbf{Пусть это не так}:
	$$ \exist Q'(t) = b_mt^m + ... + t_0, \qquad Q \ne 0, \qquad Q \text{ -- аннул. }, \qquad m < k $$
	$$ b_m\mc{A}^mv + ... + b_0v = 0 $$
	$ b_i \ne \bigodot \implies \mc{A}^mv, ..., v $ -- ЛЗ -- \contra (это был не первый момент линейной зависимости) \\
	Докажем, что $ P(t) = \pm \chi $: \\
	Знаем, что
	$$ v, ..., \mc{A}^{k - 1}v \text{ -- базис } U $$
	Матрица $ \mc{A}\clamp{U} $ в этом базисе:
	$$ A =
	\begin{pmatrix}
		v & \mc{A}v & ... & \mc{A}^{k - 2}v & \mc{A}^{k - 1}v \\
		0 & 0 & ... & 0 & -a_0 \\
		1 & 0 & ... & . & -a_1 \\
		0 & 1 & ... & . & . \\
		. & . & . & . & . \\
		0 & 0 & ... & 0 & . \\
		1 & 1 & ... & 1 & -a_{k - 1}
	\end{pmatrix} $$
	В первом столбце (начиная со второй строки) -- координаты $ \mc{A}v = 0 \cdot v + 1 \cdot \mc{A}v + 0 \cdot ... $ \\
	Во втором столбце -- координаты $ \mc{A}^2v $ \\
	\widedots[10em] \\
	В последнем столбце -- координаты $ \mc{A}^kv $
	$$ \chi_A(t) =
	\begin{vmatrix}
		-t & 0 & 0 & ... & 0 & -a_0 \\
		1 & -t & 0 & ... & 0 & -a_1 \\
		. & . & . & . & . & . \\
		. & . & . & -t & 0 & -a_{k - 2} \\
		. & . & . & . & . & -a_{k - 1} - t
	\end{vmatrix} $$
	Прибавим ко 2-й строке 1-ю, умноженную на $ \faktor1t $ \\
	Прибавим к 3-й строке 2-ю, умноженную на $ \faktor1t $
	\widedots[10em]
	$$ \chi(t) =
	\begin{vmatrix}
		-t & 0 & 0 & ... & 0 & -a_1 \\
		0 & -t & 0 & ... & 0 & -a_1 - \frac{a_0}t \\
		0 & 0 & -t & ... & 0 & -a_2 - \frac{a_1}t - \frac{a_0}{t^2} \\
		. & . & . & . & . & . \\
		0 & . & . & . & 0 & t - a_{k - 1} - \frac{a_{k -2}}t - ... - \frac{a_1}{t^{k - 2}} - \frac{a_1}{t^{k - 1}}
	\end{vmatrix} $$
	Это будет $ (-1)^kP(t) $
\end{proof}

\section{Минимальный многочлен оператора}

\begin{definition}
	Многочлен $ P(t) $ аннулирует $ \mc{A} $, если $ P(\mc{A}) = 0 $
\end{definition}

\begin{remark}
	Он является аннулятором для всех векторов
\end{remark}

\begin{eg}
	$ \mc{A} : \R^2 \to \R^2 $
	$$ \mc{A} : X \mapsto
	\begin{pmatrix}
		2 & 0 \\
		1 & 2
	\end{pmatrix}X $$
	$$ P(t) = (t - 2)^2 \text{ -- аннулирует } A $$
	$ Q(t) = t - 2 $ не аннулирует $ \mc{A} $, т. к. $ \bigg( Q(\mc{A}) \bigg) =
	\begin{pmatrix}
		1 \\
		0
	\end{pmatrix} \ne
	\begin{pmatrix}
		0 \\
		0
	\end{pmatrix} $
\end{eg}

\begin{definition}
	Минимальным многочленом оператора $ \mc{A} $ называется ненулевой многочлен наименьшей степени, аннулирующий $ \mc{A} $
\end{definition}

\begin{properties}
	$ \mc{A} $ -- оператор на $ V $
	\begin{enumerate}
		\item \label{prop:1} $ P_1, ..., P_k $ аннулируют $ \mc{A} $ \\
		Тогда для любых многочленов $ Q_1, ..., Q_k $ многочлен $ S(t) = P_1(t)Q_1(t) + ... + Pk(t)Q_k(t) $ аннулирует $ \mc{A} $
		\begin{proof}
			$ \forall v \quad P_i $ -- аннулятор $ v \implies S(\mc{A}) $ -- аннулятор $ v \implies S $ аннулирует $ \mc{A} $
		\end{proof}
		\item $ P_0 $ -- минимальный многочлен для $ \mc{A} $. Тогда
		$$ P \text{ аннулирует } \mc{A} \iff P \divby P_0 $$
		\begin{proof}
			Пусть $ P = P_0Q + R $
			\begin{itemize}
				\item Если $ P \divby P_0 $, то $ P = P_0Q \underimp{\ref{prop:1}.} P \text{ аннулирует } \mc{A} $
				\item Если $ P $ аннулирует $ \mc{A} $, то $ R = P - P_0Q $ аннулирует $ \mc{A} \underimp{\ref{prop:1}.} R = 0 \implies P \divby P_0 $
			\end{itemize}
		\end{proof}
		\item Минимальный многочлен $ \mc{A} $ единственнен с точностью до ассоциирования
	\end{enumerate}
\end{properties}
