\chapter{Общий алгоритм решения}

\begin{algo}
	\item ОДЗ:
    \begin{enumerate}
        \item Школьное ОДЗ (подкоренные выражения, знаменатели, логарифмы)
        \item В зависимости от вида уравнения:
        \begin{itemize}
        	\item Если уравнение содержит $ y' $, пишем $ x \not\equiv C $
            \item Если уравнение в симметрической форме, находим особые точки (в них уравнения нет):
            $$
            \begin{cases}
            	M(x, y) = 0 \\
                N(x, y) = 0
            \end{cases} $$
        \end{itemize}
        \item Плохие границы ($ G^* $ и $ B^* $):
        \begin{itemize}
        	\item Нули множителей при производной
        \end{itemize}
        \item Хорошие границы ($ \hat{G} $ и $ \hat{B} $):
        \begin{itemize}
        	\item Нестрогие неравенства в школьном ОДЗ (равенства из них)
        \end{itemize}
    \end{enumerate}
    \item Определяем тип уравнения
    \item Решаем в соответствии с алгоритмом для нужного типа
    \item Характеризуем граничные решения:
    \begin{itemize}
    	\item Теоремы единственности (если применимы)
        \item Пусть $ y = \psi(x) $ -- граничное решение, а $ y = \vphi(x, C) $ -- общее \\
        Ищем $ C_* $ такое, что $ \forall x_* \quad \psi(x_*) = \vphi(x_*, C_*) $
        \begin{itemize}
        	\item Если $ C_* $ нашлось и конечно, то решение особое (т. к. из граничного решения в каждой точке выходит общее)
            \item Если $ C_* $ не нашлось или бесконечное, то решение частное
        \end{itemize}
    \end{itemize}
    \item Решаем ЗК
\end{algo}

\begin{undefthm}{Особые случаи}
    \begin{enumerate}
        \item Замена переменных:
        \begin{itemize}
            \item Выписываем три замены:
            \begin{enumerate}
            	\item Прямая:
                $$ x = u(x, y), \qquad y = v(x, y) $$
                \item Производная (если необходимо) или дифференциал:
                $$ x' = ..., \quad y' = ... \qquad \text{или} \qquad \di x = ..., \quad \di y = ... $$
                \item Обратная:
                $$ u = ..., \qquad v = ... $$
            \end{enumerate}
            \item Пишем ОДЗ и на прямую и на обратную замены \\
            Если оно меньше $ \vawe{G} $ или $ \vawe{B} $, то:
            \begin{itemize}
                \item Если ``отрезается'' часть $ \hat{G} $ или $ \hat{B} $ (хорошей границы), то это может быть граничным решением -- нужно проверять отдельно
                \item Если в ОДЗ входят неравенства вида $ u \succ 0 $, то см. пункт \ref{und:2}
            \end{itemize}
        \end{itemize}
        \item\label{und:2} Полуплоскости \\
        Применяется, если:
        \begin{itemize}
        	\item В ОДЗ на замену входят неравенства вида $ u(x) \succ 0 $
            \item В ходе решения получили множителем $ \sign x $
        \end{itemize}
        Порядок действий:
        \begin{enumerate}
            \item Проверяем инвариантность \textbf{исходного} уравнения относительно $ u $ или $ x $
            \begin{note}
                Если получили несколько ``знакозависимых'' переменных, то проверяем инвариантность относительно обеих сразу. Если её нет, то относительно каждой по отдельности
            \end{note}
            \begin{remark}
                Здесь надо учитывать, что $ y' = \frac{\di x}{\di y} $, т. е. $ y' $ \textbf{не} инвариантна относительно $ x $
            \end{remark}
            \begin{itemize}
            	\item Если инвариантность есть:
                \begin{enumerate}
                	\item Пишем ``Пусть $ u > 0 $'' или ``Пусть $ x > 0 $''
                    \item Решаем в этом случае
                    \item Пишем ``Сделаем замену $ x = -\vawe{x} $. Так как уравнение инвариантно относительно $ u $ (или $ x $), получим то же самое уравнение''
                    \item В ответе вместо $ x $ пишем $ |x| $ (или $ |u| $ вместо $ u $)
                \end{enumerate}
                \item Если инвариантности нет:
                \begin{itemize}
                	\item В случае $ u(x) \succ 0 $ в ОДЗ:
                    \begin{enumerate}
                    	\item Пишем ``Пусть $ u > 0 $''
                        \item Решаем в этом случае
                        \item Пишем ``Пусть $ u < 0 $''
                        \item Решаем в этом случае
                        \item В ответ попадают оба решения (каждое со своей ОДЗ)
                    \end{enumerate}
                    \item В случае $ \sign x $:
                    \begin{enumerate}
                    	\item Обозачаем $ \sigma \define \sign x $
                        \item Решаем, считая $ \sigma $ за константу
                        \item В ответе $ \sigma $ не должно быть (скорее всего, она будет множителем при $ |x| $ -- тогда просто пишем $ x $)
                    \end{enumerate}
                \end{itemize}
            \end{itemize}
        \end{enumerate}
        \item Все логарифмы собираем под один. Константу заносим туда же:
        $$ \ln x + \ln y + C = \ln(xyC) $$
        \item Чтобы корень назвать новой буквой, надо, чтобы подкоренное выражение \bt{линейно} зависело от старой переменной
        \item Чтобы проинтегрировать рациональную дробь, её нужно разложить на простейшие
    \end{enumerate}
\end{undefthm}

