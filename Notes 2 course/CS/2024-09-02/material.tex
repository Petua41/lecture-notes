\section{Введение}

Игорь Павлович Соловьёв \\
i.soloviev@spbu.ru \\
Садись ближе, говорит как Терехов \\
Пытается открыть презентацию -- 5 минут \\
Рекламирует SumatraPDF -- 10 минут \\
Настраивает микрофон -- 5 минут \\
Каждый слайд отдельным PDF \\
Будем обсуждать:
\begin{itemize}
	\item Доп. главы информатики: структуры данных, сложность выисления, ...
	\item Алгоритмы и вычислимые функции
\end{itemize}
У кого-то будет зачёт, у кого-то -- экзамен \\
Не принимает ни в каких ситуациях записи от руки. Только техи \\
Литература -- классическая (Ахо, Кнут, Кормен, Романовский, ...) \\
Читает сначала по Кормену (Алгоритмы: построение и анализ), потом -- по Ахо

\section{Начало}

Обычно различают понятия ``информация'' и ``данные'' \\
Информация активна -- её интерпретируют \\
Данные пассивны -- их кодируют

\begin{definition}
	Данные -- это код представления некоторой информации + уровень абстракции этого представления
\end{definition}

Алгоритмы и структуры данных или Представление данных (Data Representation) как дисциплина -- это раздел информатики, который изучает ...
